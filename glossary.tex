\newglossaryentry{bien-quyet-dinh}
{
    name={biên quyết định},
    description={(decision boundary)}
}

\newglossaryentry{bo-du-doan-hop-the}
{
    name={bộ dự đoán hợp thể},
    description={(ensemble predictor)}
}

\newglossaryentry{bo-hoc-bieu-dien-phan-tan}
{
    name={bộ học biểu diễn phân tán},
    description={(distributed representation learner)}
}

\newglossaryentry{bo-kiem-soat}
{
    name={bộ kiểm soát},
    description={(regularizer)}
}


\newglossaryentry{soi-doc}
{
    name={sợi dọc},
    description={(warp). Các luồng xử lí được chia thành từng nhóm nhỏ gọi là
    ``sợi dọc'' (một số người dịch là ``bó luồng''). Cái tên sợi dọc là một lối
    chơi chữ dựa trên ý tưởng sợi dọc trong dệt may là một bó sợi song song.
    Ngoài ra còn có thread block (max $512/1024$
    threads) cộng đồng dịch là ``khối luồng'', mỗi thread block gồm nhiều warps,
    mỗi warp là $32$ threads executing same instructions.}
}

\newglossaryentry{bo-phan-loai}
{
    name={bộ phân loại},
    description={(classifier)}
}

\newglossaryentry{bo-phan-loai-hoi-quy-softmax}
{
    name={bộ phân loại hồi quy softmax},
    description={(softmax regression classifier)}
}

\newglossaryentry{bo-phan-loai-nhi-phan}
{
    name={bộ phân loại nhị phân},
    description={(binary classifier)}
}

\newglossaryentry{bo-phan-loai-softmax}
{
    name={bộ phân loại softmax},
    description={(softmax classifier)}
}

\newglossaryentry{bo-phan-loai-tiep-tuyen-da-tap}
{
    name={bộ phân loại tiếp tuyến đa tạp},
    description={(manifold tangent classifier)}
}

\newglossaryentry{bo-phan-loai-tuyen-tinh}
{
    name={bộ phân loại tuyến tính},
    description={(linear classifier)}
}

\newglossaryentry{chuan-hoa-theo-lo}
{
    name={chuẩn hóa theo lô},
    description={(batch normalization)}
}

\newglossaryentry{co-che-kiem-soat}
{
    name={cơ chế kiểm soát},
    description={(regularization)}
}

\newglossaryentry{cong-quen}
{
    name={cổng quên},
    description={(forget gate)}
}

\newglossaryentry{cong-xoa}
{
    name={cổng xóa},
    description={(reset gate)}
}

\newglossaryentry{cuc-tieu-cuc-bo}
{
    name={cực tiểu cục bộ},
    description={(local minimum)}
}

\newglossaryentry{cuc-tieu-toan-cuc}
{
    name={cực tiểu toàn cục},
    description={(global minimum)}
}

\newglossaryentry{muc-phat}
{
    name={mức phạt},
    description={(penalty)}
}

\newglossaryentry{muc-phat-thua}
{
    name={mức phạt thưa},
    description={(sparsity penalty)}
}

\newglossaryentry{muc-phat-chuan}
{
    name={mức phạt chuẩn},
    description={(norm penalty)}
}

\newglossaryentry{muc-phat-tri-tuyet-doi}
{
    name={mức phạt trị tuyệt đối},
    description={(absolute value penalty)}
}

\newglossaryentry{muc-phat-co-rut}
{
    name={mức phạt co rút},
    description={(contractive penalty)}
}

\newglossaryentry{muc-phat-sai-so-khoi-phuc}
{
    name={mức phạt sai số khôi phục},
    description={(reconstruction error penalty)}
}

\newglossaryentry{sai-so-khoi-phuc}
{
    name={sai số khôi phục},
    description={(reconstruction error)}
}

\newglossaryentry{sai-so-Bayes}
{
    name={sai số Bayes},
    description={(Bayes error)}
}

\newglossaryentry{phan-phoi-khoi-phuc}
{
    name={phân phối khôi phục},
    description={(reconstruction distribution)}
}

\newglossaryentry{mat-mat-khoi-phuc}
{
    name={mất mát khôi phục},
    description={(loss reconstruction)}
}

\newglossaryentry{co-rut}
{
    name={co rút},
    description={(contractive)}
}

\newglossaryentry{go-loi}
{
    name={gỡ lỗi},
    description={(debug)}
}

\newglossaryentry{diem-yen-ngua}
{
    name={điểm yên ngựa},
    description={(saddle point)}
}

\newglossaryentry{do-chinh-xac-toan-phan}
{
    name={độ chính xác toàn phần},
    description={(accuracy) xét trong ngữ cảnh huấn luyện học máy}
}

\newglossaryentry{do-chinh-xac-rieng-phan}
{
    name={độ chính xác riêng phần},
    description={(precision) xét trong ngữ cảnh huấn luyện học máy.
        Trong các tác vụ phân loại, precision là độ chính xác đo
        theo từng nhãn, vậy nên chúng tôi gọi là độ chính xác riêng phần,
        để phân biệt rõ với \gls{do-chinh-xac-toan-phan} (accuracy) vốn
        đo trên toàn bộ mẫu kiểm thử. Thuật ngữ precision trong tiếng Anh
        khá nhập nhằng, bạn đọc cần lưu ý văn cảnh để hiểu đúng, xem thêm
        \gls{do-tu}. Khi dịch ra thuật ngữ Việt, chúng tôi cố gắng loại bỏ
        sự nhập nhằng này bằng cách tập trung dịch sát phần nội hàm kĩ thuật
        thay vì bám sát máy móc dịch từng từ thành phần trong thuật ngữ Anh.}
}

\newglossaryentry{do-do-hieu-nang}
{
    name={độ đo hiệu năng},
    description={(performance measure)}
}

\newglossaryentry{do-hop-li}
{
    name={độ hợp lí},
    description={(likelihood)}
}

\newglossaryentry{do-phu}
{
    name={độ phủ},
    description={(coverage)}
}

\newglossaryentry{do-hop-li-gia}
{
    name={độ hợp lí giả},
    description={(pseudo-likelihood)}
}

\newglossaryentry{do-hop-li-thang-log}
{
    name={độ hợp lí thang log},
    description={(log-likelihood)}
}

\newglossaryentry{do-lowij}
{
    name={lợi ích},
    description={(gain)}
}

\newglossaryentry{do-lech-chuan}
{
    name={độ lệch chuẩn},
    description={(standard deviation)}
}


\newglossaryentry{do-tu}
{
    name={độ tụ},
    description={(precision)}
}

\newglossaryentry{dung-luong}
{
    name={dung lượng},
    description={(capacity)}
}

\newglossaryentry{dung-luong-bieu-dien}
{
    name={dung lượng biểu diễn},
    description={(representational capacity)}
}

\newglossaryentry{dung-luong-hieu-dung}
{
    name={dung lượng hiệu dụng},
    description={(effective capacity)}
}

\newglossaryentry{entropy-cheo}
{
    name={entropy chéo},
    description={(cross entropy)}
}

\newglossaryentry{gia-chuan}
{
    name={giả chuẩn},
    description={(pseudonorm)}
}

\newglossaryentry{gia-nghich-dao}
{
    name={giả nghịch đảo},
    description={(pseudoinverse)}
}

\newglossaryentry{ham-chi-phi}
{
    name={hàm chi phí},
    description={(cost function)}
}

\newglossaryentry{ham-chi-phi-ngau-nhien}
{
    name={hàm chi phí ngẫu nhiên},
    description={(stochastic cost function)}
}

\newglossaryentry{ham-doi-log-hop-li}
{
    name={hàm đối log hợp lí},
    description={(negative log-likelihood)}
}

\newglossaryentry{ham-mat-mat}
{
    name={hàm mất mát},
    description={(loss function)}
}

\newglossaryentry{ham-mat-mat-thay-the}
{
    name={hàm mất mát thay thế},
    description={(surrogate loss function)}
}

\newglossaryentry{hang-tu-kiem-soat}
{
    name={hạng tử kiểm soát},
    description={(regularization term)}
}

\newglossaryentry{hoc-ban-giam-sat}
{
    name={học bán giám sát},
    description={(semi-supervised learning)}
}

\newglossaryentry{hoc-bieu-dien}
{
    name={học biểu diễn},
    description={(representation learning)}
}

\newglossaryentry{hoc-bieu-dien-phan-tan}
{
    name={học biểu diễn phân tán},
    description={(distributed representation learning)}
}

\newglossaryentry{hoc-co-giam-sat}
{
    name={học có giám sát},
    description={(supervised learning)}
}

\newglossaryentry{hoc-tang-cuong}
{
    name={học tăng cường},
    description={(reinforcement learning). Còn gọi là ``học củng cố''}
}

\newglossaryentry{hoc-hop-the}
{
    name={học hợp thể},
    description={(ensemble learning)}
}

\newglossaryentry{hoc-khong-giam-sat}
{
    name={học không giám sát},
    description={(unsupervised learning)}
}

\newglossaryentry{hoc-may}
{
    name={học máy},
    description={(machine learning)}
}

\newglossaryentry{hoc-sau}
{
    name={học sâu},
    description={(deep learning). Hiểu theo nghĩa học một
    hệ phân cấp khái niệm, mỗi phân cấp sẽ ứng với một tầng neuron.
    Chúng tôi nghĩ cách dịch ``học đa tầng'' hoặc
    ``học đa tầng khái niệm'' sẽ phản ánh rõ hơn. Nhưng vì từ
    ``học sâu'' cũng đã khá phổ biến và được chấp nhận trước đó,
    và cũng bám sát theo thuật ngữ Anh, cũng gợi tả nghĩa khá tốt,
    nên chúng tôi lựa chọn phương án này. Còn lại, các thuật ngữ khác
    có sử dụng chữ deep, chúng tôi ưu tiên dịch thành ``đa tầng'',
    tức là chúng tôi thiên về cách dịch thuật ngữ sát phần nội hàm kĩ thuật
    hơn là bám sát theo từng từ trong thuật ngữ Anh.}
}

\newglossaryentry{hoc-truc-tuyen}
{
    name={học trực tuyến},
    description={(online learning)}
}

\newglossaryentry{hoi-quy-lang-gieng-gan-nhat}
{
    name={hồi quy láng giềng gần nhất},
    description={(nearest neighbor regression)}
}

\newglossaryentry{hoi-quy-logit}
{
    name={hồi quy logit},
    description={(logistic regression)}
}

\newglossaryentry{hoi-quy-tuyen-tinh}
{
    name={hồi quy tuyến tính},
    description={(linear regression)}
}

\newglossaryentry{hop-the}
{
    name={hợp thể},
    description={(ensemble)}
}

\newglossaryentry{huan-luyen-doi-khang}
{
    name={huấn luyện đối kháng},
    description={(adversarial training)}
}

\newglossaryentry{huan-luyen-doi-khang-ao}
{
    name={huấn luyện đối kháng ảo},
    description={(virtual adversarial training)}
}

\newglossaryentry{mau-doi-khang}
{
    name={mẫu đối kháng},
    description={(adversarial example)}
}

\newglossaryentry{mau-doi-khang-ao}
{
    name={mẫu đối kháng ảo},
    description={(virtual adversarial example)}
}

\newglossaryentry{huan-luyen-truoc}
{
    name={huấn luyện trước},
    description={(pre-training)}
}

\newglossaryentry{huan-luyen-truoc-nong}
{
    name={huấn luyện trước nông},
    description={(shallow pre-training)}
}

\newglossaryentry{huan-luyen-truoc-tham-lam}
{
    name={huấn luyện trước tham lam},
    description={(greedy pre-training)}
}

\newglossaryentry{huan-luyen-truoc-co-giam-sat}
{
    name={huấn luyện trước có giám sát},
    description={(supervised pretraining)}
}

\newglossaryentry{huan-luyen-truoc-khong-giam-sat}
{
    name={huấn luyện trước không giám sát},
    description={(unsupervised pretraining)}
}

\newglossaryentry{huan-luyen-truoc-tham-lam-co-giam-sat}
{
    name={huấn luyện trước tham lam có giám sát},
    description={(greedy supervised pretraining)}
}

\newglossaryentry{huan-luyen-truoc-tham-lam-khong-giam-sat}
{
    name={huấn luyện trước tham lam không giám sát},
    description={(greedy unsupervised pretraining)}
}

\newglossaryentry{huan-luyen-truoc-tham-lam-theo-tang}
{
    name={huấn luyện trước tham lam theo tầng},
    description={(greedy layer-wise pre-training)}
}

\newglossaryentry{huan-luyen-truoc-tham-lam-theo-tang-khong-giam-sat}
{
    name={huấn luyện trước tham lam theo tầng không giám sát},
    description={(greedy layer-wise unsupervised pre-training)}
}

\newglossaryentry{huan-luyen-truoc-tham-lam-theo-tang-co-giam-sat}
{
    name={huấn luyện trước tham lam theo tầng có giám sát},
    description={(greedy layer-wise supervised pre-training)}
}

\newglossaryentry{ket-thuc-som}
{
    name={kết thúc sớm},
    description={(early stopping)}
}

\newglossaryentry{kem-dieu-hoa}
{
    name={kém điều hòa},
    description={(ill-conditioning hoặc poor-condition)}
}

\newglossaryentry{kha-phan-tuyen-tinh}
{
    name={khả phân tuyến tính},
    description={(linearly separable)}
}

\newglossaryentry{khoa-hoc-may-tinh}
{
    name={khoa học máy tính},
    description={(computer science)}
}

\newglossaryentry{khoa-hoc-than-kinh}
{
    name={khoa học thần kinh},
    description={(neuroscience)}
}

\newglossaryentry{khoi-tao-co-chuan-hoa}
{
    name={khởi tạo có chuẩn hóa},
    description={(normalized initialization)}
}

\newglossaryentry{khoi-tao-thua}
{
    name={khởi tạo thưa},
    description={(sparse initialization)}
}

\newglossaryentry{kich-thuoc-lo}
{
    name={kích thước lô},
    description={(batch size)}
}

\newglossaryentry{kiem-dinh-cheo}
{
    name={kiểm định chéo},
    description={(cross validation)}
}

\newglossaryentry{lo}
{
    name={lô},
    description={(batch)}
}

\newglossaryentry{lo-nho}
{
    name={lô nhỏ},
    description={(mini-batch)}
}

\newglossaryentry{mang-tu-hoi-quy}
{
    name={mạng tự hồi quy},
    description={(auto-recurrent network)}
}

\newglossaryentry{mang-tu-hoi-quy-neuron}
{
    name={mạng tự hồi quy neuron},
    description={(neural auto-recurrent network)}
}

\newglossaryentry{mang-neuron-tre-thoi-gian}
{
    name={mạng neuron trễ thời gian},
    description={(Time-delay neural network)}
}

\newacronym{tdnn}{TDNN}{time-delay neural network}

\newglossaryentry{mang-tich-chap}
{
    name={mạng tích chập},
    description={(convolutional network hoặc convolutional neural network)}
}

\newacronym{cnn}{CNN}{Convolutional Neural Network}

\newglossaryentry{mang-tich-chap-da-tang}
{
    name={mạng tích chập đa tầng},
    description={(deep convolutional network)}
}

\newglossaryentry{mang-tich-chap-lan-truyen-thuan}
{
    name={mạng tích chập lan truyền thuận},
    description={(feedforward convolutional network)}
}

\newglossaryentry{mang-tich-chap-truy-hoi}
{
    name={mạng tích chập truy hồi},
    description={(recurrent convolutional network)}
}

\newglossaryentry{mang-tich-chap-xep-chong}
{
    name={mạng tích chập xen kẽ},
    description={(tiled convolutional network)}
}

\newglossaryentry{mang-truy-hoi}
{
    name={mạng truy hồi},
    description={(recurrent network hoặc recurrent neural network)}
}

\newglossaryentry{mang-truy-hoi-da-tang}
{
    name={mạng truy hồi đa tầng},
    description={(deep recurrent network)}
}

\newglossaryentry{mo-hinh-hop-the}
{
    name={mô hình hợp thể},
    description={(ensemble model)}
}

\newglossaryentry{mo-hinh-phi-tham-so}
{
    name={mô hình phi tham số},
    description={(nonparametric model) Mô hình dạng này có số lượng
    tham số không cố định. Số lượng tham số của mô hình tăng dần
    theo độ lớn của dữ liệu.}
}

\newglossaryentry{mo-hinh-sinh-mau-ma-hoa-thua}
{
    name={mô hình sinh mẫu mã hóa thưa},
    description={(sparse coding generative model)}
}

\newglossaryentry{phan-cum}
{
    name={phân cụm},
    description={(clustering)}
}

\newglossaryentry{phan-da-loai}
{
    name={phân đa loại},
    description={(multi-class classification)}
}

\newglossaryentry{loai}
{
    name={loại},
    description={(class hoặc category)}
}

\newglossaryentry{phan-loai}
{
    name={phân loại},
    description={(classification)}
}

\newglossaryentry{phan-loai-hoi-quy-logit}
{
    name={phân loại hồi quy logit},
    description={(logistic regression classification)}
}

\newglossaryentry{phan-loai-nhi-phan}
{
    name={phân loại nhị phân},
    description={(binary classification)}
}

\newglossaryentry{phan-loai-tuyen-tinh}
{
    name={phân loại tuyến tính},
    description={(linear classification)}
}

\newglossaryentry{gia-tri-suy-bien}
{
    name={giá trị suy biến},
    description={(singular value)}
}

\newglossaryentry{vector-suy-bien}
{
    name={vector suy biến},
    description={(singular vector)}
}

\newglossaryentry{ma-tran-suy-bien}
{
    name={ma trận suy biến},
    description={(singular matrix) Là ma trận vuông với các cột
    phụ thuộc tuyến tính. Hình vuông gồm các số trong đó tổng các
    đường ngang, dọc, chéo đều bằng nhau gọi là ma phương
    (magic square), ma (魔) trong ma quái, phương (方) là hình vuông.
    Tương tự, trận (陣) có nghĩa xưa là bày bố hàng lối quân lính,
    ma trận (魔陣) là cách sắp xếp hàng lối (cho các con số) một
    cách ma quái ảo diệu, một từ tiếng Việt rất hay mà ta đã dùng
    để dịch chữ matrix. Ma trận có các cột phụ thuộc tuyến tính,
    tức là hạng (rank) của ma trận này sẽ thấp hơn số cột của nó,
    không gian vector nhận các cột của ma trận này làm hệ sinh
    cũng sẽ có số chiều nhỏ hơn số cột của ma trận, hiểu một cách
    nôm na là số chiều cho phép các vector ``vùng vẫy'' trong không gian
    đó cũng nhỏ hơn, vì vậy ta gọi là ``suy biến'', suy (衰) nghĩa là giảm,
    biến (變) ở đây là biến số, khả năng biến đổi, khả năng di động.}
}

\newglossaryentry{ma-tran-do-tu}
{
    name={ma trận độ tụ},
    description={(precision matrix)}
}

\newglossaryentry{phan-tich-gia-tri-suy-bien}
{
    name={phân tích giá trị suy biến},
    description={(singular value decomposition)}
}

\newacronym{svd}{SVD}{Singular Value Decomposition}

\newglossaryentry{phan-tich-rieng}
{
    name={phân tích riêng},
    description={(eigendecomposition)}
}

\newglossaryentry{phan-tich-thanh-phan-chinh}
{
    name={phân tích thành phần chính},
    description={(Principal component analysis)}
}

\newacronym{pca}{PCA}{Principal component analysis}

\newglossaryentry{phan-phoi-thuc-nghiem}
{
    name={phân phối thực nghiệm},
    description={(empirical distribution)}
}

\newglossaryentry{phan-phoi-chuan}
{
    name={phân phối chuẩn},
    description={(normal distribution)}
}

\newglossaryentry{phan-phoi-chuan-tac}
{
    name={phân phối chuẩn tắc},
    description={(standard normal distribution)}
}

\newglossaryentry{phan-phoi-chuan-da-bien}
{
    name={phân phối chuẩn đa biến},
    description={(multivariate normal distribution)}
}

\newglossaryentry{phan-phoi-sinh-du-lieu}
{
    name={phân phối sinh dữ liệu},
    description={(data-generating distribution)}
}

\newglossaryentry{phan-phoi-thuc-su}
{
    name={phân phối thực sự},
    description={(true distribution)}
}

\newglossaryentry{phan-phoi-truong-trung-binh}
{
    name={phân phối trường trung bình},
    description={(mean-field distribution)}
}

\newglossaryentry{phuong-phap-hop-the}
{
    name={phương pháp hợp thể},
    description={(ensemble method)}
}

\newglossaryentry{qua-khop}
{
    name={quá khớp},
    description={(overfitting)}
}

\newglossaryentry{rui-ro-thuc-nghiem}
{
    name={rủi ro thực nghiệm},
    description={(empirical risk)}
}

\newglossaryentry{sai-so-kiem-dinh}
{
    name={sai số kiểm định},
    description={(validation error)}
}

\newglossaryentry{sai-so-kiem-thu}
{
    name={sai số kiểm thử},
    description={(test error)}
}

\newglossaryentry{sai-so-huan-luyen}
{
    name={sai số huấn luyện},
    description={(training error)}
}

\newglossaryentry{sai-so-phan-loai}
{
    name={sai số phân loại},
    description={(classification error)}
}

\newglossaryentry{sai-so-tong-quat-hoa}
{
    name={sai số tổng quát hóa},
    description={(generalization error)}
}

\newglossaryentry{sai-so-doi-khang}
{
    name={sai số đối kháng},
    description={(adversarial error)}
}

\newglossaryentry{sieu-tham-so}
{
    name={siêu tham số},
    description={(hyperparameter)}
}

\newglossaryentry{suy-dien-bayes}
{
    name={suy diễn Bayes},
    description={(Bayesian inference)}
}

\newglossaryentry{tang-ma-hoa}
{
    name={tầng mã hóa},
    description={code layer)}
}

\newglossaryentry{tang-giai-ma}
{
    name={tầng giải mã},
    description={decode layer)}
}

\newglossaryentry{tang-an}
{
    name={tầng ẩn},
    description={(hidden) layer)}
}

\newglossaryentry{tang-kha-kien}
{
    name={tầng khả kiến},
    description={(visible layer)}
}

\newglossaryentry{tang-nhung}
{
    name={tầng nhúng},
    description={(embedding layer)}
}

\newglossaryentry{tang-phan-loai}
{
    name={tầng phân loại},
    description={(classification layer)}
}

\newglossaryentry{tap-du-lieu}
{
    name={tập dữ liệu},
    description={(dataset)}
}

\newglossaryentry{tap-kiem-dinh}
{
    name={tập kiểm định},
    description={(validation set)}
}

\newglossaryentry{tap-kiem-thu}
{
    name={tập kiểm thử},
    description={(test set hoặc test dataset)}
}

\newglossaryentry{tap-huan-luyen}
{
    name={tập huấn luyện},
    description={(training set)}
}

\newglossaryentry{thuat-toan-hoc}
{
    name={thuật toán học tập},
    description={(learning algorithm)}
}

\newglossaryentry{tich-chap-theo-lo}
{
    name={tích chập theo lô},
    description={(batch convolution)}
}

\newglossaryentry{toi-uu-cuc-bo}
{
    name={tối ưu cục bộ},
    description={(local optimum)}
}

\newglossaryentry{toi-uu-toan-cuc}
{
    name={tối ưu toàn cục},
    description={(global optimum)}
}

\newglossaryentry{tran-so-tren}
{
    name={tràn số trên},
    description={(numerical overflow). Máy tính
    chỉ có thể biểu diễn số trong một khoảng nhất định.
    Khi các tính toán cho ra kết quả quá lớn vượt quá
    khả năng biểu diễn của máy tính thì giá trị này sẽ
    được xấp xỉ thành $\infty$ hoặc $-\infty$.}
}

\newglossaryentry{tran-so-duoi}
{
    name={tràn số dưới},
    description={(numerical underflow). Về mặt toán học,
    $1$ chia cho $10$ vô hạn lần sẽ không bao giờ bằng $0$.
    Tuy vậy, máy tính chỉ dùng một lượng hữu hạn bit để mô tả
    số thập phân, vậy nên sau mỗi lần chia như vậy, con số
    khác $0$ đầu tiên sau dấu phẩy sẽ bị đẩy dần về phải,
    cho đến khi ``tràn'' ra ngoài khoảng có thể biểu diễn
    bằng số bit hữu hạn của máy tính thì khi đó máy tính
    sẽ lưu trữ con số đó đúng bằng $0$. Vì lí do này,
    chúng tôi chọn cách dịch là ``tràn số dưới''.}
}

\newglossaryentry{tri-rieng}
{
    name={trị riêng},
    description={(eigenvalue). Xem \gls{vector-rieng}}
}

\newglossaryentry{trung-binh-binh-phuong-sai-so}
{
    name={trung bình bình phương sai số},
    description={(mean squared error)}
}

\newacronym{mse}{MSE}{mean squared error}

\newglossaryentry{truot-gradient}
{
    name={trượt gradient},
    description={(gradient descent). Gradient cũng chỉ có
    nghĩa giản dị là ``con dốc, độ dốc''. Đây là một thuật ngữ
    gợi tả hình ảnh địa hình hàm (function landscape) khi ta
    vẽ nó lên đồ thị. Do đó, thuật ngữ gradient descent
    hoàn toàn có thể dịch một cách chuẩn xác và gợi hình
    cũng rất tốt là ``trượt dốc'', tương tự gradient ascent
    sẽ là ``leo dốc''. Bởi vì từ grandient đã quá phổ biến
    trong cộng đồng vật lí lẫn toán học xưa nay, nên chúng
    tôi chấp nhận dịch thành ``trượt gradient, leo gradient''.
    Tuy vậy, bạn đọc có thể thoải mái sử dụng các cách dịch trên
    tùy văn cảnh}
}

\newglossaryentry{truot-gradient-ngau-nhien}
{
    name={trượt gradient ngẫu nhiên},
    description={(stochastic gradient descent)}
}

\newglossaryentry{truot-gradient-phan-tan-bat-dong-bo}
{
    name={trượt gradient phân tán bất đồng bộ},
    description={(distributed asynchronous gradient descent)}
}

\newacronym{sgd}{SGD}{stochastic gradient descent}

\newglossaryentry{truot-gradient-ngau-nhien-khong-dong-bo}
{
    name={trượt gradient ngẫu nhiên bất đồng bộ},
    description={(asynchronous stochastic gradient descent)}
}

\newglossaryentry{truot-gradient-ngau-nhien-theo-lo}
{
    name={trượt gradient ngẫu nhiên theo lô},
    description={(batch stochastic gradient descent)}
}

\newglossaryentry{truot-gradient-ngau-nhien-theo-lo-nho}
{
    name={trượt gradient ngẫu nhiên theo lô nhỏ},
    description={(minibatch stochastic gradient descent)}
}

\newglossaryentry{truot-gradient-theo-lo}
{
    name={trượt gradient theo lô},
    description={(batch gradient descent)}
}

\newglossaryentry{truot-gradient-theo-lo-nho}
{
    name={trượt gradient theo lô nhỏ},
    description={(minibatch gradient descent)}
}

\newglossaryentry{truot-theo-toa-do}
{
    name={trượt theo tọa độ},
    description={(coordinate descent)}
}

\newglossaryentry{truot-theo-khoi-toa-do}
{
    name={trượt theo khối tọa độ},
    description={(block coordinate descent)}
}

\newglossaryentry{uoc-luong-hop-li-cuc-dai}
{
    name={ước lượng hợp lí cực đại},
    description={(maximum likelihood estimation)}
}

\newglossaryentry{uoc-luong-tan-suat}
{
    name={ước lượng tần suất},
    description={(frequentist estimator)}
}

\newglossaryentry{uoc-luong-khong-chech}
{
    name={ước lượng không chệch},
    description={(unbiased estimator)}
}

\newglossaryentry{uoc-luong-chech}
{
    name={ước lượng chệch},
    description={(biased estimator)}
}

\newglossaryentry{vi-khop}
{
    name={vị khớp},
    description={(underfitting) có nghĩa là chưa đủ khớp.
    Vị (未) ở đây có nghĩa là chưa đủ, tương tự như ``vị''
    trong ``vị thành niên'' (chưa đủ tuổi), ``vị hôn phu/thê''
    (chồng/vợ chưa cưới). Thuật ngữ này cũng có thể dịch là
    ``chưa khớp'', nhưng chúng tôi xét thấy từ ``chưa khớp''
    có thể gây nhập nhằng ở nhiều văn cảnh khác nên chúng tôi
    thiên về phương án ``vị khớp'' để tạo đủ sự khác biệt
    cho một thuật ngữ. Ngoài ra, xin đừng dịch under thành ``thấp''
    ghép lại thành ``thấp khớp''. Thấp khớp là thuật ngữ y học
    phổ biến của một loại bệnh.}
}

\newglossaryentry{vector-rieng}
{
    name={vector riêng},
    description={(eigenvector). Từ ``eigen'' vốn là tiếng Đức,
    có nghĩa là ``riêng, của riêng''. Eigenvector là vector
    vẫn giữ được hướng của có khi bị biến đổi tuyến tính bởi T
    nào đó. Thuật ngữ eigenvector và một từ lai Đức-Anh. Ngoài ra,
    vector thực ra lại là một từ thuần latin, tức là gốc gác
    cũng không phải tiếng Anh! Nếu xét ở phiên bản Anh hóa hơn,
    thì nó nên là principle-vector, hoặc characteristic-vector.
    Chính vì vậy, chúng tôi chọn cách dịch là vector riêng.
    Tương tự với \gls{tri-rieng}.}
}

\newglossaryentry{nhom-nut-day-du}
{
    name={nhóm nút đầy đủ},
    description={(clique). Clique là từ cổ gốc Pháp, có nghĩa là phe nhóm,
    bè lũ. Người ta chọn lấy một chữ có nghĩa lạ làm thuật ngữ.
    Bám sát theo dịch là phe, hay bè lũ đều không hợp trong nhiều
    văn cảnh diễn đạt tiếng Việt. Do vậy chúng tôi canh
    theo nội hàm kĩ thuật mạnh dạn dịch từ này thành
    ``nhóm nút kết nối đầy đủ'', gọi tắt là ``nhóm nút đầy đủ''.}
}

\newglossaryentry{the-nang-nhom-nut-day-du}
{
    name={thế năng nhóm nút đầy đủ},
    description={(clique potential)}
}

\newglossaryentry{mo-hinh-vo-huong}
{
    name={mô hình vô hướng},
    description={(undirected model)}
}

\newglossaryentry{mo-hinh-co-huong}
{
    name={mô hình có hướng},
    description={(directed model)}
}

\newglossaryentry{phan-ham}
{
    name={hàm phân hoạch},
    description={(partition function). Đây là thuật ngữ vay mượn
    từ vật lí thống kê. Trong vật lí thống kê, người ta đã dịch là
    ``hàm trạng thái'' hoặc ``hàm trạng thái thống kê'' hoặc
    ``hàm phân bố''. Một partition mô tả cách n hạt phân bổ giữa
    $k$ mức năng lượng. Partition function mô tả đặc tính thống kê
    của hệ nhiệt động lực học trong trạng thái cân bằng, và đây là
    một hàm vô hướng. Chữ partition trong tên gọi có thể do
    hàm này liên hệ tới cách mà các hạt phân bổ giữa các mức năng
    lượng khác nhau. Về mặt toán học, thuật ngữ này đã dịch là
    ``hàm phân hoạch''.}
}

\newglossaryentry{mo-hinh-nang-luong}
{
    name={mô hình năng lượng},
    description={(energy-based model)}
}

\newglossaryentry{nang-luong-tu-do}
{
    name={năng lượng tự do},
    description={(free energy)}
}

\newglossaryentry{nang-luong-tu-do-bien-phan}
{
    name={năng lượng tự do biến phân},
    description={(variational free energy)}
}

\newglossaryentry{ham-nang-luong}
{
    name={hàm năng lượng},
    description={(energy-funtion)}
}

\newglossaryentry{toi-uu-loi}
{
    name={tối ưu lồi},
    description={(convex optimization)}
}

\newglossaryentry{toi-uu-so}
{
    name={tối ưu số},
    description={(numerical optimization)}
}

\newglossaryentry{chuan-gradient}
{
    name={chuẩn gradient},
    description={(gradient norm)}
}

\newglossaryentry{ham-loivex}
{
    name={hàm lồi},
    description={(convex function)}
}

\newglossaryentry{bieu-do-tan-xa}
{
    name={biểu đồ tán xạ},
    description={(scatter-plot) Biểu diễn dữ liệu bằng
    đồ thị gồm nhiều điểm, trong đó mỗi điểm ứng với
    giá trị quan sát được của một biến so với giá trị
    tương ứng của biến kia mà không nối các điểm đó
    lại với nhau.}
}

\newglossaryentry{phat-hien-vat-the}
{
    name={phát hiện vật thể},
    description={(object detection)}
}

\newglossaryentry{luot-huan-luyen}
{
    name={lượt huấn luyện},
    description={(epoch)}
}

\newglossaryentry{bien-tiem-an}
{
    name={biến tiềm ẩn},
    description={(latent variable)}
}

% https://web.stanford.edu/class/archive/ee/ee392m/ee392m.1034/Lecture8_ID.pdf
\newglossaryentry{kha-dinh}
{
    name={khả định},
    description={(identifiability)}
}

\newglossaryentry{bat-kha-dinh}
{
    name={bất khả định},
    description={(nonidentifiability)}
}

\newglossaryentry{doi-xung-khong-gian-trong-so}
{
    name={đối xứng không gian trọng số},
    description={(weight space symmetry)}
}

\newglossaryentry{tuyen-tinh-hieu-chinh}
{
    name={tuyến tính hiệu chỉnh},
    description={(rectified linear)}
}

\newglossaryentry{cuc-dai-dau-ra}
{
    name={cực đại đầu ra},
    description={(maxout)}
}

\newglossaryentry{don-vi-cuc-dai-dau-ra}
{
    name={đơn vị cực đại đầu ra},
    description={(maxout unit)}
}

\newglossaryentry{don-vi-tuyen-tinh-hieu-chinh}
{
    name={đơn vị tuyến tính hiệu chỉnh},
    description={(rectified linear unit)}
}

\newglossaryentry{don-vi-an}
{
    name={đơn vị ẩn},
    description={(hidden unit)}
}

\newglossaryentry{don-vi-dodetect}
{
    name={đơn vị dò},
    description={(detector unit)}
}

\newglossaryentry{don-vi-dodetect-nhi-phan}
{
    name={đơn vị dò nhị phân},
    description={(binary detector unit)}
}

\newglossaryentry{don-vi-gop}
{
    name={đơn vị gộp},
    description={(pooling unit)}
}

\newglossaryentry{don-vi-gop-cuc-dai}
{
    name={đơn vị gộp cực đại},
    description={(max-pooling unit)}
}

\newglossaryentry{don-vi-gop-nhi-phan}
{
    name={đơn vị gộp nhị phân},
    description={(binary pooling unit)}
}

\newglossaryentry{cao-nguyen}
{
    name={cao nguyên},
    description={(plateaus)}
}

\newglossaryentry{cuc-dai-cuc-bo}
{
    name={cực đại cục bộ},
    description={(local maximum)}
}

\newglossaryentry{xu-ly-ngon-ngu-tu-nhien}
{
    name={xử lí ngôn ngữ tự nhiên},
    description={(natural language processing - NLP)}
}

\newglossaryentry{phuong-phap-Newton-thoat-diem-yen-ngua}
{
    name={phương pháp Newton thoát điểm yên ngựa},
    description={(saddle-free Newton method)}
}

\newglossaryentry{diem-cuc-tieu}
{
    name={điểm cực tiểu},
    description={(minimum point)}
}

\newglossaryentry{ham-muc-tieu}
{
    name={hàm mục tiêu},
    description={(objective function)}
}

\newglossaryentry{siet-gradient}
{
    name={siết gradient},
    description={(gradient clipping)}
}

\newglossaryentry{siet-chuan-gradient}
{
    name={siết chuẩn gradient},
    description={(gradient norm clipping)}
}

\newglossaryentry{siet-gradient-cam-tinh}
{
    name={siết gradient cảm tính},
    description={(gradient clipping heuristic)}
}

\newglossaryentry{phu-thuoc-dai-han}
{
    name={phụ thuộc dài hạn},
    description={(long-term dependency)}
}

\newglossaryentry{tieu-bien-gradient}
{
    name={tiêu biến gradient},
    description={(vanishing gradient)}
}

\newglossaryentry{bung-no-gradient}
{
    name={bùng nổ gradient},
    description={(exploding gradient)}
}

\newglossaryentry{mo-hinh-do-thi-co-huong}
{
    name={mô hình đồ thị có hướng},
    description={(directed graphical model)}
}

\newglossaryentry{mo-hinh-do-thi-vo-huong}
{
    name={mô hình đồ thị vô hướng},
    description={(undirected graphical model)}
}

\newglossaryentry{mang-Markov}
{
    name={mạng Markov},
    description={(Markov network)}
}

\newglossaryentry{mo-hinh-xac-suat}
{
    name={mô hình xác suất},
    description={(probabilistic model)}
}

\newglossaryentry{mo-hinh-xac-suat-ma-hoa-thua-truc-tiep}
{
    name={mô hình xác suất mã hóa thưa trực tiếp},
    description={(directed sparse coding probabilistic model)}
}

\newglossaryentry{mo-hinh-xac-suat-da-tang}
{
    name={mô hình xác suất đa tầng},
    description={(deep probabilistic model)}
}

\newglossaryentry{mo-hinh-xac-suat-co-cau-truc}
{
    name={mô hình xác suất có cấu trúc},
    description={(structured probabilistic model)}
}

\newglossaryentry{mo-hinh-xac-suat-co-huong}
{
    name={mô hình xác suất có hướng},
    description={(directed probabilistic model)}
}

\newglossaryentry{mo-hinh-xac-suat-vo-huong}
{
    name={mô hình xác suất vô hướng},
    description={(undirected probabilistic model)}
}

\newglossaryentry{mo-hinh-do-thi}
{
    name={mô hình đồ thị},
    description={(graphical model)}
}
\newglossaryentry{mo-hinh-do-thi-nong}
{
    name={mô hình đồ thị nông},
    description={(shallow graphical model)}
}

\newglossaryentry{mo-hinh-do-thi-da-tang}
{
    name={mô hình đồ thị đa tầng},
    description={(deep graphical model)}
}

\newglossaryentry{mo-hinh-do-thi-trai-qua-phai}
{
    name={mô hình đồ thị trái-qua-phải},
    description={(left-to-right graphical model)}
}

\newglossaryentry{mo-hinh-do-thi-lai}
{
    name={mô hình đồ thị lai},
    description={(hybrid graphical model)}
}

\newglossaryentry{do-thi-dan-trai}
{
    name={đồ thị dàn trải},
    description={(unfold graph)}
}

\newglossaryentry{mo-hinh-do-thi-co-cau-truc}
{
    name={mô hình đồ thị có cấu trúc},
    description={(structured graphical model)}
}

\newglossaryentry{mo-hinh-do-thi-xac-suat}
{
    name={mô hình đồ thị xác suất},
    description={(probabilistic graphical model)}
}

\newglossaryentry{mo-hinh-do-thi-xac-suat-vo-huong}
{
    name={mô hình đồ thị xác suất vô hướng},
    description={(undirected probabilistic graphical model)}
}

\newglossaryentry{mo-hinh-do-thi-xac-suat-da-tang}
{
    name={mô hình đồ thị xác suất đa tầng},
    description={(deep probabilistic graphical model)}
}

\newglossaryentry{do-thi-vo-huong}
{
    name={đồ thị vô hướng},
    description={(undirected graph)}
}

\newglossaryentry{do-thi-co-huong}
{
    name={đồ thị có hướng},
    description={(directed graph)}
}

\newglossaryentry{do-thi-co-huong-phi-chu-trinh}
{
    name={đồ thị có hướng phi chu trình},
    description={(directed acyclic graph)}
}

\newglossaryentry{phan-phoi-xac-suat}
{
    name={phân phối xác suất},
    description={(probability distribution)}
}

\newglossaryentry{phan-phoi-xac-suat-tien-nghiem}
{
    name={phân phối xác suất tiên nghiệm},
    description={(prior probability distribution)}
}

\newglossaryentry{phan-phoi-hon-hop}
{
    name={phân phối hỗn hợp},
    description={(mixture distribution)}
}

\newglossaryentry{phan-phoi-xac-suat-bien}
{
    name={phân phối xác suất biên},
    description={(marginal probability distribution)}
}

\newglossaryentry{phan-phoi-xac-suat-dong-thoi}
{
    name={phân phối xác suất đồng thời},
    description={(joint probability distribution)}
}

\newglossaryentry{ham-mat-do-xac-suat}
{
    name={hàm mật độ xác suất},
    description={(probability density function)}
}

\newglossaryentry{phan-phoi-xac-suat-chua-chuan-hoa}
{
    name={phân phối xác suất chưa chuẩn hóa},
    description={(unnormalized probability distribution)}
}

\newglossaryentry{phan-phoi-xac-suat-chuan-hoa}
{
    name={phân phối xác suất chuẩn hóa},
    description={(normalized probability distribution)}
}

\newglossaryentry{n-gram}
{
    name={$n$-gram},
    description={($n$-gram)}
}

\newglossaryentry{hoi-cap}
{
    name={hồi cấp},
    description={(back-off)}
}

\newglossaryentry{n-gram-hoi-cap}
{
    name={$n$-gram hồi cấp},
    description={(back-off $n$-gram)}
}

\newglossaryentry{mang-lan-truyen-thuan}
{
    name={mạng lan truyền thuận},
    description={(feedforward network)}
}

\newglossaryentry{mang-lan-truyen-thuan-da-tang}
{
    name={mạng lan truyền thuận đa tầng},
    description={(feedforward deep network)}
}

\newglossaryentry{phan-ki-tuong-phan}
{
    name={phân kì tương phản},
    description={(contrastive divergence)}
}

\newacronym{cd}{CD}{contrastive divergence}

\newglossaryentry{phan-ki-tuong-phan-lien-tuc}
{
    name={phân kì tương phản liên tục},
    description={(persistent contrastive divergence)}
}

\newacronym{pcd}{PCD}{persistent contrastive divergence}

\newacronym{fpcd}{FPCD}{fast persistent contrastive divergence}

\newglossaryentry{may-boltzmann}
{
    name={máy Boltzmann},
    description={(Boltzmann machine)}
}

\newglossaryentry{may-boltzmann-ket-noi-day-du}
{
    name={máy Boltzmann kết nối đầy đủ},
    description={(fully connected Boltzmann machine)}
}

\newglossaryentry{may-boltzmann-nhi-phan}
{
    name={máy Boltzmann nhị phân},
    description={(binary Boltzmann machine)}
}

\newglossaryentry{may-boltzmann-da-tang}
{
    name={máy Boltzmann đa tầng},
    description={(deep Boltzmann machine)}
}

\newacronym{dbm}{DBM}{deep Boltzmann machine}

\newglossaryentry{may-boltzmann-da-tang-dinh-tam}
{
    name={máy Boltzmann đa tầng định tâm},
    description={(centered deep Boltzmann machine)}
}

\newglossaryentry{may-boltzmann-da-tang-da-du-doan}
{
    name={máy Boltzmann đa tầng đa dự đoán},
    description={(multi-prediction deep Boltzmann machine)}
}

\newacronym{mp-dbm}{MP-DBM}{multi-prediction deep Boltzmann machine}

\newglossaryentry{may-boltzmann-gioi-han}
{
    name={máy Boltzmann giới hạn},
    description={(restricted Boltzmann machine)}
}

\newacronym{rbm}{RBM}{restricted Boltzmann machine}

\newacronym{ssrbm}{ssRBM}{spike and slab restricted Boltzmann machine}

\newacronym{mcrbm}{mcRBM}{mean and covariance restricted Boltzmann machine}

\newacronym{crbm}{cRBM}{covariance restricted Boltzmann machine}

\newglossaryentry{may-boltzmann-ban-gioi-han}
{
    name={máy Boltzmann bán giới hạn},
    description={(semi-restricted Boltzmann machine)}
}

\newglossaryentry{may-boltzmann-tich-chap}
{
    name={máy Boltzmann tích chập},
    description={(convolutional Boltzmann machine)}
}

\newglossaryentry{cau-truc-cuc-bo}
{
    name={cấu trúc cục bộ},
    description={(local structure)}
}

\newglossaryentry{cau-truc-toan-cuc}
{
    name={cấu trúc toàn cục},
    description={(global structure)}
}

\newglossaryentry{diem-toi-han}
{
    name={điểm tới hạn},
    description={(critical point)}
}

\newglossaryentry{phan-phoi-xac-suat-co-dieu-kien}
{
    name={phân phối xác suất có điều kiện},
    description={(conditional probability distribution)}
}

\newglossaryentry{phan-phoi-xac-suat-co-dieu-kien-cuc-bo}
{
    name={phân phối xác suất có điều kiện cục bộ},
    description={(local conditional probability distribution)}
}

\newglossaryentry{cau-truc-vo-luan-li}
{
    name={cấu trúc vô luân lí},
    description={(immorality)}
}

\newglossaryentry{do-thi-luan-li}
{
    name={đồ thị luân lí},
    description={(moralized graph)}
}

\newglossaryentry{do-thi-nhan-tu}
{
    name={đồ thị nhân tử},
    description={(factor graph)}
}

\newglossaryentry{lay-mau-pha-he}
{
    name={lấy mẫu phả hệ},
    description={(ancestral sampling)}
}

\newglossaryentry{phan-phoi-dong-thoi}
{
    name={phân phối đồng thời},
    description={(joint distribution)}
}

\newglossaryentry{thu-tu-cau-truc-to-po}
{
    name={thứ tự cấu trúc tô-pô},
    description={(topological ordering)}
}

\newglossaryentry{phan-phoi-hau-nghiem}
{
    name={phân phối hậu nghiệm},
    description={(posterior distribution)}
}

\newglossaryentry{phan-phoi-tien-nghiem}
{
    name={phân phối tiên nghiệm},
    description={(prior distribution)}
}

\newglossaryentry{phan-phoi-bien}
{
    name={phân phối biên},
    description={(marginal distribution)}
}

\newglossaryentry{lay-mau-Gibbs}
{
    name={lấy mẫu Gibbs},
    description={(Gibbs sampling)}
}

\newglossaryentry{lay-mau-Gibbs-theo-khoi}
{
    name={lấy mẫu Gibbs theo khối},
    description={(block Gibbs sampling)}
}

\newglossaryentry{phan-phoi-giai-thua}
{
    name={phân phối nhân tử},
    description={(factorial distribution)}
}

\newglossaryentry{phan-phoi-Gibbs}
{
    name={phân phối Gibbs},
    description={(Gibbs distribution)}
}

\newglossaryentry{phan-phoi-Boltzmann}
{
    name={phân phối Boltzmann},
    description={(Boltzmann distribution)}
}

\newglossaryentry{mo-hinh-log-tuyen-tinh}
{
    name={mô hình log-tuyến tính},
    description={(log-linear model)}
}

\newglossaryentry{mo-hinh-sinh-mau}
{
    name={mô hình sinh mẫu},
    description={(generative model)}
}
\newglossaryentry{mo-hinh-sinh-mau-da-tang}
{
    name={mô hình sinh mẫu đa tầng},
    description={(deep generative model)}
}

\newglossaryentry{mo-hinh-sinh-mau-nong-co-huong}
{
    name={mô hình sinh mẫu nông có hướng},
    description={(shallow directed generative model)}
}

\newglossaryentry{mang-sinh-mau}
{
    name={mạng sinh mẫu},
    description={(generative network)}
}

\newglossaryentry{mang-sinh-mau-tich-chap}
{
    name={mạng sinh mẫu tích chập},
    description={(convolutional generative network)}
}

\newglossaryentry{mang-sinh-mau-co-huong}
{
    name={mạng sinh mẫu có hướng},
    description={(directed generative network)}
}

\newglossaryentry{mang-sinh-mau-tien-doan}
{
    name={mạng sinh mẫu tiên đoán},
    description={(predictive generative network)}
}

\newglossaryentry{mang-sinh-mau-kha-vi}
{
    name={mạng sinh mẫu khả vi},
    description={(differentiable generative network)}
}

\newglossaryentry{mang-sinh-mau-ngau-nhien}
{
    name={mạng sinh mẫu ngẫu nhiên},
    description={(stochastic generative network)}
}

\newacronym{gsn}{GSN}{stochastic generative network}

\newglossaryentry{mang-sinh-mau-khop-moment}
{
    name={mạng sinh mẫu khớp moment},
    description={(generative moment matching network)}
}

\newglossaryentry{khop-moment}
{
    name={khớp moment},
    description={(moment matching)}
}

\newglossaryentry{mang-Bayes}
{
    name={mạng Bayes},
    description={(Bayesian network)}
}
\newglossaryentry{mang-Bayes-dong}
{
    name={mạng Bayes động},
    description={(dynamic Bayesian network)}
}

\newglossaryentry{mang-Bayes-hoan-toan-kha-kien}
{
    name={mạng Bayes hoàn toàn khả kiến},
    description={(fully-visible Bayes network)}
}

\newglossaryentry{mo-hinh-Gauss}
{
    name={mô hình Gauss},
    description={(Gaussian model)}
}

\newglossaryentry{mo-hinh-Gauss-hon-hop}
{
    name={mô hình Gauss hỗn hợp},
    description={(Gaussian mixture model)}
}

\newglossaryentry{bien-an}
{
    name={biến ẩn},
    description={(hidden variable)}
}

\newglossaryentry{bien-kha-kien}
{
    name={biến khả kiến},
    description={(visible variable)}
}

\newglossaryentry{cay-quy-dan}
{
    name={cây quy dẫn},
    description={(reduction tree)}
}

\newglossaryentry{hoc-da-tap}
{
    name={học đa tạp},
    description={(manifold learning)}
}

\newglossaryentry{gia-thuyet-da-tap}
{
    name={giả thuyết đa tạp},
    description={(manifold hypothesis)}
}

\newglossaryentry{hoc-da-tap-phi-tham-so}
{
    name={học đa tạp phi tham số},
    description={(nonparametric manifold learning)}
}

\newglossaryentry{da-tap}
{
    name={đa tạp},
    description={(manifold)}
}

\newglossaryentry{phan-giai-tu-co-nghia-nhap-nhang}
{
    name={phân giải từ có nghĩa nhập nhằng},
    description={(word-sense disambiguation)}
}

\newglossaryentry{bieu-dien-phan-tan}
{
    name={biểu diễn phân tán},
    description={(distributed representation)}
}

\newglossaryentry{bieu-dien-phan-tan-da-tang}
{
    name={biểu diễn phân tán đa tầng},
    description={(deep distributed representation)}
}

\newglossaryentry{thuat-toan-suy-dien}
{
    name={thuật toán suy diễn},
    description={(inference algorithm)}
}

\newglossaryentry{thuat-toan-suy-dien-xap-xi}
{
    name={thuật toán suy diễn xấp xỉ},
    description={(approximate inference algorithm)}
}

\newglossaryentry{thuat-toan-suy-dien-bien-phan}
{
    name={thuật toán suy diễn biến phân},
    description={(variational inference algorithm)}
}

\newglossaryentry{truyen-ba-niem-tin-theo-vong}
{
    name={truyền bá niềm tin theo vòng},
    description={(loopy belief propagation)}
}

\newglossaryentry{ma-tran-cheo}
{
    name={ma trận chéo},
    description={(diagonal matrix)}
}

\newglossaryentry{u-lay-mau-theo-do-quan-trong}
{
    name={ủ lấy mẫu theo độ quan trọng},
    description={(annealed importance sampling)}
}

\newacronym{ais}{AIS}{annealed importance sampling}

\newglossaryentry{lay-mau-theo-do-quan-trong}
{
    name={lấy mẫu theo độ quan trọng},
    description={(importance sampling)}
}

\newglossaryentry{lay-mau-theo-do-quan-trong-co-chech}
{
    name={lấy mẫu theo độ quan trọng có chệch},
    description={(biased importance sampling)}
}

\newglossaryentry{lay-mau-theo-do-quan-trong-toi-uu}
{
    name={lấy mẫu theo độ quan trọng tối ưu},
    description={(optimal importance sampling)}
}

\newglossaryentry{do-chech}
{
    name={độ chệch},
    description={(bias) trong thống kê}
}

\newglossaryentry{chech}
{
    name={chệch},
    description={(biased)}
}

\newglossaryentry{he-so-tu-do}
{
    name={hệ số tự do},
    description={(bias) trong giải tích}
}

\newglossaryentry{he-so-chan}
{
    name={hệ số chặn},
    description={(intercept)}
}

\newglossaryentry{khong-chech}
{
    name={không chệch},
    description={(unbiased)}
}

\newglossaryentry{khong-chech-tiem-can}
{
    name={không chệch tiệm cận},
    description={(asymptotically unbiased)}
}

\newglossaryentry{toc-do-hoc}
{
    name={tốc độ học},
    description={(learning rate)}
}

\newglossaryentry{bo-uoc-luong}
{
    name={bộ ước lượng},
    description={(estimator)}
}

\newglossaryentry{dieu-kien-du}
{
    name={điều kiện đủ},
    description={(sufficient condition)}
}

\newglossaryentry{dieu-kien-can}
{
    name={điều kiện cần},
    description={(necessary condition)}
}

\newglossaryentry{hoi-tu}
{
    name={hội tụ},
    description={(convergence)}
}

\newglossaryentry{hau-nghiem-cuc-dai}
{
    name={hậu nghiệm cực đại},
    description={(maximum a posteriori)}
}

\newacronym{map}{MAP}{maximum a posteriori}

\newglossaryentry{xap-xi-hau-nghiem-cuc-dai}
{
    name={xấp xỉ hậu nghiệm cực đại},
    description={(maximum a posteriori approximation)}
}

\newglossaryentry{toc-do-hoi-tu}
{
    name={tốc độ hội tụ},
    description={(convergence rate)}
}

\newglossaryentry{thu-sai}
{
    name={thử sai},
    description={(trial and error)}
}

\newglossaryentry{duong-cong-hoc-tap}
{
    name={đường cong học tập},
    description={(learning curve)}
}

\newglossaryentry{tat-ngau-nhien}
{
    name={tắt ngẫu nhiên},
    description={(dropout) - hoặc có thể gọi đầy đủ là ``cơ chế tắt ngẫu nhiên''}
}

\newglossaryentry{tat-ngau-nhien-nhanh}
{
    name={tắt ngẫu nhiên nhanh},
    description={(fast dropout) - hoặc có thể gọi đầy đủ là ``cơ chế tắt ngẫu nhiên nhanh''}
}

\newglossaryentry{tat-ngau-nhien-tang-cuong}
{
    name={tắt ngẫu nhiên tăng cường},
    description={(dropout boosting) - hoặc có thể gọi đầy đủ là ``cơ chế tắt ngẫu nhiên tăng cường''}
}

\newglossaryentry{truc-tuyen}
{
    name={trực tuyến},
    description={(online). Thời điện thoại còn sử dụng dây điện, người Việt mình khi
    bắt máy hay hỏi ``ai ở đầu dây bên kia vậy ạ''. Thời nay dùng mạng không dây,
    ta không còn nghe thấy câu đó phổ biến nữa. Cũng cùng một nghĩa đó,
    tiếng Anh dùng online, tức là on-the-line (trên dây) đồng nghĩa với việc
    người kia đang có mặt và đang sẵn sàng kết nối trao đổi thông tin. Tuyến (線)
    nghĩa là sợi dây, trực (値) nghĩa là có mặt, hiện diện. Trực tuyến tức là có mặt
    trên dây, đem dịch chữ online là rất hay. Ngoài ra, theo nghĩa trên, online
    (và cả từ trực tuyến) cũng hàm ý liên tục kết nối và chia sẻ thông tin, vây nên
    xét trong ngữ cảnh học sâu, khi mà mô hình học liên tục nhận mẫu huấn luyện
    trong suốt thời gian vận hành, thì người ta gọi đó là các
    phương pháp học trực tuyến (online learning method) xem ra rất hợp lí.}
}

\newglossaryentry{dung-sai}
{
    name={dung sai},
    description={(tolerance)}
}

\newglossaryentry{sai-so-vuot-muc}
{
    name={sai số vượt mức},
    description={(excess error)}
}

\newglossaryentry{giai-tich-tiem-can}
{
    name={phân tích tiệm cận},
    description={(asymptotic analysis)}
}

\newglossaryentry{dong-luong}
{
    name={động lượng},
    description={(momentum)}
}

\newglossaryentry{duong-dong-muc}
{
    name={đường đồng mức},
    description={(contour line)}
}

\newglossaryentry{phuong-trinh-vi-phan}
{
    name={phương trình vi phân},
    description={(differential equation)}
}

\newglossaryentry{suc-can-nhot}
{
    name={sức cản nhớt},
    description={(viscous drag)}
}

\newglossaryentry{moi-truong-co-tinh-can}
{
    name={môi trường có tính cản},
    description={(resistant medium)}
}

\newglossaryentry{suy-giam-trong-so}
{
    name={suy giảm trọng số},
    description={(weight decay)}
}

\newglossaryentry{thong-ke-can-bien}
{
    name={thống kê cận biên},
    description={(marginal statistics)}
}

\newglossaryentry{khong-gian-Euclid}
{
    name={không gian Euclid},
    description={(Euclidean space)}
}

\newglossaryentry{khong-gian-nhung-tu}
{
    name={không gian nhúng từ},
    description={(word embedding space)}
}

\newglossaryentry{khong-gian-dac-trung}
{
    name={không gian đặc trưng},
    description={(feature space)}
}

\newglossaryentry{mang-niem-tin}
{
    name={mạng niềm tin},
    description={(belief network)}
}

\newglossaryentry{mang-niem-tin-da-tang}
{
    name={mạng niềm tin đa tầng},
    description={(deep belief network)}
}

\newacronym{dbn}{BDN}{deep belief network}

\newglossaryentry{mang-niem-tin-hoan-toan-kha-kien}
{
    name={mạng niềm tin hoàn toàn khả kiến},
    description={(fully visible belief network)}
}

\newglossaryentry{mang-niem-tin-sigmoid}
{
    name={mạng niềm tin sigmoid},
    description={(sigmoid belief network)}
}

\newglossaryentry{mang-niem-tin-tich-chap-da-tang}
{
    name={mạng niềm tin tích chập đa tầng},
    description={(convolutional deep belief network)}
}

\newglossaryentry{tich-cua-cac-chuyen-gia}
{
    name={tích của các chuyên gia},
    description={(product of experts)}
}

\newglossaryentry{phep-chieu-so-chieu-thap}
{
    name={phép chiếu số chiều thấp},
    description={(low-dimentional projection)}
}

\newglossaryentry{hai-hoa}
{
    name={hài hòa},
    description={(harmony)}
}

\newglossaryentry{hieu-ung-thanh-minh}
{
    name={hiệu ứng thanh minh},
    description={(explaining away effect)}
}

\newglossaryentry{tuong-tac-thanh-minh}
{
    name={tương tác thanh minh},
    description={(explaining away interaction)}
}

\newglossaryentry{cau-truc-V}
{
    name={cấu trúc V},
    description={(V structure)}
}

\newglossaryentry{day-cung}
{
    name={dây cung},
    description={(chord)}
}

\newglossaryentry{do-thi-day-cung}
{
    name={đồ thị dây cung},
    description={(choral graph)}
}

\newglossaryentry{do-thi-tam-giac}
{
    name={đồ thị tam giác},
    description={(triangulated graph)}
}

\newglossaryentry{pha-duong}
{
    name={pha dương},
    description={(positive phase)}
}

\newglossaryentry{pha-am}
{
    name={pha âm},
    description={(negative phase)}
}

\newglossaryentry{mode-gia-mao}
{
    name={mode giả mạo},
    description={(spurious mode)}
}

\newglossaryentry{khoang-cach-Levenshtein}
{
    name={khoảng cách Levenshtein},
    description={(Levenshtein distance). Khoảng cách Levenshtein (hay ``edit distance'') thể hiện khoảng
    cách khác biệt giữa $2$ chuỗi kí tự. Khoảng cách Levenshtein giữa chuỗi $\mathrm{S}$ và
    chuỗi $\mathrm{T}$ là số bước ít nhất để biến chuỗi $\mathrm{S}$ thành chuỗi $\mathrm{T}$
    thông qua $3$ thao tác là: chèn kí tự, xoá kí tự, và thay kí tự này bằng kí tự khác.}
}

\newglossaryentry{hop-li-cuc-dai-ngau-nhien}
{
    name={hợp lí cực đại ngẫu nhiên},
    description={(stochastic maximum likelihood)}
}

\newacronym{sml}{SML}{stochastic maximum likelihood}

\newglossaryentry{hop-li-cuc-dai-ngau-nhien-bien-phan}
{
    name={hợp lí cực đại ngẫu nhiên biến phân},
    description={(variational stochastic maximum likelihood)}
}

\newglossaryentry{bo-tu-ma-hoa}
{
    name={bộ tự mã hóa},
    description={(autoencoder)}
}

\newglossaryentry{bo-tu-ma-hoa-bien-phan}
{
    name={bộ tự mã hóa biến phân},
    description={(variational autoencoder)}
}

\newacronym{vae}{VAE}{variational autoencoder}

\newglossaryentry{bo-tu-ma-hoa-phan-tan}
{
    name={bộ tự mã hóa phân tán},
    description={(distributed autoencoder)}
}

\newglossaryentry{bo-tu-ma-hoa-khu-nhieu}
{
    name={bộ tự mã hóa khử nhiễu},
    description={(denoising autoencoder)}
}

\newacronym{dae}{DAE}{denoising autoencoder}

\newglossaryentry{bo-tu-ma-hoa-co-rut}
{
    name={bộ tự mã hóa co rút},
    description={(contractive autoencoder)}
}

\newacronym{cae}{CAE}{contractive autoencoder}

\newglossaryentry{bo-tu-ma-hoa-co-trong-so}
{
    name={bộ tự mã hóa có trọng số},
    description={(importance-weighted autoencoder)}
}

\newglossaryentry{bo-tu-ma-hoa-ngau-nhien}
{
    name={bộ tự mã hóa ngẫu nhiên},
    description={(stochastic autoencoder)}
}

\newglossaryentry{bo-tu-ma-hoa-da-tang}
{
    name={bộ tự mã hóa đa tầng},
    description={(deep autoencoder)}
}

\newglossaryentry{bo-tu-ma-hoa-nong}
{
    name={bộ tự mã hóa nông},
    description={(shallow autoencoder)}
}

\newglossaryentry{bo-tu-ma-hoa-duoi-muc}
{
    name={bộ tự mã hóa dưới mức},
    description={(undercomplete autoencoder)}
}

\newglossaryentry{bo-tu-ma-hoa-thua}
{
    name={bộ tự mã hóa thưa},
    description={(sparse autoencoder)}
}

\newglossaryentry{bo-tu-ma-hoa-co-kiem-soat}
{
    name={bộ tự mã hóa có kiểm soát},
    description={(Regularized autoencoder)}
}

\newglossaryentry{bo-ma-hoa}
{
    name={bộ mã hóa},
    description={(encoder)}
}

\newglossaryentry{bo-ma-hoa-phi-tham-so}
{
    name={bộ mã hóa phi tham số},
    description={(nonparametric encoder)}
}

\newglossaryentry{bo-ma-hoa-truy-hoi}
{
    name={bộ mã hóa truy hồi},
    description={(recurrent encoder)}
}

\newglossaryentry{bo-ma-hoa-ngau-nhien}
{
    name={bộ mã hóa ngẫu nhiên},
    description={(stochastic encoder)}
}

\newglossaryentry{bo-giai-ma}
{
    name={bộ giải mã},
    description={(decoder)}
}

\newglossaryentry{bo-giai-ma-ngau-nhien}
{
    name={bộ giải mã ngẫu nhiên},
    description={(stochastic decoder)}
}

\newglossaryentry{bo-giai-ma-truy-hoi}
{
    name={bộ giải mã truy hồi},
    description={(recurrent decoder)}
}

\newglossaryentry{khop-diem-so}
{
    name={khớp điểm số},
    description={(score matching)}
}

\newglossaryentry{khop-diem-so-khu-nhieu}
{
    name={khớp điểm số khử nhiễu},
    description={(denoising score matching)}
}

\newglossaryentry{chuoi-Markov}
{
    name={chuỗi Markov},
    description={(Markov chain)}
}

\newglossaryentry{chuoi-Markov-Monte-Carlo}
{
    name={chuỗi Markov Monte Carlo},
    description={(Markov chain Monte Carlo)}
}

\newglossaryentry{gradient-lien-hop}
{
    name={gradient liên hợp},
    description={(conjugate gradient)}
}

\newglossaryentry{khop-ti-le}
{
    name={khớp tỉ lệ},
    description={(ratio matching)}
}

\newglossaryentry{gradient-lien-hop-chia-ti-le}
{
    name={gradient liên hợp chia tỉ lệ},
    description={(scaled conjugate gradient - SCG)}
}

\newglossaryentry{huong-lien-hop}
{
    name={hướng liên hợp},
    description={(conjugate direction)}
}

\newglossaryentry{do-duong}
{
    name={dò đường},
    description={(line search)}
}

\newacronym{bfgs}{BFGS}{Broyden--Fletcher--Goldfarb--Shanno}

\newglossaryentry{BFGS-huu-han-bo-nho}
{
    name={BFGS giới hạn bộ nhớ},
    description={(Limited Memory BFGS - L-BFGS)}
}

\newacronym{lbfgs}{L-BFGS}{Limited Memory Broyden--Fletcher--Goldfarb--Shanno}

\newglossaryentry{xac-suat-bien}
{
    name={xác suất biên},
    description={(marginal probability)}
}

\newglossaryentry{tai-tham-so-hoa-tuy-bien}
{
    name={tái tham số hóa thích nghi},
    description={(adaptive reparameterization)}
}

\newglossaryentry{tai-tham-so-hoa}
{
    name={tái tham số hóa},
    description={(reparameterization)}
}

\newglossaryentry{phat-tan}
{
    name={phát tán},
    description={(broadcasting) Trong \gls{hoc-sau}, người ta sử dụng thêm một số kí hiệu không phổ biến
    theo thông lệ. Cụ thể, sách này cho phép thực hiện cộng giữa một ma
    trận và một vector, tạo ra một ma trận khác: $\boldsymbol{C} =
    \boldsymbol{A} + \boldsymbol{b}$, trong đó $C_{i,j} = A_{i,j} + b_j$.
    Nói cách khác, vector $\boldsymbol {b}$ được cộng vào từng hàng của ma
    trận $\boldsymbol{A}$. Sử dụng kí hiệu này, ta không cần phải tạo ra
    một ma trận mới, với mỗi cột là một bản sao của $\boldsymbol b$, trước
    khi thực hiện phép cộng. Kiểu sao chép vector $\boldsymbol b$ ngầm
    định tới nhiều vị trí như thế này được gọi là phép \textit{\gls{phat-tan}}
    (broadcasting)}
}

\newglossaryentry{ban-do-dac-trung}
{
    name={bản đồ đặc trưng},
    description={(feature map)}
}

\newglossaryentry{ma-hoa-thua}
{
    name={mã hóa thưa},
    description={(sparse coding)}
}

\newglossaryentry{ma-hoa-thua-nhi-phan}
{
    name={mã hóa thưa nhị phân},
    description={(binary sparse coding)}
}

\newglossaryentry{hon-hop-mot-xung}
{
    name={hỗn hợp một xung},
    description={(spike và slab). Xem \gls{xung-nhon} (spike) và \gls{xung-det} (slab)}
}

\newglossaryentry{xung-nhon}
{
    name={xung nhọn},
    description={(spike)}
}

\newglossaryentry{xung-det}
{
    name={xung dẹt},
    description={(slab)}
}

\newglossaryentry{khoang-cach-Hamming}
{
    name={khoảng cách Hamming}, description={(Hamming distance). Trong lí thuyết
    thông tin, khoảng cách Hamming giữa hai chuỗi có độ dài bằng nhau là số các
    kí tự ở vị trí tương đương có giá trị khác nhau. Nói cách khác, khoảng cách
    Hamming đo số lượng thay thế cần phải có để đổi giá trị của một dãy kí tự
    sang một dãy kí tự khác.}
}

\newglossaryentry{tinh-chinh}
{
    name={tinh chỉnh},
    description={(fine-tuning)}
}

\newglossaryentry{thuat-toan-tham-lam}
{
    name={thuật toán tham lam},
    description={(greedy algorithm)}
}

\newglossaryentry{perceptron-da-tang}
{
    name={perceptron đa tầng},
    description={(multilayer perceptron)}
}

\newacronym{mlp}{MLP}{multilayer perceptron}

\newglossaryentry{hoc-chuyen-giao}
{
    name={học chuyển giao},
    description={(transfer learning)}
}

\newglossaryentry{khong-don-dieu}
{
    name={không đơn điệu},
    description={(nonmonotonic)}
}

\newglossaryentry{don-vi-sigmoid}
{
    name={đơn vị sigmoid},
    description={(sigmoid unit)}
}

\newglossaryentry{don-vi-ro-ri}
{
    name={đơn vị rò rỉ},
    description={(leaky unit)}
}

\newglossaryentry{don-vi-kha-kien}
{
    name={đơn vị khả kiến},
    description={(visible unit)}
}

\newglossaryentry{ket-noi-nhay-coc}
{
    name={kết nối nhảy cóc},
    description={(skip connection)}
}

\newglossaryentry{ket-noi-nhay-coc-xuyen-thoi-gian}
{
    name={kết nối nhảy cóc xuyên thời gian},
    description={(Skip Connections through Time)}
}

\newglossaryentry{dau-bo-tro}
{
    name={đầu bổ trợ},
    description={(auxiliary head)}
}

\newglossaryentry{phuong-phap-min-dan}
{
    name={phương pháp mịn dần},
    description={(continuation method)}
}

\newglossaryentry{toi-luyen-mo-phong}
{
    name={tôi luyện mô phỏng},
    description={(simulated annealing)}
}

\newglossaryentry{ngan-xep}
{
    name={ngăn xếp},
    description={(stack)}
}

\newglossaryentry{ngan-xep-loi}
{
    name={ngăn xếp lõi},
    description={(kernel stack)}
}

\newglossaryentry{hoc-theo-giao-trinh}
{
    name={học theo giáo trình},
    description={(curriculum learning)}
}

\newglossaryentry{tao-khuon}
{
    name={tạo khuôn},
    description={(shaping)}
}

\newglossaryentry{uoc-luong-tuong-phan-nhieu}
{
    name={ước lượng tương phản nhiễu},
    description={(noise-contrastive estimation)}
}

\newacronym{nce}{NCE}{noise-contrastive estimation}

\newglossaryentry{uoc-luong-tu-tuong-phan}
{
    name={ước lượng tự tương phản},
    description={(self-contrastive estimation)}
}

\newglossaryentry{phan-phoi-nhieu}
{
    name={phân phối nhiễu},
    description={(noise distribution)}
}

\newglossaryentry{mang-doi-khang-sinh-mau}
{
    name={mạng đối kháng sinh mẫu},
    description={(generative adversarial network)}
}

\newacronym{gan}{GAN}{generative adversarial network}

\newacronym{dcgan}{DCGAN}{deep convolutional generative adversarial network}

\newacronym{lapgan}{LAPGAN}{Laplacian generative adversarial network}

\newglossaryentry{lay-mau-bac-cau}
{
    name={lấy mẫu bắc cầu},
    description={(bridge sampling)}
}

\newglossaryentry{tuong-cuop}
{
    name={tướng cướp},
    description={(bandit). Máy đánh bạc $1$ cần gạt là loại thường thấy trong sòng bài (casino). Người chơi đa phần sẽ thua tiền, và vì vậy nên họ gọi nó là ``đồ ăn cướp'', bandit có nghĩa là ``tướng cướp''. Cũng giống như xúc xắc thường được dùng làm ví dụ cho các bài toán xác suất thống kê cơ bản, thì ở đây người ta thường dùng bandit làm ví dụ cơ bản cho các bài toán \gls{hoc-tang-cuong}.}
}

\newglossaryentry{tuong-cuop-ngu-canh}
{
    name={tướng cướp ngữ cảnh},
    description={(contextual bandit). Xem \gls{tuong-cuop}}
}

\newglossaryentry{bieu-dien-dung-chung}
{
    name={biểu diễn dùng chung},
    description={(shared representation)}
}

\newglossaryentry{cay-do-den}
{
    name={cây đỏ đen},
    description={(red-black tree)}
}

\newglossaryentry{bieu-dien-da-tang}
{
    name={biểu diễn đa tầng},
    description={(deep representation)}
}

\newglossaryentry{can-duoi-thuc-nghiem}
{
    name={cận dưới thực nghiệm},
    description={(evidence lower bound)}
}

\newacronym{elbo}{ELBO}{evidence lower bound}

\newglossaryentry{day-ep-buoc}
{
    name={dạy ép buộc},
    description={(teacher forcing)}
}

\newglossaryentry{may-turing-pho-quat}
{
    name={máy Turing phổ quát},
    description={(universal Turing machine)}
}

\newglossaryentry{may-turing}
{
    name={máy Turing},
    description={(Turing machine)}
}

\newglossaryentry{RBM-phan-biet}
{
    name={RBM phân biệt},
    description={(discriminative RBM)}
}

\newglossaryentry{mang-bac-thang}
{
    name={mạng bậc thang},
    description={(ladder network)}
}

\newglossaryentry{ma-tran-Hesse}
{
    name={ma trận Hesse},
    description={(Hessian matrix)}
}

\newglossaryentry{ma-tran-Jacobi}
{
    name={ma trận Jacobi},
    description={(Jacobi matrix)}
}

\newglossaryentry{nhung-tu}
{
    name={nhúng từ},
    description={(word embedding)}
}

\newglossaryentry{vector-nhung-tu}
{
    name={vector nhúng từ},
    description={(word embedding vector)}
}

\newglossaryentry{vector-don-troi}
{
    name={vector đơn trội},
    description={(one-hot vector)}
}

\newglossaryentry{ma-don-troi}
{
    name={mã đơn trội},
    description={(one-hot code)}
}

\newglossaryentry{bieu-dien-don-troi}
{
    name={biểu diễn đơn trội},
    description={(one-hot representation)}
}

\newglossaryentry{don-troi}
{
    name={đơn trội},
    description={(one-hot)}
}

\newglossaryentry{thich-ung-mien}
{
    name={thích ứng miền},
    description={(domain adaptation)}
}

\newglossaryentry{hoc-da-nhiem}
{
    name={học đa nhiệm},
    description={(multitask learning)}
}

\newglossaryentry{chuyen-dich-khai-niem}
{
    name={chuyển dịch khái niệm},
    description={(concept drift)}
}

\newglossaryentry{truong-trung-binh}
{
    name={trường trung bình},
    description={(mean field)}
}

\newglossaryentry{xap-xi-truong-trung-binh}
{
    name={xấp xỉ trường trung bình},
    description={(mean field) hay ``phép xấp xỉ trường trung bình''}
}

\newglossaryentry{suy-dien-truong-trung-binh}
{
    name={suy diễn trường trung bình},
    description={(mean-field inference)}
}

\newglossaryentry{suy-dien-truy-hoi-truong-trung-binh}
{
    name={suy diễn truy hồi trường trung bình},
    description={(mean-field recurrent inference)}
}

\newglossaryentry{suy-dien-hau-nghiem-cuc-dai}
{
    name={suy diễn hậu nghiệm cực đại},
    description={(maximization a posteriori inference - MAP inference)}
}

\newglossaryentry{uoc-luong-hau-nghiem-cuc-dai}
{
    name={ước lượng hậu nghiệm cực đại},
    description={(maximization a posteriori estimation)}
}

\newglossaryentry{truong-ngau-nhien-Markov}
{
    name={trường ngẫu nhiên Markov},
    description={(Markov random field)}
}

\newacronym{mrf}{MRF}{Markov random field}

\newglossaryentry{ham-loiker}
{
    name={hàm lõi},
    description={(kernel function)}
}

\newglossaryentry{may-loiker}
{
    name={máy lõi},
    description={(kernel machine) hoặc có thể gọi là ``mô hình hàm lõi''}
}

\newglossaryentry{thu-thuat-chon-ham-loiker}
{
    name={thủ thuật chọn hàm lõi},
    description={(kernel trick)}
}

\newglossaryentry{hoc-one-shot}
{
    name={học 1-mẫu},
    description={(one-shot learning)}
}

\newglossaryentry{hoc-zero-shot}
{
    name={học 0-mẫu},
    description={(zero-shot learning)}
}

\newglossaryentry{hoc-phi-du-lieu}
{
    name={học phi dữ liệu},
    description={(zero-data learning)}
}

\newglossaryentry{hoc-da-phuong-thuc}
{
    name={học đa phương thức},
    description={(multimodal learning)}
}

\newglossaryentry{phan-tu-tuyen-tinh-thich-nghi}
{
    name={phần tử tuyến tính thích nghi},
    description={(adaptive linear element)}
}

\newacronym{adaline}{ADALINE}{adaptive linear element}

\newglossaryentry{mang-trang-thai-vong-hoi}
{
    name={mạng trạng thái vọng hồi},
    description={(echo state network)}
}

\newacronym{esn}{ESN}{echo state network}

\newglossaryentry{dieu-khien-hoc}
{
    name={điều khiển học},
    description={(cybernetics)}
}

\newglossaryentry{thuyet-ket-noi}
{
    name={thuyết kết nối},
    description={(connectionism)}
}

\newglossaryentry{xu-ly-phan-tan-song-song}
{
    name={xử lí phân tán song song},
    description={(parallel distributed processing)}
}

\newglossaryentry{suy-luan-bieu-tuong}
{
    name={suy luận biểu tượng},
    description={(symbolic reasoning)}
}

\newglossaryentry{bien-to}
{
    name={biến tố},
    description={(factor of variation)}
}

\newglossaryentry{suy-dien-bien-phan}
{
    name={suy diễn biến phân},
    description={(variational inference)}
}

\newglossaryentry{hoc-bien-phan}
{
    name={học biến phân},
    description={(variational learning)}
}

\newglossaryentry{suy-dien-bien-phan-co-cau-truc}
{
    name={suy diễn biến phân có cấu trúc},
    description={(structured variational inference)}
}

\newglossaryentry{tri-tue-nhan-tao}
{
    name={trí tuệ nhân tạo},
    description={(artificial intelligence)}
}

\newglossaryentry{gian-luan-Bayes}
{
    name={giản luận Bayes},
    description={(naive Bayes)}
}

\newglossaryentry{bo-nho-ngan-han-huong-dai-han}
{
    name={bộ nhớ ngắn hạn hướng dài hạn},
    description={(long short-term memory)}
}

\newacronym{lstm}{LSTM}{long short-term memory}

\newglossaryentry{chuoi-chuoi}
{
    name={chuỗi-chuỗi},
    description={(sequence-to-sequence)}
}

\newglossaryentry{dich-may}
{
    name={dịch máy},
    description={(machine translation)}
}

\newglossaryentry{ma-hoa-giai-ma}
{
    name={mã hoá-giải mã},
    description={(encoder-decoder)}
}

\newglossaryentry{mo-hinh-hon-hop}
{
    name={mô hình hỗn hợp},
    description={(mixture model)}
}

\newglossaryentry{mo-hinh-hon-hop-Gauss}
{
    name={mô hình hỗn hợp Gauss},
    description={(Gaussian mixture model)}
}

\newglossaryentry{quy-tac-Bayes}
{
    name={quy tắc Bayes},
    description={(Bayes' rule)}
}

\newglossaryentry{thu-nho-mau}
{
    name={thu nhỏ mẫu},
    description={(downsampling). Ngược lại sẽ là \gls{phong-to-mau}}
}

\newglossaryentry{phong-to-mau}
{
    name={phóng to mẫu},
    description={(upsampling)}
}

\newglossaryentry{phep-gop}
{
    name={phép gộp},
    description={(pooling)}
}

\newglossaryentry{phep-gop-cuc-dai}
{
    name={phép gộp cực đại},
    description={(max pooling)}
}

\newglossaryentry{phep-toan-gop-cuc-dai}
{
    name={phép toán gộp cực đại},
    description={(max pooling operation)}
}

\newglossaryentry{ham-gop-cuc-dai}
{
    name={hàm gộp cực đại},
    description={(max pooling function)}
}

\newglossaryentry{phep-gop-cuc-dai-xac-suat}
{
    name={phép gộp cực đại xác suất},
    description={(probabilistic max pooling)}
}

\newglossaryentry{phep-gop-trung-binh}
{
    name={phép gộp trung bình},
    description={(average pooling)}
}

\newglossaryentry{phep-gop-ngau-nhien}
{
    name={phép gộp ngẫu nhiên},
    description={(stochastic pooling)}
}

\newglossaryentry{dac-trung-nhi-phan}
{
    name={đặc trưng nhị phân},
    description={(binary feature)}
}

\newglossaryentry{bo-do-dac-trung}
{
    name={bộ dò đặc trưng},
    description={(feature detector)}
}

\newglossaryentry{bo-trich-xuat-dac-trung}
{
    name={bộ trích xuất đặc trưng},
    description={(feature extractor)}
}

\newglossaryentry{bo-trich-xuat-dac-trung-cham-phi-tuyen-da-tang}
{
    name={bộ trích xuất đặc trưng chậm phi tuyến đa tầng},
    description={(deep nonlinear slow featature extractor)}
}

\newglossaryentry{xap-xi-pho-quat}
{
    name={xấp xỉ phổ quát},
    description={(universal approximation)}
}

\newglossaryentry{dinh-li-xap-xi-pho-quat}
{
    name={định lí xấp xỉ phổ quát},
    description={(universal approximation theorem)}
}

\newglossaryentry{bo-xap-xi-pho-quat}
{
    name={bộ xấp xỉ phổ quát},
    description={(universal approximator)}
}

\newglossaryentry{mang-tong-tich}
{
    name={mạng tổng-tích},
    description={(sum-product network - SPN)}
}

\newacronym{spn}{SPN}{sum-product network}

\newglossaryentry{ham-co-so-xuyen-tam}
{
    name={hàm cơ sở xuyên tâm},
    description={(radial basis function)}
}

\newacronym{rbf}{RBF}{radial basis function}

\newglossaryentry{mang-ham-co-so-xuyen-tam}
{
    name={mạng hàm cơ sở xuyên tâm},
    description={(radial basis function network)}
}

\newglossaryentry{phan-tich-dac-trung-cham}
{
    name={phân tích đặc trưng chậm},
    description={(slow feature analysis)}
}

\newacronym{sfa}{SFA}{slow feature analysis}

\newglossaryentry{hieu-trung-binh-binh-phuong}
{
    name={hiệu trung bình bình phương},
    description={(mean squared difference)}
}

\newglossaryentry{mach-da-thuc}
{
    name={mạch đa thức},
    description={(polynomial circuit)}
}

\newglossaryentry{mach-da-tang}
{
    name={mạch đa tầng},
    description={(deep circuit)}
}

\newglossaryentry{thong-tin-tuong-ho}
{
    name={thông tin tương hỗ},
    description={(mutual information)}
}

\newglossaryentry{lan-truyen-tiep-tuyen}
{
    name={lan truyền tiếp tuyến},
    description={(tangent propagation)}
}

\newglossaryentry{khoang-cach-tiep-tuyen}
{
    name={khoảng cách tiếp tuyến},
    description={(tangent distance)}
}

\newglossaryentry{lan-truyen-nguoc-kep}
{
    name={lan truyền ngược kép},
    description={(double backprop)}
}

\newglossaryentry{lan-truyen-nguoc-thoi-gian}
{
    name={lan truyền ngược thời gian},
    description={(back-propagation through time)}
}

\newacronym{bptt}{BPTT}{back-propagation through time}

\newglossaryentry{lan-truyen-nguoc}
{
    name={lan truyền ngược},
    description={(back-propagation)}
}

\newglossaryentry{mang-nguong-tuyen-tinh-da-tang}
{
    name={mạng ngưỡng tuyến tính đa tầng},
    description={(deep linear-threshold network)}
}

\newglossaryentry{can-duoi-bien-phan}
{
    name={cận dưới biến phân},
    description={(variational lower bound)}
}

\newglossaryentry{phuong-trinh-diem-co-dinh}
{
    name={phương trình điểm cố định},
    description={(fixed-point equation)}
}

\newglossaryentry{phiem-ham}
{
    name={phiếm hàm},
    description={(functional)}
}

\newglossaryentry{phiem-ham-chi-phi}
{
    name={phiếm hàm chi phí},
    description={(cost functional)}
}

\newglossaryentry{trung-binh-dong}
{
    name={trung bình động},
    description={(moving average hoặc running average)}
}

\newglossaryentry{nhan-dang-tieng-noi}
{
    name={nhận dạng tiếng nói},
    description={(speech recognition)}
}

\newglossaryentry{nhan-dang-vat-the}
{
    name={nhận dạng vật thể},
    description={(object recognition)}
}

\newglossaryentry{co-so-tri-thuc}
{
    name={cơ sở tri thức},
    description={(knowledge base)}
}

\newglossaryentry{he-phan-cap-khai-niem}
{
    name={hệ phân cấp khái niệm},
    description={(hierarchy of concepts)}
}

\newglossaryentry{bieu-dien}
{
    name={biểu diễn},
    description={(representation)}
}

\newglossaryentry{mo-thuc}
{
    name={mô thức},
    description={(pattern)}
}

\newglossaryentry{dac-trung}
{
    name={đặc trưng},
    description={(feature)}
}

\newglossaryentry{diem-anh}
{
    name={điểm ảnh},
    description={(pixel)}
}

\newglossaryentry{boc-tach}
{
    name={bóc tách},
    description={(disentangle)}
}

\newglossaryentry{luu-do}
{
    name={lưu đồ},
    description={(flow chart)}
}

\newglossaryentry{tang-bieu-dien}
{
    name={tầng biểu diễn},
    description={(representation layer)}
}

\newglossaryentry{phuong-phap-tinh}
{
    name={phương pháp tính},
    description={(calculus)}
}

\newglossaryentry{phuong-phap-tinh-bien-phan}
{
    name={phương pháp tính biến phân},
    description={(calculus of variations)}
}

\newglossaryentry{mang-neuron-nhan-tao}
{
    name={mạng neuron nhân tạo},
    description={(artificial neuron network - ANN)}
}

\newglossaryentry{mang-cong-chan}
{
    name={mạng cổng chặn},
    description={(gater (network))}
}

\newglossaryentry{thi-giac-may-tinh}
{
    name={thị giác máy tính},
    description={(computer vision)}
}

\newglossaryentry{phan-vung-anh}
{
    name={phân vùng ảnh},
    description={(image segmentation)}
}

\newglossaryentry{luoi-chinh-quy}
{
    name={lưới chính quy},
    description={(regular grid)}
}

\newglossaryentry{dang-huong}
{
    name={đẳng hướng},
    description={(isotropic)}
}

\newglossaryentry{diem-dung}
{
    name={điểm dừng},
    description={(stationary point)}
}

\newglossaryentry{leo-gradient}
{
    name={leo gradient},
    description={(gradient ascent)}
}

\newglossaryentry{leo-gradient-ngau-nhien}
{
    name={leo gradient ngẫu nhiên},
    description={(stochastic gradient ascent)}
}

\newglossaryentry{khong-gian-gia-thuyet}
{
    name={không gian giả thuyết},
    description={(hypothesis space)}
}

\newglossaryentry{tong-hop-tu-luc}
{
    name={tổng hợp tự lực},
    description={(bootstrap aggregating - BAGGING)}
}

\newglossaryentry{lua-chon-dac-trung}
{
    name={lựa chọn đặc trưng},
    description={(feature selection)}
}

\newglossaryentry{trich-xuat-dac-trung}
{
    name={trích xuất đặc trưng},
    description={(feature extraction)}
}

\newglossaryentry{tang-cuong-du-lieu}
{
    name={tăng cường dữ liệu},
    description={(dataset augmentation)}
}

\newglossaryentry{gradient-doi-nghich}
{
    name={gradient đối nghịch},
    description={(negative gradient)}
}

\newglossaryentry{phuong-phap-gradient-tang-toc}
{
    name={phương pháp gradient tăng tốc},
    description={(accelerated gradient method)}
}

\newglossaryentry{khong-gian-rong}
{
    name={không gian rỗng},
    description={(null space)}
}

\newglossaryentry{tang-ket-noi-day-du}
{
    name={tầng kết nối đầy đủ},
    description={(fully connected layer)}
}

\newglossaryentry{buoc-ngau-nhien}
{
    name={bước ngẫu nhiên},
    description={(random walk)}
}

\newglossaryentry{dung-chung-tham-so}
{
    name={dùng chung tham số},
    description={(parameter sharing)}
}

\newglossaryentry{troi-buoc-tham-so}
{
    name={trói buộc tham số},
    description={(parameter tying)}
}

\newglossaryentry{truong-tiep-nhan}
{
    name={trường tiếp nhận},
    description={(receptive field)}
}

\newglossaryentry{tich-chap-sai}
{
    name={tích chập sải},
    description={(strided convolution)}
}

\newglossaryentry{sai-chap}
{
    name={sải chập},
    description={(stride)}
}

\newglossaryentry{cau-truc-to-po}
{
    name={cấu trúc tô-pô},
    description={(topology)}
}

\newglossaryentry{tich-chap-xen-ke}
{
    name={tích chập xen kẽ},
    description={(tiled convolution)}
}

\newglossaryentry{cap-cau-phuong}
{
    name={cặp cầu phương},
    description={(quadrature pair)}
}

\newglossaryentry{mang-neuron-truy-hoi}
{
    name={mạng neuron truy hồi},
    description={(recurrent neural network)}
}

\newacronym{rnn}{RNN}{recurrent neural network}

\newglossaryentry{mang-neuron-truy-hoi-co-cong}
{
    name={mạng neuron truy hồi có cổng},
    description={(gated recurrent neural network)}
}

\newglossaryentry{don-vi-truy-hoi-co-cong}
{
    name={đơn vị truy hồi có cổng},
    description={(gated recurrent unit)}
}

\newacronym{gru}{GRU}{gated recurrent unit}

\newglossaryentry{may-vector-ho-tro}
{
    name={máy vector hỗ trợ},
    description={(support vector machine)}
}

\newacronym{svm}{SVM}{support vector machine}

\newglossaryentry{mang-de-quy}
{
    name={mạng đệ quy},
    description={(recursive network)}
}

\newglossaryentry{buoc-thoi-gian}
{
    name={bước thời gian},
    description={(time step)}
}

\newglossaryentry{RNN-song-huong}
{
    name={RNN song hướng},
    description={(Bidirectional RNN)}
}

\newglossaryentry{co-che-chu-y}
{
    name={cơ chế chú ý},
    description={(attention mechanism)}
}

\newglossaryentry{song-tuyen-tinh}
{
    name={song tuyến tính},
    description={(bilinear)}
}

\newglossaryentry{ban-xac-dinh-duong}
{
    name={bán xác định dương},
    description={(positive semidefinite)}
}

\newglossaryentry{ban-xac-dinh-am}
{
    name={bán xác định âm},
    description={(negative semidefinite)}
}

\newglossaryentry{toan-tu-vet}
{
    name={toán tử vết},
    description={(trace operator)}
}

\newglossaryentry{hoan-vi-tuan-hoan}
{
    name={hoán vị tuần hoàn},
    description={(circular permutation)}
}

\newglossaryentry{truy-van-can-thiep}
{
    name={truy vấn can thiệp},
    description={(intervention query)}
}

\newglossaryentry{sai-so-khai-quat-hoa}
{
    name={sai số khái quát hóa},
    description={(generalization error)}
}

\newglossaryentry{trung-binh-sai-so-tuyet-doi}
{
    name={trung bình sai số tuyệt đối},
    description={(mean absolute error)}
}

\newacronym{mae}{MAE}{mean absolute error}

\newglossaryentry{hoc-hop-li-cuc-dai}
{
    name={học hợp lí cực đại},
    description={(maximum likelihood learning)}
}

\newglossaryentry{ham-tuyen-tinh-tung-doan}
{
    name={hàm tuyến tính từng đoạn},
    description={(piecewise linear function)}
}

\newglossaryentry{don-vi-tuyen-tinh-tung-doan}
{
    name={đơn vị tuyến tính từng đoạn},
    description={(piecewise linear unit)}
}

\newglossaryentry{tuyen-tinh-tung-doan}
{
    name={tuyến tính từng đoạn},
    description={(piecewise linear)}
}

\newglossaryentry{khop-ban-mau}
{
    name={khớp bản mẫu},
    description={(template matching)}
}

\newglossaryentry{tinh-toan-bieu-tuong}
{
    name={tính toán biểu tượng},
    description={(symbolic computation)}
}

\newglossaryentry{tich-luy-che-do-nguoc}
{
    name={tích lũy chế độ ngược},
    description={(reverse mode accumulation)}
}

\newglossaryentry{tich-luy-che-do-thuan}
{
    name={tích lũy chế độ thuận},
    description={(forward mode accumulation)}
}

\newglossaryentry{hoi-quy-ngon-song}
{
    name={hồi quy ngọn sóng},
    description={(ridge regression)}
}

\newglossaryentry{khang-nhieu}
{
    name={kháng nhiễu},
    description={(robust to noise, noise robustness)}
}

\newglossaryentry{theo-duoi-doi-sanh-truc-giao}
{
    name={theo đuổi đối sánh trực giao},
    description={(orthogonal matching pursuit)}
}

\newacronym{omp}{OMP}{Orthogonal Matching Pursuit}

\newglossaryentry{trung-binh-hoa-mo-hinh}
{
    name={trung bình hóa mô hình},
    description={(model averaging)}
}

\newglossaryentry{co-gian-trong-so}
{
    name={co giãn trọng số},
    description={(weight scaling)}
}

\newglossaryentry{ngat-ket-noi}
{
    name={ngắt kết nối},
    description={(DropConnect)}
}

\newglossaryentry{don-mode}
{
    name={đơn mode},
    description={(unimodal)}
}

\newglossaryentry{da-mode}
{
    name={đa mode},
    description={(multimodal)}
}

\newglossaryentry{hoi-quy-da-mode}
{
    name={hồi quy đa mode},
    description={(multimodal regression)}
}

\newglossaryentry{tro-choi-co-tong-bang-khong}
{
    name={trò chơi có tổng bằng không},
    description={(zero-sum game)}
}

\newglossaryentry{giam-xoc}
{
    name={giảm xóc},
    description={(damping)}
}

\newglossaryentry{tim-kiem-dau-dac-trung}
{
    name={tìm kiếm dấu đặc trưng},
    description={(feature-sign search)}
}

\newglossaryentry{leo-theo-toa-do}
{
    name={leo theo tọa độ},
    description={(coordinate ascent)}
}

\newglossaryentry{kho-tinh-toan}
{
    name={khó tính toán},
    description={(intractable)}
}

\newglossaryentry{PCA-huong-xac-suat}
{
    name={PCA hướng xác suất},
    description={(probabilistic PCA)}
}

\newglossaryentry{dieu-hoa-song-song}
{
    name={điều hòa song song},
    description={(parallel tempering)}
}

\newglossaryentry{chuyen-tiep-dieu-hoa}
{
    name={chuyển tiếp điều hòa},
    description={(tempered transition)}
}

\newglossaryentry{phan-phoi-can-bang}
{
    name={phân phối cân bằng},
    description={(equilibrium distribution)}
}

\newglossaryentry{phan-phoi-dung}
{
    name={phân phối dừng},
    description={(stationary distribution)}
}

\newglossaryentry{tach-biet-d}
{
    name={tách-biệt-D},
    description={(D-seperation)}
}

\newglossaryentry{ai-luc}
{
    name={ái lực},
    description={(affinity)}
}

\newglossaryentry{cuong-do-thong-ke}
{
    name={cường độ thống kê},
    description={(statistical strength)}
}

\newglossaryentry{phan-tich-thanh-phan-doc-lap}
{
    name={phân tích thành phần độc lập},
    description={(independent component analysis - ICA)}
}

\newglossaryentry{chinh-sach}
{
    name={chính sách},
    description={(policy)}
}

\newglossaryentry{khai-pha}
{
    name={khai phá},
    description={(explore)}
}

\newglossaryentry{khai-thac}
{
    name={khai thác},
    description={(exploit)}
}

\newglossaryentry{ma-tran-dich-thuat}
{
    name={ma trận dịch thuật},
    description={(translation matrix)}
}

\newglossaryentry{vector-tu-xuyen-ngon-ngu}
{
    name={vector từ xuyên ngôn ngữ},
    description={(cross-lingual word vector)}
}

\newglossaryentry{can-chinh-xuyen-ngon-ngu}
{
    name={căn chỉnh xuyên ngôn ngữ},
    description={(cross-lingual alignment)}
}

\newglossaryentry{khong-tim-thay-trong-bo-dem}
{
    name={không tìm thấy trong bộ đệm},
    description={(cache-miss)}
}

\newglossaryentry{bo-dem}
{
    name={bộ đệm},
    description={(cache). Còn gọi là ``bộ nhớ đệm''}
}

\newglossaryentry{giao-tac-bo-nho}
{
    name={giao tác bộ nhớ},
    description={(memory transaction)}
}

\newglossaryentry{hop-nhat}
{
    name={hợp nhất},
    description={(coalesce)}
}

\newglossaryentry{song-song-du-lieu}
{
    name={song song dữ liệu},
    description={(data parallelism)}
}

\newglossaryentry{song-song-mo-hinh}
{
    name={song song mô hình},
    description={(model parallelism)}
}

\newglossaryentry{do-nhay}
{
    name={độ nhạy},
    description={(recall)}
}

\newglossaryentry{chuoi-phan-tang}
{
    name={chuỗi phân tầng},
    description={(cascade)}
}

\newglossaryentry{tui-tu}
{
    name={túi từ},
    description={(bag of words)}
}

\newglossaryentry{bang-bam}
{
    name={bảng băm},
    description={(hash table)}
}

\newglossaryentry{bam-ngu-nghia}
{
    name={băm ngữ nghĩa},
    description={(semantic hashing)}
}

\newacronym{asr}{ASR}{Automatic Speech Recognition}

\newacronym{ctc}{CTC}{Connectionist Temporal Classification}

\newacronym{ebm}{EBM}{Energy-Based Model}

\newacronym{em}{EM}{Expectation Maximization}

\newacronym{gcn}{GCN}{Global Contrast Normalization}

\newacronym{gpu}{GPU}{Graphics Processing Unit}

\newacronym{gp-gpu}{GP-GPU}{General Purpose Graphics Processing Unit}

\newacronym{ilsvrc}{ILSVRC}{ImageNet Large Scale Visual Recognition Challenge}

\newacronym{kkt}{KKT}{Karush-Kuhn-Tucker}

\newacronym{kl}{KL}{Kullback-Leibler}

\newacronym{iid}{i.i.d}{Independent and identically distributed}

\newacronym{mcmc}{MCMC}{Markov chain Monte Carlo method}

\newacronym{nade}{NADE}{neural auto-regressive density estimator}

\newacronym{draw}{DRAW}{deep recurrent attention writer}

\newacronym{nlm}{NLM}{neural language model}

\newacronym{nlp}{NLP}{Natural Language Processing}

\newacronym{reinforce}{REINFORCE}{REward Increment = nonnegative Factor x Offset
reinforcement x Characteristic Eligibility}

\newacronym{ica}{ICA}{Independent Component Analysis}

\newacronym{nice}{NICE}{nonlinear independent components estimation}

\newacronym{vc}{VC}{Vapnik-Chervonenkis}

\newacronym{lst}{LST}{liquid state machine}

\newacronym{abc}{ABC}{approximate Bayesian computation}

\newacronym{pot}{PoT}{product of Student t-distribution}

\newacronym{mpot}{mPoT}{mean product of Student t-distribution}

\newacronym{pdf}{PDF}{probability density function}

\newacronym{hmm}{HMM}{hidden Markov model}

\newacronym{gmm}{GMM}{Gaussian mixture model}

\newacronym{psd}{PSD}{predictive sparse decomposition}

\newacronym{ista}{ISTA}{Iterative Shrinkage and Thresholding Algorithm}

\newacronym{ai}{AI}{artificial intelligence}

\newacronym{it}{IT}{inferotemporal cortex / inferior temporal cortex}

\newacronym{cpu}{CPU}{Central Processing Unit}

\newacronym{relu}{ReLU}{rectified linear unit}

\newacronym{prelu}{PReLU}{parametric rectified linear unit}

\newacronym{mmd}{MMD}{maximum mean discrepancy}

\newacronym{fvbn}{FVBN}{fully-visible Bayes network}

\newacronym{rnade}{RNADE}{real-valued neural autoregressive density-estimator}


\newglossaryentry{bien-quyet-dinh}
{
    name={biên quyết định},
    description={(decision boundary)}
}

\newglossaryentry{bo-du-doan-hop-the}
{
    name={bộ dự đoán hợp thể},
    description={(ensemble predictor)}
}

\newglossaryentry{bo-hoc-bieu-dien-phan-tan}
{
    name={bộ học biểu diễn phân tán},
    description={(distributed representation learner)}
}

\newglossaryentry{bo-kiem-soat}
{
    name={bộ kiểm soát},
    description={(regularizer)}
}


\newglossaryentry{soi-doc}
{
    name={sợi dọc},
    description={(warp). Các luồng xử lí được chia thành từng nhóm nhỏ gọi là
    ``sợi dọc'' (một số người dịch là ``bó luồng"). Cái tên sợi dọc là một lối
    chơi chữ dựa trên ý tưởng sợi dọc trong dệt may là một bó sợi song song.
    Ngoài ra còn có thread block (max $512/1024$
    threads) cộng đồng dịch là ``khối luồng", mỗi thread block gồm nhiều warps,
    mỗi warp là $32$ threads executing same instructions.}
}

\newglossaryentry{bo-phan-loai}
{
    name={bộ phân loại},
    description={(classifier)}
}

\newglossaryentry{bo-phan-loai-hoi-quy-softmax}
{
    name={bộ phân loại hồi quy softmax},
    description={(softmax regression classifier)}
}

\newglossaryentry{bo-phan-loai-nhi-phan}
{
    name={bộ phân loại nhị phân},
    description={(binary classifier)}
}

\newglossaryentry{bo-phan-loai-softmax}
{
    name={bộ phân loại softmax},
    description={(softmax classifier)}
}

\newglossaryentry{bo-phan-loai-tiep-tuyen-da-tap}
{
    name={bộ phân loại tiếp tuyến đa tạp},
    description={(manifold tangent classifier)}
}

\newglossaryentry{bo-phan-loai-tuyen-tinh}
{
    name={bộ phân loại tuyến tính},
    description={(linear classifier)}
}

\newglossaryentry{chuan-hoa-theo-lo}
{
    name={chuẩn hóa theo lô},
    description={(batch normalization)}
}

\newglossaryentry{co-che-kiem-soat}
{
    name={cơ chế kiểm soát},
    description={(regularization)}
}

\newglossaryentry{cong-quen}
{
    name={cổng quên},
    description={(forget gate)}
}

\newglossaryentry{cong-xoa}
{
    name={cổng xóa},
    description={(reset gate)}
}

\newglossaryentry{cuc-tieu-cuc-bo}
{
    name={cực tiểu cục bộ},
    description={(local minimum)}
}

\newglossaryentry{cuc-tieu-toan-cuc}
{
    name={cực tiểu toàn cục},
    description={(global minimum)}
}

\newglossaryentry{muc-phat}
{
    name={mức phạt},
    description={(penalty)}
}

\newglossaryentry{muc-phat-thua}
{
    name={mức phạt thưa},
    description={(sparsity penalty)}
}

\newglossaryentry{muc-phat-chuan}
{
    name={mức phạt chuẩn},
    description={(norm penalty)}
}

\newglossaryentry{muc-phat-tri-tuyet-doi}
{
    name={mức phạt trị tuyệt đối},
    description={(absolute value penalty)}
}

\newglossaryentry{muc-phat-co-rut}
{
    name={mức phạt co rút},
    description={(contractive penalty)}
}

\newglossaryentry{muc-phat-sai-so-khoi-phuc}
{
    name={mức phạt sai số khôi phục},
    description={(reconstruction error penalty)}
}

\newglossaryentry{sai-so-khoi-phuc}
{
    name={sai số khôi phục},
    description={(reconstruction error)}
}

\newglossaryentry{sai-so-Bayes}
{
    name={sai số Bayes},
    description={(Bayes error)}
}

\newglossaryentry{phan-phoi-khoi-phuc}
{
    name={phân phối khôi phục},
    description={(reconstruction distribution)}
}

\newglossaryentry{mat-mat-khoi-phuc}
{
    name={mất mát khôi phục},
    description={(loss reconstruction)}
}

\newglossaryentry{co-rut}
{
    name={co rút},
    description={(contractive)}
}

\newglossaryentry{go-loi}
{
    name={gỡ lỗi},
    description={(debug)}
}

\newglossaryentry{diem-yen-ngua}
{
    name={điểm yên ngựa},
    description={(saddle point)}
}

\newglossaryentry{do-chinh-xac-toan-phan}
{
    name={độ chính xác toàn phần},
    description={(accuracy) xét trong ngữ cảnh huấn luyện học máy}
}

\newglossaryentry{do-chinh-xac-rieng-phan}
{
    name={độ chính xác riêng phần},
    description={(precision) xét trong ngữ cảnh huấn luyện học máy.
        Trong các tác vụ phân loại, precision là độ chính xác đo
        theo từng nhãn, vậy nên chúng tôi gọi là độ chính xác riêng phần,
        để phân biệt rõ với \gls{do-chinh-xac-toan-phan} (accuracy) vốn
        đo trên toàn bộ mẫu kiểm thử. Thuật ngữ precision trong tiếng Anh
        khá nhập nhằng, bạn đọc cần lưu ý văn cảnh để hiểu đúng, xthêm
        \gls{do-tu}. Khi dịch ra thuật ngữ Việt, chúng tôi cố gắng loại bỏ
        sự nhập nhằng này bằng cách tập trung dịch sát phần nội hàm kĩ thuật
        thay vì bám sát máy móc dịch từng từ thành phần trong thuật ngữ Anh.}
}

\newglossaryentry{do-do-hieu-nang}
{
    name={độ đo hiệu năng},
    description={(performance measure)}
}

\newglossaryentry{do-hop-li}
{
    name={độ hợp lí},
    description={(likelihood)}
}

\newglossaryentry{do-phu}
{
    name={độ phủ},
    description={(coverage)}
}

\newglossaryentry{do-hop-li-gia}
{
    name={độ hợp lí giả},
    description={(pseudo-likelihood)}
}

\newglossaryentry{do-hop-li-thang-log}
{
    name={độ hợp lí thang log},
    description={(log-likelihood)}
}

\newglossaryentry{do-lowij}
{
    name={lợi ích},
    description={(gain)}
}

\newglossaryentry{do-lech-chuan}
{
    name={độ lệch chuẩn},
    description={(standard deviation)}
}


\newglossaryentry{do-tu}
{
    name={độ tụ},
    description={(precision)}
}

\newglossaryentry{dung-luong}
{
    name={dung lượng},
    description={(capacity)}
}

\newglossaryentry{dung-luong-bieu-dien}
{
    name={dung lượng biểu diễn},
    description={(representational capacity)}
}

\newglossaryentry{dung-luong-hieu-dung}
{
    name={dung lượng hiệu dụng},
    description={(effective capacity)}
}

\newglossaryentry{entropy-cheo}
{
    name={entropy chéo},
    description={(cross entropy)}
}

\newglossaryentry{gia-chuan}
{
    name={giả chuẩn},
    description={(pseudonorm)}
}

\newglossaryentry{gia-nghich-dao}
{
    name={giả nghịch đảo},
    description={(pseudoinverse)}
}

\newglossaryentry{ham-chi-phi}
{
    name={hàm chi phí},
    description={(cost function)}
}

\newglossaryentry{ham-chi-phi-ngau-nhien}
{
    name={hàm chi phí ngẫu nhiên},
    description={(stochastic cost function)}
}

\newglossaryentry{ham-doi-log-hop-li}
{
    name={hàm đối log hợp lí},
    description={(negative log-likelihood)}
}

\newglossaryentry{ham-mat-mat}
{
    name={hàm mất mát},
    description={(loss function)}
}

\newglossaryentry{ham-mat-mat-thay-the}
{
    name={hàm mất mát thay thế},
    description={(surrogate loss function)}
}

\newglossaryentry{hang-tu-kiem-soat}
{
    name={hạng tử kiểm soát},
    description={(regularization term)}
}

\newglossaryentry{hoc-ban-giam-sat}
{
    name={học bán giám sát},
    description={(semi-supervised learning)}
}

\newglossaryentry{hoc-bieu-dien}
{
    name={học biểu diễn},
    description={(representation learning)}
}

\newglossaryentry{hoc-bieu-dien-phan-tan}
{
    name={học biểu diễn phân tán},
    description={(distributed representation learning)}
}

\newglossaryentry{hoc-co-giam-sat}
{
    name={học có giám sát},
    description={(supervised learning)}
}

\newglossaryentry{hoc-tang-cuong}
{
    name={học tăng cường},
    description={(reinforcement learning). Còn gọi là ``học củng cố''}
}

\newglossaryentry{hoc-hop-the}
{
    name={học hợp thể},
    description={(ensemble learning)}
}

\newglossaryentry{hoc-khong-giam-sat}
{
    name={học không giám sát},
    description={(unsupervised learning)}
}

\newglossaryentry{hoc-may}
{
    name={học máy},
    description={(machine learning)}
}

\newglossaryentry{hoc-sau}
{
    name={học sâu},
    description={(deep learning). Hiểu theo nghĩa học một
    hệ phân cấp khái niệm, mỗi phân cấp sẽ ứng với một tầng neuron.
    Chúng tôi nghĩ cách dịch ``học đa tầng'' hoặc
    ``học đa tầng khái niệm'' sẽ phản ánh rõ hơn. Nhưng vì từ
    ``học sâu'' cũng đã khá phổ biến và được chấp nhận trước đó,
    và cũng bám sát theo thuật ngữ Anh, cũng gợi tả nghĩa khá tốt,
    nên chúng tôi lựa chọn phương án này. Còn lại, các thuật ngữ khác
    có sử dụng chữ deep, chúng tôi ưu tiên dịch thành ``đa tầng'',
    tức là chúng tôi thiên về cách dịch thuật ngữ sát phần nội hàm kĩ thuật
    hơn là bám sát theo từng từ trong thuật ngữ Anh.}
}

\newglossaryentry{hoc-truc-tuyen}
{
    name={học trực tuyến},
    description={(online learning)}
}

\newglossaryentry{hoi-quy-lang-gieng-gan-nhat}
{
    name={hồi quy láng giềng gần nhất},
    description={(nearest neighbor regression)}
}

\newglossaryentry{hoi-quy-logit}
{
    name={hồi quy logit},
    description={(logistic regression)}
}

\newglossaryentry{hoi-quy-tuyen-tinh}
{
    name={hồi quy tuyến tính},
    description={(linear regression)}
}

\newglossaryentry{hop-the}
{
    name={hợp thể},
    description={(ensemble)}
}

\newglossaryentry{huan-luyen-doi-khang}
{
    name={huấn luyện đối kháng},
    description={(adversarial training)}
}

\newglossaryentry{huan-luyen-doi-khang-ao}
{
    name={huấn luyện đối kháng ảo},
    description={(virtual adversarial training)}
}

\newglossaryentry{mau-doi-khang}
{
    name={mẫu đối kháng},
    description={(adversarial example)}
}

\newglossaryentry{mau-doi-khang-ao}
{
    name={mẫu đối kháng ảo},
    description={(virtual adversarial example)}
}

\newglossaryentry{huan-luyen-truoc}
{
    name={huấn luyện trước},
    description={(pre-training)}
}

\newglossaryentry{huan-luyen-truoc-nong}
{
    name={huấn luyện trước nông},
    description={(shallow pre-training)}
}

\newglossaryentry{huan-luyen-truoc-tham-lam}
{
    name={huấn luyện trước tham lam},
    description={(greedy pre-training)}
}

\newglossaryentry{huan-luyen-truoc-co-giam-sat}
{
    name={huấn luyện trước có giám sát},
    description={(supervised pretraining)}
}

\newglossaryentry{huan-luyen-truoc-khong-giam-sat}
{
    name={huấn luyện trước không giám sát},
    description={(unsupervised pretraining)}
}

\newglossaryentry{huan-luyen-truoc-tham-lam-co-giam-sat}
{
    name={huấn luyện trước tham lam có giám sát},
    description={(greedy supervised pretraining)}
}

\newglossaryentry{huan-luyen-truoc-tham-lam-khong-giam-sat}
{
    name={huấn luyện trước tham lam không giám sát},
    description={(greedy unsupervised pretraining)}
}

\newglossaryentry{huan-luyen-truoc-tham-lam-theo-tang}
{
    name={huấn luyện trước tham lam theo tầng},
    description={(greedy layer-wise pre-training)}
}

\newglossaryentry{huan-luyen-truoc-tham-lam-theo-tang-khong-giam-sat}
{
    name={huấn luyện trước tham lam theo tầng không giám sát},
    description={(greedy layer-wise unsupervised pre-training)}
}

\newglossaryentry{huan-luyen-truoc-tham-lam-theo-tang-co-giam-sat}
{
    name={huấn luyện trước tham lam theo tầng có giám sát},
    description={(greedy layer-wise supervised pre-training)}
}

\newglossaryentry{ket-thuc-som}
{
    name={kết thúc sớm},
    description={(early stopping)}
}

\newglossaryentry{kem-dieu-hoa}
{
    name={kém điều hòa},
    description={(ill-conditioning hoặc poor-condition)}
}

\newglossaryentry{kha-phan-tuyen-tinh}
{
    name={khả phân tuyến tính},
    description={(linearly separable)}
}

\newglossaryentry{khoa-hoc-may-tinh}
{
    name={khoa học máy tính},
    description={(computer science)}
}

\newglossaryentry{khoa-hoc-than-kinh}
{
    name={khoa học thần kinh},
    description={(neuroscience)}
}

\newglossaryentry{khoi-tao-co-chuan-hoa}
{
    name={khởi tạo có chuẩn hóa},
    description={(normalized initialization)}
}

\newglossaryentry{khoi-tao-thua}
{
    name={khởi tạo thưa},
    description={(sparse initialization)}
}

\newglossaryentry{kich-thuoc-lo}
{
    name={kích thước lô},
    description={(batch size)}
}

\newglossaryentry{kiem-dinh-cheo}
{
    name={kiểm định chéo},
    description={(cross validation)}
}

\newglossaryentry{lo}
{
    name={lô},
    description={(batch)}
}

\newglossaryentry{lo-nho}
{
    name={lô nhỏ},
    description={(mini-batch)}
}

\newglossaryentry{mang-tu-hoi-quy}
{
    name={mạng tự hồi quy},
    description={(auto-recurrent network)}
}

\newglossaryentry{mang-tu-hoi-quy-neuron}
{
    name={mạng tự hồi quy neuron},
    description={(neural auto-recurrent network)}
}

\newglossaryentry{mang-neuron-tre-thoi-gian}
{
    name={mạng neuron trễ thời gian},
    description={(Time-delay neural network)}
}

\newacronym{tdnn}{TDNN}{time-delay neural network}

\newglossaryentry{mang-tich-chap}
{
    name={mạng tích chập},
    description={(convolutional network hoặc convolutional neural network)}
}

\newacronym{cnn}{CNN}{Convolutional Neural Network}

\newglossaryentry{mang-tich-chap-da-tang}
{
    name={mạng tích chập đa tầng},
    description={(deep convolutional network)}
}

\newglossaryentry{mang-tich-chap-lan-truyen-thuan}
{
    name={mạng tích chập lan truyền thuận},
    description={(feedforward convolutional network)}
}

\newglossaryentry{mang-tich-chap-truy-hoi}
{
    name={mạng tích chập truy hồi},
    description={(recurrent convolutional network)}
}

\newglossaryentry{mang-tich-chap-xep-chong}
{
    name={mạng tích chập xếp chồng},
    description={(tiled convolutional network)}
}

\newglossaryentry{mang-truy-hoi}
{
    name={mạng truy hồi},
    description={(recurrent network hoặc recurrent neural network)}
}

\newglossaryentry{mang-truy-hoi-da-tang}
{
    name={mạng truy hồi đa tầng},
    description={(deep recurrent network)}
}

\newglossaryentry{mo-hinh-hop-the}
{
    name={mô hình hợp thể},
    description={(ensemble model)}
}

\newglossaryentry{mo-hinh-phi-tham-so}
{
    name={mô hình phi tham số},
    description={(nonparametric model) Mô hình dạng này có số lượng
    tham số không cố định. Số lượng tham số của mô hình tăng dần
    theo độ lớn của dữ liệu.}
}

\newglossaryentry{mo-hinh-sinh-mau-ma-hoa-thua}
{
    name={mô hình sinh mẫu mã hóa thưa},
    description={(sparse coding generative model)}
}

\newglossaryentry{phan-cum}
{
    name={phân cụm},
    description={(clustering)}
}

\newglossaryentry{phan-da-loai}
{
    name={phân đa loại},
    description={(multi-class classification)}
}

\newglossaryentry{loai}
{
    name={loại},
    description={(class hoặc category)}
}

\newglossaryentry{phan-loai}
{
    name={phân loại},
    description={(classification)}
}

\newglossaryentry{phan-loai-hoi-quy-logit}
{
    name={phân loại hồi quy logit},
    description={(logistic regression classification)}
}

\newglossaryentry{phan-loai-nhi-phan}
{
    name={phân loại nhị phân},
    description={(binary classification)}
}

\newglossaryentry{phan-loai-tuyen-tinh}
{
    name={phân loại tuyến tính},
    description={(linear classification)}
}

\newglossaryentry{gia-tri-suy-bien}
{
    name={giá trị suy biến},
    description={(singular value)}
}

\newglossaryentry{vector-suy-bien}
{
    name={vector suy biến},
    description={(singular vector)}
}

\newglossaryentry{ma-tran-suy-bien}
{
    name={ma trận suy biến},
    description={(singular matrix) Là ma trận vuông với các cột
    phụ thuộc tuyến tính. Hình vuông gồm các số trong đó tổng các
    đường ngang, dọc, chéo đều bằng nhau gọi là ma phương
    (magic square), ma (魔) trong ma quái, phương (方) là hình vuông.
    Tương tự, trận (陣) có nghĩa xưa là bày bố hàng lối quân lính,
    ma trận (魔陣) là cách sắp xếp hàng lối (cho các con số) một
    cách ma quái ảo diệu, một từ tiếng Việt rất hay mà ta đã dùng
    để dịch chữ matrix. Ma trận có các cột phụ thuộc tuyến tính,
    tức là hạng (rank) của ma trận này sẽ thấp hơn số cột của nó,
    không gian vector nhận các cột của ma trận này làm hệ sinh
    cũng sẽ có số chiều nhỏ hơn số cột của ma trận, hiểu một cách
    nôm na là số chiều cho phép các vector ``vùng vẫy'' trong không gian
    đó cũng nhỏ hơn, vì vậy ta gọi là ``suy biến'', suy (衰) nghĩa là giảm,
    biến (變) ở đây là biến số, khả năng biến đổi, khả năng di động.}
}

\newglossaryentry{ma-tran-do-tu}
{
    name={ma trận độ tụ},
    description={(precision matrix)}
}

\newglossaryentry{phan-tich-gia-tri-suy-bien}
{
    name={phân tích giá trị suy biến},
    description={(singular value decomposition)}
}

\newacronym{svd}{SVD}{Singular Value Decomposition}

\newglossaryentry{phan-tich-rieng}
{
    name={phân tích riêng},
    description={(eigendecomposition)}
}

\newglossaryentry{phan-tich-thanh-phan-chinh}
{
    name={phân tích thành phần chính},
    description={(Principal component analysis)}
}

\newacronym{pca}{PCA}{Principal component analysis}

\newglossaryentry{phan-phoi-thuc-nghiem}
{
    name={phân phối thực nghiệm},
    description={(empirical distribution)}
}

\newglossaryentry{phan-phoi-chuan}
{
    name={phân phối chuẩn},
    description={(normal distribution)}
}

\newglossaryentry{phan-phoi-chuan-tac}
{
    name={phân phối chuẩn tắc},
    description={(standard normal distribution)}
}

\newglossaryentry{phan-phoi-chuan-da-bien}
{
    name={phân phối chuẩn đa biến},
    description={(multivariate normal distribution)}
}

\newglossaryentry{phan-phoi-sinh-du-lieu}
{
    name={phân phối sinh dữ liệu},
    description={(data-generating distribution)}
}

\newglossaryentry{phan-phoi-thuc-su}
{
    name={phân phối thực sự},
    description={(true distribution)}
}

\newglossaryentry{phan-phoi-truong-trung-binh}
{
    name={phân phối trường trung bình},
    description={(mean-field distribution)}
}

\newglossaryentry{phuong-phap-hop-the}
{
    name={phương pháp hợp thể},
    description={(ensemble method)}
}

\newglossaryentry{qua-khop}
{
    name={quá khớp},
    description={(overfitting)}
}

\newglossaryentry{rui-ro-thuc-nghiem}
{
    name={rủi ro thực nghiệm},
    description={(empirical risk)}
}

\newglossaryentry{sai-so-kiem-dinh}
{
    name={sai số kiểm định},
    description={(validation error)}
}

\newglossaryentry{sai-so-kiem-thu}
{
    name={sai số kiểm thử},
    description={(test error)}
}

\newglossaryentry{sai-so-huan-luyen}
{
    name={sai số huấn luyện},
    description={(training error)}
}

\newglossaryentry{sai-so-phan-loai}
{
    name={sai số phân loại},
    description={(classification error)}
}

\newglossaryentry{sai-so-tong-quat-hoa}
{
    name={sai số tổng quát hóa},
    description={(generalization error)}
}

\newglossaryentry{sai-so-doi-khang}
{
    name={sai số đối kháng},
    description={(adversarial error)}
}

\newglossaryentry{sieu-tham-so}
{
    name={siêu tham số},
    description={(hyperparameter)}
}

\newglossaryentry{suy-dien-bayes}
{
    name={suy diễn Bayes},
    description={(Bayesian inference)}
}

\newglossaryentry{tang-ma-hoa}
{
    name={tầng mã hóa},
    description={code layer)}
}

\newglossaryentry{tang-giai-ma}
{
    name={tầng giải mã},
    description={decode layer)}
}

\newglossaryentry{tang-an}
{
    name={tầng ẩn},
    description={(hidden) layer)}
}

\newglossaryentry{tang-kha-kien}
{
    name={tầng khả kiến},
    description={(visible layer)}
}

\newglossaryentry{tang-nhung}
{
    name={tầng nhúng},
    description={(embedding layer)}
}

\newglossaryentry{tang-phan-loai}
{
    name={tầng phân loại},
    description={(classification layer)}
}

\newglossaryentry{tap-du-lieu}
{
    name={tập dữ liệu},
    description={(dataset)}
}

\newglossaryentry{tap-kiem-dinh}
{
    name={tập kiểm định},
    description={(validation set)}
}

\newglossaryentry{tap-kiem-thu}
{
    name={tập kiểm thử},
    description={(test set hoặc test dataset)}
}

\newglossaryentry{tap-huan-luyen}
{
    name={tập huấn luyện},
    description={(training set)}
}

\newglossaryentry{thuat-toan-hoc}
{
    name={thuật toán học tập},
    description={(learning algorithm)}
}

\newglossaryentry{tich-chap-theo-lo}
{
    name={tích chập theo lô},
    description={(batch convolution)}
}

\newglossaryentry{toi-uu-cuc-bo}
{
    name={tối ưu cục bộ},
    description={(local optimum)}
}

\newglossaryentry{toi-uu-toan-cuc}
{
    name={tối ưu toàn cục},
    description={(global optimum)}
}

\newglossaryentry{tran-so-tren}
{
    name={tràn số trên},
    description={(numerical overflow). Máy tính
    chỉ có thể biểu diễn số trong một khoảng nhất định.
    Khi các tính toán cho ra kết quả quá lớn vượt quá
    khả năng biểu diễn của máy tính thì giá trị này sẽ
    được xấp xỉ thành $\infty$ hoặc $-\infty$.}
}

\newglossaryentry{tran-so-duoi}
{
    name={tràn số dưới},
    description={(numerical underflow). Về mặt toán học,
    $1$ chia cho $10$ vô hạn lần sẽ không bao giờ bằng $0$.
    Tuy vậy, máy tính chỉ dùng một lượng hữu hạn bit để mô tả
    số thập phân, vậy nên sau mỗi lần chia như vậy, con số
    khác $0$ đầu tiên sau dấu phẩy sẽ bị đẩy dần về phải,
    cho đến khi "tràn" ra ngoài khoảng có thể biểu diễn
    bằng số bit hữu hạn của máy tính thì khi đó máy tính
    sẽ lưu trữ con số đó đúng bằng $0$. Vì lí do này,
    chúng tôi chọn cách dịch là ``tràn số dưới''.}
}

\newglossaryentry{tri-rieng}
{
    name={trị riêng},
    description={(eigenvalue). Xem \gls{vector-rieng}}
}

\newglossaryentry{trung-binh-binh-phuong-sai-so}
{
    name={trung bình bình phương sai số},
    description={(mean squared error)}
}

\newacronym{mse}{MSE}{mean squared error}

\newglossaryentry{truot-gradient}
{
    name={trượt gradient},
    description={(gradient descent). Gradient cũng chỉ có
    nghĩa giản dị là ``con dốc, độ dốc''. Đây là một thuật ngữ
    gợi tả hình ảnh địa hình hàm (function landscape) khi ta
    vẽ nó lên đồ thị. Do đó, thuật ngữ gradient descent
    hoàn toàn có thể dịch một cách chuẩn xác và gợi hình
    cũng rất tốt là ``trượt dốc'', tương tự gradient ascent
    sẽ là ``leo dốc''. Bởi vì từ grandient đã quá phổ biến
    trong cộng đồng vật lí lẫn toán học xưa nay, nên chúng
    tôi chấp nhận dịch thành ``trượt gradient, leo gradient''.
    Tuy vậy, bạn đọc có thể thoải mái sử dụng các cách dịch trên
    tùy văn cảnh}
}

\newglossaryentry{truot-gradient-ngau-nhien}
{
    name={trượt gradient ngẫu nhiên},
    description={(stochastic gradient descent)}
}

\newglossaryentry{truot-gradient-phan-tan-bat-dong-bo}
{
    name={trượt gradient phân tán bất đồng bộ},
    description={(distributed asynchronous gradient descent)}
}

\newacronym{sgd}{SGD}{stochastic gradient descent}

\newglossaryentry{truot-gradient-ngau-nhien-khong-dong-bo}
{
    name={trượt gradient ngẫu nhiên không đồng bộ},
    description={(asynchronous stochastic gradient descent)}
}

\newglossaryentry{truot-gradient-ngau-nhien-theo-lo}
{
    name={trượt gradient ngẫu nhiên theo lô},
    description={(batch stochastic gradient descent)}
}

\newglossaryentry{truot-gradient-ngau-nhien-theo-lo-nho}
{
    name={trượt gradient ngẫu nhiên theo lô nhỏ},
    description={(minibatch stochastic gradient descent)}
}

\newglossaryentry{truot-gradient-theo-lo}
{
    name={trượt gradient theo lô},
    description={(batch gradient descent)}
}

\newglossaryentry{truot-gradient-theo-lo-nho}
{
    name={trượt gradient theo lô nhỏ},
    description={(minibatch gradient descent)}
}

\newglossaryentry{truot-theo-toa-do}
{
    name={trượt theo tọa độ},
    description={(coordinate descent)}
}

\newglossaryentry{truot-theo-khoi-toa-do}
{
    name={trượt theo khối tọa độ},
    description={(block coordinate descent)}
}

\newglossaryentry{uoc-luong-hop-li-cuc-dai}
{
    name={ước lượng hợp lí cực đại},
    description={(maximum likelihood estimation)}
}

\newglossaryentry{uoc-luong-tan-suat}
{
    name={ước lượng tần suất},
    description={(frequentist estimator)}
}

\newglossaryentry{uoc-luong-khong-chech}
{
    name={ước lượng không chệch},
    description={(unbiased estimator)}
}

\newglossaryentry{uoc-luong-chech}
{
    name={ước lượng chệch},
    description={(biased estimator)}
}

\newglossaryentry{vi-khop}
{
    name={vị khớp},
    description={(underfitting) có nghĩa là chưa đủ khớp.
    Vị (未) ở đây có nghĩa là chưa đủ, tương tự như ``vị''
    trong ``vị thành niên'' (chưa đủ tuổi), ``vị hôn phu/thê''
    (chồng/vợ chưa cưới). Thuật ngữ này cũng có thể dịch là
    ``chưa khớp'', nhưng chúng tôi xét thấy từ ``chưa khớp''
    có thể gây nhập nhằng ở nhiều văn cảnh khác nên chúng tôi
    thiên về phương án ``vị khớp'' để tạo đủ sự khác biệt
    cho một thuật ngữ. Ngoài ra, xin đừng dịch under thành ``thấp''
    ghép lại thành ``thấp khớp''. Thấp khớp là thuật ngữ y học
    phổ biến của một loại bệnh.}
}

\newglossaryentry{vector-rieng}
{
    name={vector riêng},
    description={(eigenvector). Từ ``eigen'' vốn là tiếng Đức,
    có nghĩa là ``riêng, của riêng''. Eigenvector là vector
    vẫn giữ được hướng của có khi bị biến đổi tuyến tính bởi T
    nào đó. Thuật ngữ eigenvector và một từ lai Đức-Anh. Ngoài ra,
    vector thực ra lại là một từ thuần latin, tức là gốc gác
    cũng không phải tiếng Anh! Nếu xét ở phiên bản Anh hóa hơn,
    thì nó nên là principle-vector, hoặc characteristic-vector.
    Chính vì vậy, chúng tôi chọn cách dịch là vector riêng.
    Tương tự với \gls{tri-rieng}.}
}

\newglossaryentry{nhom-nut-day-du}
{
    name={nhóm nút đầy đủ},
    description={(clique). Clique là từ cổ gốc Pháp, có nghĩa là phe nhóm,
    bè lũ. Người ta chọn lấy một chữ có nghĩa ``lạ'' làm thuật ngữ.
    Bám sát theo dịch là phe, hay bè lũ đều không hợp trong nhiều
    văn cảnh diễn đạt tiếng Việt. Do vậy chúng tôi canh
    theo nội hàm kĩ thuật mạnh dạn dịch từ này thành
    ``nhóm nút kết nối đầy đủ'', gọi tắt là ``nhóm nút đầy đủ''.}
}

\newglossaryentry{the-nang-nhom-nut-day-du}
{
    name={thế năng nhóm nút đầy đủ},
    description={(clique potential)}
}

\newglossaryentry{mo-hinh-vo-huong}
{
    name={mô hình vô hướng},
    description={(undirected model)}
}

\newglossaryentry{mo-hinh-co-huong}
{
    name={mô hình có hướng},
    description={(directed model)}
}

\newglossaryentry{phan-ham}
{
    name={hàm phân hoạch},
    description={(partition function). Đây là thuật ngữ vay mượn
    từ vật lí thống kê. Trong vật lí thống kê, người ta đã dịch là
    ``hàm trạng thái'' hoặc ``hàm trạng thái thống kê'' hoặc
    ``hàm phân bố''. Một partition mô tả cách n hạt phân bổ giữa
    $k$ mức năng lượng. Partition function mô tả đặc tính thống kê
    của hệ nhiệt động lực học trong trạng thái cân bằng, và đây là
    một hàm vô hướng. Chữ partition trong tên gọi có thể là vì là
    hàm này liên hệ tới cách mà các hạt phân bổ giữa các mức năng
    lượng khác nhau. Về mặt toán học, thuật ngữ này đã dịch là
    ``hàm phân hoạch''.}
}

\newglossaryentry{mo-hinh-nang-luong}
{
    name={mô hình năng lượng},
    description={(energy-based model)}
}

\newglossaryentry{nang-luong-tu-do}
{
    name={năng lượng tự do},
    description={(free energy)}
}

\newglossaryentry{nang-luong-tu-do-bien-phan}
{
    name={năng lượng tự do biến phân},
    description={(variational free energy)}
}

\newglossaryentry{ham-nang-luong}
{
    name={hàm năng lượng},
    description={(energy-funtion)}
}

\newglossaryentry{toi-uu-loi}
{
    name={tối ưu lồi},
    description={(convex optimization)}
}

\newglossaryentry{toi-uu-so}
{
    name={tối ưu số},
    description={(numerical optimization)}
}

\newglossaryentry{chuan-gradient}
{
    name={chuẩn gradient},
    description={(gradient norm)}
}

\newglossaryentry{ham-loivex}
{
    name={hàm lồi},
    description={(convex function)}
}

\newglossaryentry{bieu-do-tan-xa}
{
    name={biểu đồ tán xạ},
    description={(scatter-plot) Biểu diễn dữ liệu bằng
    đồ thị gồm nhiều điểm, trong đó mỗi điểm ứng với
    giá trị quan sát được của một biến so với giá trị
    tương ứng của biến kia mà không nối các điểm đó
    lại với nhau.}
}

\newglossaryentry{phat-hien-vat-the}
{
    name={phát hiện vật thể},
    description={(object detection)}
}

\newglossaryentry{luot-huan-luyen}
{
    name={lượt huấn luyện},
    description={(epoch)}
}

\newglossaryentry{bien-tiem-an}
{
    name={biến tiềm ẩn},
    description={(latent variable)}
}

% https://web.stanford.edu/class/archive/ee/ee392m/ee392m.1034/Lecture8_ID.pdf
\newglossaryentry{kha-dinh}
{
    name={khả định},
    description={(identifiability)}
}

\newglossaryentry{bat-kha-dinh}
{
    name={bất khả định},
    description={(nonidentifiability)}
}

\newglossaryentry{doi-xung-khong-gian-trong-so}
{
    name={đối xứng không gian trọng số},
    description={(weight space symmetry)}
}

\newglossaryentry{tuyen-tinh-hieu-chinh}
{
    name={tuyến tính hiệu chỉnh},
    description={(rectified linear)}
}

\newglossaryentry{cuc-dai-dau-ra}
{
    name={cực đại đầu ra},
    description={(maxout)}
}

\newglossaryentry{don-vi-cuc-dai-dau-ra}
{
    name={đơn vị cực đại đầu ra},
    description={(maxout unit)}
}

\newglossaryentry{don-vi-tuyen-tinh-hieu-chinh}
{
    name={đơn vị tuyến tính hiệu chỉnh},
    description={(rectified linear unit)}
}

\newglossaryentry{don-vi-an}
{
    name={đơn vị ẩn},
    description={(hidden unit)}
}

\newglossaryentry{don-vi-dodetect}
{
    name={đơn vị dò},
    description={(detector unit)}
}

\newglossaryentry{don-vi-dodetect-nhi-phan}
{
    name={đơn vị dò nhị phân},
    description={(binary detector unit)}
}

\newglossaryentry{don-vi-gop}
{
    name={đơn vị gộp},
    description={(pooling unit)}
}

\newglossaryentry{don-vi-gop-cuc-dai}
{
    name={đơn vị gộp cực đại},
    description={(max-pooling unit)}
}

\newglossaryentry{don-vi-gop-nhi-phan}
{
    name={đơn vị gộp nhị phân},
    description={(binary pooling unit)}
}

\newglossaryentry{cao-nguyen}
{
    name={cao nguyên},
    description={(plateaus)}
}

\newglossaryentry{cuc-dai-cuc-bo}
{
    name={cực đại cục bộ},
    description={(local maximum)}
}

\newglossaryentry{xu-ly-ngon-ngu-tu-nhien}
{
    name={xử lí ngôn ngữ tự nhiên},
    description={(natural language processing - NLP)}
}

\newglossaryentry{phuong-phap-Newton-thoat-diem-yen-ngua}
{
    name={phương pháp Newton thoát điểm yên ngựa},
    description={(saddle-free Newton method)}
}

\newglossaryentry{diem-cuc-tieu}
{
    name={điểm cực tiểu},
    description={(minimum point)}
}

\newglossaryentry{ham-muc-tieu}
{
    name={hàm mục tiêu},
    description={(objective function)}
}

\newglossaryentry{siet-gradient}
{
    name={siết gradient},
    description={(gradient clipping)}
}

\newglossaryentry{siet-chuan-gradient}
{
    name={siết chuẩn gradient},
    description={(gradient norm clipping)}
}

\newglossaryentry{siet-gradient-cam-tinh}
{
    name={siết gradient cảm tính},
    description={(gradient clipping heuristic)}
}

\newglossaryentry{phu-thuoc-dai-han}
{
    name={phụ thuộc dài hạn},
    description={(long-term dependency)}
}

\newglossaryentry{tieu-bien-gradient}
{
    name={tiêu biến gradient},
    description={(vanishing gradient)}
}

\newglossaryentry{bung-no-gradient}
{
    name={bùng nổ gradient},
    description={(exploding gradient)}
}

\newglossaryentry{mo-hinh-do-thi-co-huong}
{
    name={mô hình đồ thị có hướng},
    description={(directed graphical model)}
}

\newglossaryentry{mo-hinh-do-thi-vo-huong}
{
    name={mô hình đồ thị vô hướng},
    description={(undirected graphical model)}
}

\newglossaryentry{mang-Markov}
{
    name={mạng Markov},
    description={(Markov network)}
}

\newglossaryentry{mo-hinh-xac-suat}
{
    name={mô hình xác suất},
    description={(probabilistic model)}
}

\newglossaryentry{mo-hinh-xac-suat-ma-hoa-thua-truc-tiep}
{
    name={mô hình xác suất mã hóa thưa trực tiếp},
    description={(directed sparse coding probabilistic model)}
}

\newglossaryentry{mo-hinh-xac-suat-da-tang}
{
    name={mô hình xác suất đa tầng},
    description={(deep probabilistic model)}
}

\newglossaryentry{mo-hinh-xac-suat-co-cau-truc}
{
    name={mô hình xác suất có cấu trúc},
    description={(structured probabilistic model)}
}

\newglossaryentry{mo-hinh-xac-suat-co-huong}
{
    name={mô hình xác suất có hướng},
    description={(directed probabilistic model)}
}

\newglossaryentry{mo-hinh-xac-suat-vo-huong}
{
    name={mô hình xác suất vô hướng},
    description={(undirected probabilistic model)}
}

\newglossaryentry{mo-hinh-do-thi}
{
    name={mô hình đồ thị},
    description={(graphical model)}
}
\newglossaryentry{mo-hinh-do-thi-nong}
{
    name={mô hình đồ thị nông},
    description={(shallow graphical model)}
}

\newglossaryentry{mo-hinh-do-thi-da-tang}
{
    name={mô hình đồ thị đa tầng},
    description={(deep graphical model)}
}

\newglossaryentry{mo-hinh-do-thi-trai-qua-phai}
{
    name={mô hình đồ thị trái-qua-phải},
    description={(left-to-right graphical model)}
}

\newglossaryentry{mo-hinh-do-thi-lai}
{
    name={mô hình đồ thị lai},
    description={(hybrid graphical model)}
}

\newglossaryentry{do-thi-dan-trai}
{
    name={đồ thị dàn trải},
    description={(unfold graph)}
}

\newglossaryentry{mo-hinh-do-thi-co-cau-truc}
{
    name={mô hình đồ thị có cấu trúc},
    description={(structured graphical model)}
}

\newglossaryentry{mo-hinh-do-thi-xac-suat}
{
    name={mô hình đồ thị xác suất},
    description={(probabilistic graphical model)}
}

\newglossaryentry{mo-hinh-do-thi-xac-suat-vo-huong}
{
    name={mô hình đồ thị xác suất vô hướng},
    description={(undirected probabilistic graphical model)}
}

\newglossaryentry{mo-hinh-do-thi-xac-suat-da-tang}
{
    name={mô hình đồ thị xác suất đa tầng},
    description={(deep probabilistic graphical model)}
}

\newglossaryentry{do-thi-vo-huong}
{
    name={đồ thị vô hướng},
    description={(undirected graph)}
}

\newglossaryentry{do-thi-co-huong}
{
    name={đồ thị có hướng},
    description={(directed graph)}
}

\newglossaryentry{do-thi-co-huong-phi-chu-trinh}
{
    name={đồ thị có hướng phi chu trình},
    description={(directed acyclic graph)}
}

\newglossaryentry{phan-phoi-xac-suat}
{
    name={phân phối xác suất},
    description={(probability distribution)}
}

\newglossaryentry{phan-phoi-xac-suat-tien-nghiem}
{
    name={phân phối xác suất tiên nghiệm},
    description={(prior probability distribution)}
}

\newglossaryentry{phan-phoi-hon-hop}
{
    name={phân phối hỗn hợp},
    description={(mixture distribution)}
}

\newglossaryentry{phan-phoi-xac-suat-bien}
{
    name={phân phối xác suất biên},
    description={(marginal probability distribution)}
}

\newglossaryentry{phan-phoi-xac-suat-dong-thoi}
{
    name={phân phối xác suất đồng thời},
    description={(joint probability distribution)}
}

\newglossaryentry{ham-mat-do-xac-suat}
{
    name={hàm mật độ xác suất},
    description={(probability density function)}
}

\newglossaryentry{phan-phoi-xac-suat-chua-chuan-hoa}
{
    name={phân phối xác suất chưa chuẩn hóa},
    description={(unnormalized probability distribution)}
}

\newglossaryentry{phan-phoi-xac-suat-chuan-hoa}
{
    name={phân phối xác suất chuẩn hóa},
    description={(normalized probability distribution)}
}

\newglossaryentry{n-gram}
{
    name={$n$-gram},
    description={($n$-gram)}
}

\newglossaryentry{hoi-cap}
{
    name={hồi cấp},
    description={(back-off)}
}

\newglossaryentry{n-gram-hoi-cap}
{
    name={$n$-gram hồi cấp},
    description={(back-off $n$-gram)}
}

\newglossaryentry{mang-lan-truyen-thuan}
{
    name={mạng lan truyền thuận},
    description={(feedforward network)}
}

\newglossaryentry{mang-lan-truyen-thuan-da-tang}
{
    name={mạng lan truyền thuận đa tầng},
    description={(feedforward deep network)}
}

\newglossaryentry{phan-ki-tuong-phan}
{
    name={phân kì tương phản},
    description={(contrastive divergence)}
}

\newacronym{cd}{CD}{contrastive divergence}

\newglossaryentry{phan-ki-tuong-phan-lien-tuc}
{
    name={phân kì tương phản liên tục},
    description={(persistent contrastive divergence)}
}

\newacronym{pcd}{PCD}{persistent contrastive divergence}

\newacronym{fpcd}{FPCD}{fast persistent contrastive divergence}

\newglossaryentry{may-boltzmann}
{
    name={máy Boltzmann},
    description={(Boltzmann machine)}
}

\newglossaryentry{may-boltzmann-ket-noi-day-du}
{
    name={máy Boltzmann kết nối đầy đủ},
    description={(fully connected Boltzmann machine)}
}

\newglossaryentry{may-boltzmann-nhi-phan}
{
    name={máy Boltzmann nhị phân},
    description={(binary Boltzmann machine)}
}

\newglossaryentry{may-boltzmann-da-tang}
{
    name={máy Boltzmann đa tầng},
    description={(deep Boltzmann machine)}
}

\newacronym{dbm}{DBM}{deep Boltzmann machine}

\newglossaryentry{may-boltzmann-da-tang-dinh-tam}
{
    name={máy Boltzmann đa tầng định tâm},
    description={(centered deep Boltzmann machine)}
}

\newglossaryentry{may-boltzmann-da-tang-da-du-doan}
{
    name={máy Boltzmann đa tầng đa dự đoán},
    description={(multi-prediction deep Boltzmann machine)}
}

\newacronym{mp-dbm}{MP-DBM}{multi-prediction deep Boltzmann machine}

\newglossaryentry{may-boltzmann-gioi-han}
{
    name={máy Boltzmann giới hạn},
    description={(restricted Boltzmann machine)}
}

\newacronym{rbm}{RBM}{restricted Boltzmann machine}

\newacronym{ssrbm}{ssRBM}{spike and slab restricted Boltzmann machine}

\newacronym{mcrbm}{mcRBM}{mean and covariance restricted Boltzmann machine}

\newacronym{crbm}{cRBM}{covariance restricted Boltzmann machine}

\newglossaryentry{may-boltzmann-ban-gioi-han}
{
    name={máy Boltzmann bán giới hạn},
    description={(semi-restricted Boltzmann machine)}
}

\newglossaryentry{may-boltzmann-tich-chap}
{
    name={máy Boltzmann tích chập},
    description={(convolutional Boltzmann machine)}
}

\newglossaryentry{cau-truc-cuc-bo}
{
    name={cấu trúc cục bộ},
    description={(local structure)}
}

\newglossaryentry{cau-truc-toan-cuc}
{
    name={cấu trúc toàn cục},
    description={(global structure)}
}

\newglossaryentry{diem-toi-han}
{
    name={điểm tới hạn},
    description={(critical point)}
}

\newglossaryentry{phan-phoi-xac-suat-co-dieu-kien}
{
    name={phân phối xác suất có điều kiện},
    description={(conditional probability distribution)}
}

\newglossaryentry{phan-phoi-xac-suat-co-dieu-kien-cuc-bo}
{
    name={phân phối xác suất có điều kiện cục bộ},
    description={(local conditional probability distribution)}
}

\newglossaryentry{cau-truc-vo-luan-li}
{
    name={cấu trúc vô luân lí},
    description={(immorality)}
}

\newglossaryentry{do-thi-luan-li}
{
    name={đồ thị luân lí},
    description={(moralized graph)}
}

\newglossaryentry{do-thi-nhan-tu}
{
    name={đồ thị nhân tử},
    description={(factor graph)}
}

\newglossaryentry{lay-mau-pha-he}
{
    name={lấy mẫu phả hệ},
    description={(ancestral sampling)}
}

\newglossaryentry{phan-phoi-dong-thoi}
{
    name={phân phối đồng thời},
    description={(joint distribution)}
}

\newglossaryentry{thu-tu-cau-truc-to-po}
{
    name={thứ tự cấu trúc tô-pô},
    description={(topological ordering)}
}

\newglossaryentry{phan-phoi-hau-nghiem}
{
    name={phân phối hậu nghiệm},
    description={(posterior distribution)}
}

\newglossaryentry{phan-phoi-tien-nghiem}
{
    name={phân phối tiên nghiệm},
    description={(prior distribution)}
}

\newglossaryentry{phan-phoi-bien}
{
    name={phân phối biên},
    description={(marginal distribution)}
}

\newglossaryentry{lay-mau-Gibbs}
{
    name={lấy mẫu Gibbs},
    description={(Gibbs sampling)}
}

\newglossaryentry{lay-mau-Gibbs-theo-khoi}
{
    name={lấy mẫu Gibbs theo khối},
    description={(block Gibbs sampling)}
}

\newglossaryentry{phan-phoi-giai-thua}
{
    name={phân phối nhân tử},
    description={(factorial distribution)}
}

\newglossaryentry{phan-phoi-Gibbs}
{
    name={phân phối Gibbs},
    description={(Gibbs distribution)}
}

\newglossaryentry{phan-phoi-Boltzmann}
{
    name={phân phối Boltzmann},
    description={(Boltzmann distribution)}
}

\newglossaryentry{mo-hinh-log-tuyen-tinh}
{
    name={mô hình log-tuyến tính},
    description={(log-linear model)}
}

\newglossaryentry{mo-hinh-sinh-mau}
{
    name={mô hình sinh mẫu},
    description={(generative model)}
}
\newglossaryentry{mo-hinh-sinh-mau-da-tang}
{
    name={mô hình sinh mẫu đa tầng},
    description={(deep generative model)}
}

\newglossaryentry{mo-hinh-sinh-mau-nong-co-huong}
{
    name={mô hình sinh mẫu nông có hướng},
    description={(shallow directed generative model)}
}

\newglossaryentry{mang-sinh-mau}
{
    name={mạng sinh mẫu},
    description={(generative network)}
}

\newglossaryentry{mang-sinh-mau-tich-chap}
{
    name={mạng sinh mẫu tích chập},
    description={(convolutional generative network)}
}

\newglossaryentry{mang-sinh-mau-co-huong}
{
    name={mạng sinh mẫu có hướng},
    description={(directed generative network)}
}

\newglossaryentry{mang-sinh-mau-tien-doan}
{
    name={mạng sinh mẫu tiên đoán},
    description={(predictive generative network)}
}

\newglossaryentry{mang-sinh-mau-kha-vi}
{
    name={mạng sinh mẫu khả vi},
    description={(differentiable generative network)}
}

\newglossaryentry{mang-sinh-mau-ngau-nhien}
{
    name={mạng sinh mẫu ngẫu nhiên},
    description={(stochastic generative network)}
}

\newacronym{gsn}{GSN}{stochastic generative network}

\newglossaryentry{mang-sinh-mau-khop-moment}
{
    name={mạng sinh mẫu khớp moment},
    description={(generative moment matching network)}
}

\newglossaryentry{khop-moment}
{
    name={khớp moment},
    description={(moment matching)}
}

\newglossaryentry{mang-Bayes}
{
    name={mạng Bayes},
    description={(Bayesian network)}
}
\newglossaryentry{mang-Bayes-dong}
{
    name={mạng Bayes động},
    description={(dynamic Bayesian network)}
}

\newglossaryentry{mang-Bayes-hoan-toan-kha-kien}
{
    name={mạng Bayes hoàn toàn khả kiến},
    description={(fully-visible Bayes network)}
}

\newglossaryentry{mo-hinh-Gauss}
{
    name={mô hình Gauss},
    description={(Gaussian model)}
}

\newglossaryentry{mo-hinh-Gauss-hon-hop}
{
    name={mô hình Gauss hỗn hợp},
    description={(Gaussian mixture model)}
}

\newglossaryentry{bien-an}
{
    name={biến ẩn},
    description={(hidden variable)}
}

\newglossaryentry{bien-kha-kien}
{
    name={biến khả kiến},
    description={(visible variable)}
}

\newglossaryentry{cay-quy-dan}
{
    name={cây quy dẫn},
    description={(reduction tree)}
}

\newglossaryentry{hoc-da-tap}
{
    name={học đa tạp},
    description={(manifold learning)}
}

\newglossaryentry{gia-thuyet-da-tap}
{
    name={giả thuyết đa tạp},
    description={(manifold hypothesis)}
}

\newglossaryentry{hoc-da-tap-phi-tham-so}
{
    name={học đa tạp phi tham số},
    description={(nonparametric manifold learning)}
}

\newglossaryentry{da-tap}
{
    name={đa tạp},
    description={(manifold)}
}

\newglossaryentry{phan-giai-tu-co-nghia-nhap-nhang}
{
    name={phân giải từ có nghĩa nhập nhằng},
    description={(word-sense disambiguation)}
}

\newglossaryentry{bieu-dien-phan-tan}
{
    name={biểu diễn phân tán},
    description={(distributed representation)}
}

\newglossaryentry{bieu-dien-phan-tan-da-tang}
{
    name={biểu diễn phân tán đa tầng},
    description={(deep distributed representation)}
}

\newglossaryentry{thuat-toan-suy-dien}
{
    name={thuật toán suy diễn},
    description={(inference algorithm)}
}

\newglossaryentry{thuat-toan-suy-dien-xap-xi}
{
    name={thuật toán suy diễn xấp xỉ},
    description={(approximate inference algorithm)}
}

\newglossaryentry{thuat-toan-suy-dien-bien-phan}
{
    name={thuật toán suy diễn biến phân},
    description={(variational inference algorithm)}
}

\newglossaryentry{truyen-ba-niem-tin-theo-vong}
{
    name={truyền bá niềm tin theo vòng},
    description={(loopy belief propagation)}
}

\newglossaryentry{ma-tran-cheo}
{
    name={ma trận chéo},
    description={(diagonal matrix)}
}

\newglossaryentry{u-lay-mau-theo-do-quan-trong}
{
    name={ủ lấy mẫu theo độ quan trọng},
    description={(annealed importance sampling)}
}

\newacronym{ais}{AIS}{annealed importance sampling}

\newglossaryentry{lay-mau-theo-do-quan-trong}
{
    name={lấy mẫu theo độ quan trọng},
    description={(importance sampling)}
}

\newglossaryentry{lay-mau-theo-do-quan-trong-co-chech}
{
    name={lấy mẫu theo độ quan trọng có chệch},
    description={(biased importance sampling)}
}

\newglossaryentry{lay-mau-theo-do-quan-trong-toi-uu}
{
    name={lấy mẫu theo độ quan trọng tối ưu},
    description={(optimal importance sampling)}
}

\newglossaryentry{do-chech}
{
    name={độ chệch},
    description={(bias) trong thống kê}
}

\newglossaryentry{chech}
{
    name={chệch},
    description={(biased)}
}

\newglossaryentry{he-so-tu-do}
{
    name={hệ số tự do},
    description={(bias) trong giải tích}
}

\newglossaryentry{he-so-chan}
{
    name={hệ số chặn},
    description={(intercept)}
}

\newglossaryentry{khong-chech}
{
    name={không chệch},
    description={(unbiased)}
}

\newglossaryentry{khong-chech-tiem-can}
{
    name={không chệch tiệm cận},
    description={(asymptotically unbiased)}
}

\newglossaryentry{toc-do-hoc}
{
    name={tốc độ học},
    description={(learning rate)}
}

\newglossaryentry{bo-uoc-luong}
{
    name={bộ ước lượng},
    description={(estimator)}
}

\newglossaryentry{dieu-kien-du}
{
    name={điều kiện đủ},
    description={(sufficient condition)}
}

\newglossaryentry{dieu-kien-can}
{
    name={điều kiện cần},
    description={(necessary condition)}
}

\newglossaryentry{hoi-tu}
{
    name={hội tụ},
    description={(convergence)}
}

\newglossaryentry{hau-nghiem-cuc-dai}
{
    name={hậu nghiệm cực đại},
    description={(maximum a posteriori)}
}

\newacronym{map}{MAP}{maximum a posteriori}

\newglossaryentry{xap-xi-hau-nghiem-cuc-dai}
{
    name={xấp xỉ hậu nghiệm cực đại},
    description={(maximum a posteriori approximation)}
}

\newglossaryentry{toc-do-hoi-tu}
{
    name={tốc độ hội tụ},
    description={(convergence rate)}
}

\newglossaryentry{thu-sai}
{
    name={thử sai},
    description={(trial and error)}
}

\newglossaryentry{duong-cong-hoc-tap}
{
    name={đường cong học tập},
    description={(learning curve)}
}

\newglossaryentry{tat-ngau-nhien}
{
    name={tắt ngẫu nhiên},
    description={(dropout) - hoặc có thể gọi đầy đủ là ``cơ chế tắt ngẫu nhiên''}
}

\newglossaryentry{tat-ngau-nhien-nhanh}
{
    name={tắt ngẫu nhiên nhanh},
    description={(fast dropout) - hoặc có thể gọi đầy đủ là ``cơ chế tắt ngẫu nhiên nhanh''}
}

\newglossaryentry{tat-ngau-nhien-tang-cuong}
{
    name={tắt ngẫu nhiên tăng cường},
    description={(dropout boosting) - hoặc có thể gọi đầy đủ là ``cơ chế tắt ngẫu nhiên tăng cường''}
}

\newglossaryentry{truc-tuyen}
{
    name={trực tuyến},
    description={(online). Thời điện thoại còn sử dụng dây điện, người Việt mình khi
    bắt máy hay hỏi ``ai ở đầu dây bên kia vậy ạ''. Thời nay dùng mạng không dây,
    ta không còn nghe thấy câu đó phổ biến nữa. Cũng cùng một nghĩa đó,
    tiếng Anh dùng online, tức là on-the-line (trên dây) đồng nghĩa với việc
    người kia đang có mặt và đang sẵn sàng kết nối trao đổi thông tin. Tuyến (線)
    nghĩa là sợi dây, trực (値) nghĩa là có mặt, hiện diện. Trực tuyến tức là có mặt
    trên dây, đem dịch chữ online là rất hay. Ngoài ra, theo nghĩa trên, online
    (và cả từ trực tuyến) cũng hàm ý liên tục kết nối và chia sẻ thông tin, vây nên
    xét trong ngữ cảnh học sâu, khi mà mô hình học liên tục nhận mẫu huấn luyện
    trong suốt thời gian vận hành, thì người ta gọi đó là các
    phương pháp học trực tuyến (online learning method) xem ra rất hợp lí.}
}

\newglossaryentry{dung-sai}
{
    name={dung sai},
    description={(tolerance)}
}

\newglossaryentry{sai-so-vuot-muc}
{
    name={sai số vượt mức},
    description={(excess error)}
}

\newglossaryentry{giai-tich-tiem-can}
{
    name={phân tích tiệm cận},
    description={(asymptotic analysis)}
}

\newglossaryentry{dong-luong}
{
    name={động lượng},
    description={(momentum)}
}

\newglossaryentry{duong-dong-muc}
{
    name={đường đồng mức},
    description={(contour line)}
}

\newglossaryentry{phuong-trinh-vi-phan}
{
    name={phương trình vi phân},
    description={(differential equation)}
}

\newglossaryentry{suc-can-nhot}
{
    name={sức cản nhớt},
    description={(viscous drag)}
}

\newglossaryentry{moi-truong-co-tinh-can}
{
    name={môi trường có tính cản},
    description={(resistant medium)}
}

\newglossaryentry{suy-giam-trong-so}
{
    name={suy giảm trọng số},
    description={(weight decay)}
}

\newglossaryentry{thong-ke-can-bien}
{
    name={thống kê cận biên},
    description={(marginal statistics)}
}

\newglossaryentry{khong-gian-Euclid}
{
    name={không gian Euclid},
    description={(Euclidean space)}
}

\newglossaryentry{khong-gian-nhung-tu}
{
    name={không gian nhúng từ},
    description={(word embedding space)}
}

\newglossaryentry{khong-gian-dac-trung}
{
    name={không gian đặc trưng},
    description={(feature space)}
}

\newglossaryentry{mang-niem-tin}
{
    name={mạng niềm tin},
    description={(belief network)}
}

\newglossaryentry{mang-niem-tin-da-tang}
{
    name={mạng niềm tin đa tầng},
    description={(deep belief network)}
}

\newacronym{dbn}{BDN}{deep belief network}

\newglossaryentry{mang-niem-tin-hoan-toan-kha-kien}
{
    name={mạng niềm tin hoàn toàn khả kiến},
    description={(fully visible belief network)}
}

\newglossaryentry{mang-niem-tin-sigmoid}
{
    name={mạng niềm tin sigmoid},
    description={(sigmoid belief network)}
}

\newglossaryentry{mang-niem-tin-tich-chap-da-tang}
{
    name={mạng niềm tin tích chập đa tầng},
    description={(convolutional deep belief network)}
}

\newglossaryentry{tich-cua-cac-chuyen-gia}
{
    name={tích của các chuyên gia},
    description={(product of experts)}
}

\newglossaryentry{phep-chieu-so-chieu-thap}
{
    name={phép chiếu số chiều thấp},
    description={(low-dimentional projection)}
}

\newglossaryentry{hai-hoa}
{
    name={hài hòa},
    description={(harmony)}
}

\newglossaryentry{hieu-ung-thanh-minh}
{
    name={hiệu ứng thanh minh},
    description={(explaining away effect)}
}

\newglossaryentry{tuong-tac-thanh-minh}
{
    name={tương tác thanh minh},
    description={(explaining away interaction)}
}

\newglossaryentry{cau-truc-V}
{
    name={cấu trúc V},
    description={(V structure)}
}

\newglossaryentry{day-cung}
{
    name={dây cung},
    description={(chord)}
}

\newglossaryentry{do-thi-day-cung}
{
    name={đồ thị dây cung},
    description={(choral graph)}
}

\newglossaryentry{do-thi-tam-giac}
{
    name={đồ thị tam giác},
    description={(triangulated graph)}
}

\newglossaryentry{pha-duong}
{
    name={pha dương},
    description={(positive phase)}
}

\newglossaryentry{pha-am}
{
    name={pha âm},
    description={(negative phase)}
}

\newglossaryentry{mode-gia-mao}
{
    name={mode giả mạo},
    description={(spurious mode)}
}

\newglossaryentry{khoang-cach-Levenshtein}
{
    name={khoảng cách Levenshtein},
    description={(Levenshtein distance). Khoảng cách Levenshtein (hay ``edit distance'') thể hiện khoảng
    cách khác biệt giữa $2$ chuỗi kí tự. Khoảng cách Levenshtein giữa chuỗi $\mathrm{S}$ và
    chuỗi $\mathrm{T}$ là số bước ít nhất để biến chuỗi $\mathrm{S}$ thành chuỗi $\mathrm{T}$
    thông qua $3$ thao tác là: chèn kí tự, xoá kí tự, và thay kí tự này bằng kí tự khác.}
}

\newglossaryentry{hop-li-cuc-dai-ngau-nhien}
{
    name={hợp lí cực đại ngẫu nhiên},
    description={(stochastic maximum likelihood)}
}

\newacronym{sml}{SML}{stochastic maximum likelihood}

\newglossaryentry{hop-li-cuc-dai-ngau-nhien-bien-phan}
{
    name={hợp lí cực đại ngẫu nhiên biến phân},
    description={(variational stochastic maximum likelihood)}
}

\newglossaryentry{bo-tu-ma-hoa}
{
    name={bộ tự mã hóa},
    description={(autoencoder)}
}

\newglossaryentry{bo-tu-ma-hoa-bien-phan}
{
    name={bộ tự mã hóa biến phân},
    description={(variational autoencoder)}
}

\newacronym{vae}{VAE}{variational autoencoder}

\newglossaryentry{bo-tu-ma-hoa-phan-tan}
{
    name={bộ tự mã hóa phân tán},
    description={(distributed autoencoder)}
}

\newglossaryentry{bo-tu-ma-hoa-khu-nhieu}
{
    name={bộ tự mã hóa khử nhiễu},
    description={(denoising autoencoder)}
}

\newacronym{dae}{DAE}{denoising autoencoder}

\newglossaryentry{bo-tu-ma-hoa-co-rut}
{
    name={bộ tự mã hóa co rút},
    description={(contractive autoencoder)}
}

\newacronym{cae}{CAE}{contractive autoencoder}

\newglossaryentry{bo-tu-ma-hoa-co-trong-so}
{
    name={bộ tự mã hóa có trọng số},
    description={(importance-weighted autoencoder)}
}

\newglossaryentry{bo-tu-ma-hoa-ngau-nhien}
{
    name={bộ tự mã hóa ngẫu nhiên},
    description={(stochastic autoencoder)}
}

\newglossaryentry{bo-tu-ma-hoa-da-tang}
{
    name={bộ tự mã hóa đa tầng},
    description={(deep autoencoder)}
}

\newglossaryentry{bo-tu-ma-hoa-nong}
{
    name={bộ tự mã hóa nông},
    description={(shallow autoencoder)}
}

\newglossaryentry{bo-tu-ma-hoa-duoi-muc}
{
    name={bộ tự mã hóa dưới mức},
    description={(undercomplete autoencoder)}
}

\newglossaryentry{bo-tu-ma-hoa-thua}
{
    name={bộ tự mã hóa thưa},
    description={(sparse autoencoder)}
}

\newglossaryentry{bo-tu-ma-hoa-co-kiem-soat}
{
    name={bộ tự mã hóa có kiểm soát},
    description={(Regularized autoencoder)}
}

\newglossaryentry{bo-ma-hoa}
{
    name={bộ mã hóa},
    description={(encoder)}
}

\newglossaryentry{bo-ma-hoa-phi-tham-so}
{
    name={bộ mã hóa phi tham số},
    description={(nonparametric encoder)}
}

\newglossaryentry{bo-ma-hoa-truy-hoi}
{
    name={bộ mã hóa truy hồi},
    description={(recurrent encoder)}
}

\newglossaryentry{bo-ma-hoa-ngau-nhien}
{
    name={bộ mã hóa ngẫu nhiên},
    description={(stochastic encoder)}
}

\newglossaryentry{bo-giai-ma}
{
    name={bộ giải mã},
    description={(decoder)}
}

\newglossaryentry{bo-giai-ma-ngau-nhien}
{
    name={bộ giải mã ngẫu nhiên},
    description={(stochastic decoder)}
}

\newglossaryentry{bo-giai-ma-truy-hoi}
{
    name={bộ giải mã truy hồi},
    description={(recurrent decoder)}
}

\newglossaryentry{khop-diem-so}
{
    name={khớp điểm số},
    description={(score matching)}
}

\newglossaryentry{khop-diem-so-khu-nhieu}
{
    name={khớp điểm số khử nhiễu},
    description={(denoising score matching)}
}

\newglossaryentry{chuoi-Markov}
{
    name={chuỗi Markov},
    description={(Markov chain)}
}

\newglossaryentry{chuoi-Markov-Monte-Carlo}
{
    name={chuỗi Markov Monte Carlo},
    description={(Markov chain Monte Carlo)}
}

\newglossaryentry{gradient-lien-hop}
{
    name={gradient liên hợp},
    description={(conjugate gradient)}
}

\newglossaryentry{khop-ti-le}
{
    name={khớp tỉ lệ},
    description={(ratio matching)}
}

\newglossaryentry{gradient-lien-hop-chia-ti-le}
{
    name={gradient liên hợp chia tỉ lệ},
    description={(scaled conjugate gradient - SCG)}
}

\newglossaryentry{huong-lien-hop}
{
    name={hướng liên hợp},
    description={(conjugate direction)}
}

\newglossaryentry{do-duong}
{
    name={dò đường},
    description={(line search)}
}

\newacronym{bfgs}{BFGS}{Broyden--Fletcher--Goldfarb--Shanno}

\newglossaryentry{BFGS-huu-han-bo-nho}
{
    name={BFGS giới hạn bộ nhớ},
    description={(Limited Memory BFGS - L-BFGS)}
}

\newacronym{lbfgs}{L-BFGS}{Limited Memory Broyden--Fletcher--Goldfarb--Shanno}

\newglossaryentry{xac-suat-bien}
{
    name={xác suất biên},
    description={(marginal probability)}
}

\newglossaryentry{tai-tham-so-hoa-tuy-bien}
{
    name={tái tham số hóa thích nghi},
    description={(adaptive reparameterization)}
}

\newglossaryentry{tai-tham-so-hoa}
{
    name={tái tham số hóa},
    description={(reparameterization)}
}

\newglossaryentry{phat-tan}
{
    name={phát tán},
    description={(broadcasting) Trong \gls{hoc-sau}, người ta sử dụng thêm một số kí hiệu không phổ biến
    theo thông lệ. Cụ thể, sách này cho phép thực hiện cộng giữa một ma
    trận và một vector, tạo ra một ma trận khác: $\boldsymbol{C} =
    \boldsymbol{A} + \boldsymbol{b}$, trong đó $C_{i,j} = A_{i,j} + b_j$.
    Nói cách khác, vector $\boldsymbol {b}$ được cộng vào từng hàng của ma
    trận $\boldsymbol{A}$. Sử dụng kí hiệu này, ta không cần phải tạo ra
    một ma trận mới, với mỗi cột là một bản sao của $\boldsymbol b$, trước
    khi thực hiện phép cộng. Kiểu sao chép vector $\boldsymbol b$ ngầm
    định tới nhiều vị trí như thế này được gọi là phép \textit{\gls{phat-tan}}
    (broadcasting)}
}

\newglossaryentry{ban-do-dac-trung}
{
    name={bản đồ đặc trưng},
    description={(feature map)}
}

\newglossaryentry{ma-hoa-thua}
{
    name={mã hóa thưa},
    description={(sparse coding)}
}

\newglossaryentry{ma-hoa-thua-nhi-phan}
{
    name={mã hóa thưa nhị phân},
    description={(binary sparse coding)}
}

\newglossaryentry{hon-hop-mot-xung}
{
    name={hỗn hợp một xung},
    description={(spike và slab). Xem \gls{xung-nhon} (spike) và \gls{xung-det} (slab)}
}

\newglossaryentry{xung-nhon}
{
    name={xung nhọn},
    description={(spike)}
}

\newglossaryentry{xung-det}
{
    name={xung dẹt},
    description={(slab)}
}

\newglossaryentry{khoang-cach-Hamming}
{
    name={khoảng cách Hamming}, description={(Hamming distance). Trong lí thuyết
    thông tin, khoảng cách Hamming giữa hai chuỗi có độ dài bằng nhau là số các
    kí tự ở vị trí tương đương có giá trị khác nhau. Nói cách khác, khoảng cách
    Hamming đo số lượng thay thế cần phải có để đổi giá trị của một dãy kí tự
    sang một dãy kí tự khác.}
}

\newglossaryentry{tinh-chinh}
{
    name={tinh chỉnh},
    description={(fine-tuning)}
}

\newglossaryentry{thuat-toan-tham-lam}
{
    name={thuật toán tham lam},
    description={(greedy algorithm)}
}

\newglossaryentry{perceptron-da-tang}
{
    name={perceptron đa tầng},
    description={(multilayer perceptron)}
}

\newacronym{mlp}{MLP}{multilayer perceptron}

\newglossaryentry{hoc-chuyen-giao}
{
    name={học chuyển giao},
    description={(transfer learning)}
}

\newglossaryentry{khong-don-dieu}
{
    name={không đơn điệu},
    description={(nonmonotonic)}
}

\newglossaryentry{don-vi-sigmoid}
{
    name={đơn vị sigmoid},
    description={(sigmoid unit)}
}

\newglossaryentry{don-vi-ro-ri}
{
    name={đơn vị rò rỉ},
    description={(leaky unit)}
}

\newglossaryentry{don-vi-kha-kien}
{
    name={đơn vị khả kiến},
    description={(visible unit)}
}

\newglossaryentry{ket-noi-nhay-coc}
{
    name={kết nối nhảy cóc},
    description={(skip connection)}
}

\newglossaryentry{ket-noi-nhay-coc-xuyen-thoi-gian}
{
    name={kết nối nhảy cóc xuyên thời gian},
    description={(Skip Connections through Time)}
}

\newglossaryentry{dau-bo-tro}
{
    name={đầu bổ trợ},
    description={(auxiliary head)}
}

\newglossaryentry{phuong-phap-min-dan}
{
    name={phương pháp mịn dần},
    description={(continuation method)}
}

\newglossaryentry{toi-luyen-mo-phong}
{
    name={tôi luyện mô phỏng},
    description={(simulated annealing)}
}

\newglossaryentry{ngan-xep}
{
    name={ngăn xếp},
    description={(stack)}
}

\newglossaryentry{ngan-xep-loi}
{
    name={ngăn xếp lõi},
    description={(kernel stack)}
}

\newglossaryentry{hoc-theo-giao-trinh}
{
    name={học theo giáo trình},
    description={(curriculum learning)}
}

\newglossaryentry{tao-khuon}
{
    name={tạo khuôn},
    description={(shaping)}
}

\newglossaryentry{uoc-luong-tuong-phan-nhieu}
{
    name={ước lượng tương phản nhiễu},
    description={(noise-contrastive estimation)}
}

\newacronym{nce}{NCE}{noise-contrastive estimation}

\newglossaryentry{uoc-luong-tu-tuong-phan}
{
    name={ước lượng tự tương phản},
    description={(self-contrastive estimation)}
}

\newglossaryentry{phan-phoi-nhieu}
{
    name={phân phối nhiễu},
    description={(noise distribution)}
}

\newglossaryentry{mang-doi-khang-sinh-mau}
{
    name={mạng đối kháng sinh mẫu},
    description={(generative adversarial network)}
}

\newacronym{gan}{GAN}{generative adversarial network}

\newacronym{dcgan}{DCGAN}{deep convolutional generative adversarial network}

\newacronym{lapgan}{LAPGAN}{Laplacian generative adversarial network}

\newglossaryentry{lay-mau-bac-cau}
{
    name={lấy mẫu bắc cầu},
    description={(bridge sampling)}
}

\newglossaryentry{tuong-cuop}
{
    name={tướng cướp},
    description={(bandit). Máy đánh bạc $1$ cần gạt là loại thường thấy trong sòng bài (casino). Người chơi đa phần sẽ thua tiền, và vì vậy nên họ gọi nó là ``đồ ăn cướp'', bandit có nghĩa là ``tướng cướp''. Cũng giống như xúc xắc thường được dùng làm ví dụ cho các bài toán xác suất thống kê cơ bản, thì ở đây người ta thường dùng bandit làm ví dụ cơ bản cho các bài toán \gls{hoc-tang-cuong}.}
}

\newglossaryentry{tuong-cuop-ngu-canh}
{
    name={tướng cướp ngữ cảnh},
    description={(contextual bandit). Xem \gls{tuong-cuop}}
}

\newglossaryentry{bieu-dien-dung-chung}
{
    name={biểu diễn dùng chung},
    description={(shared representation)}
}

\newglossaryentry{cay-do-den}
{
    name={cây đỏ đen},
    description={(red-black tree)}
}

\newglossaryentry{bieu-dien-da-tang}
{
    name={biểu diễn đa tầng},
    description={(deep representation)}
}

\newglossaryentry{can-duoi-thuc-nghiem}
{
    name={cận dưới thực nghiệm},
    description={(evidence lower bound)}
}

\newacronym{elbo}{ELBO}{evidence lower bound}

\newglossaryentry{day-ep-buoc}
{
    name={dạy ép buộc},
    description={(teacher forcing)}
}

\newglossaryentry{may-turing-pho-quat}
{
    name={máy Turing phổ quát},
    description={(universal Turing machine)}
}

\newglossaryentry{may-turing}
{
    name={máy Turing},
    description={(Turing machine)}
}

\newglossaryentry{RBM-phan-biet}
{
    name={RBM phân biệt},
    description={(discriminative RBM)}
}

\newglossaryentry{mang-bac-thang}
{
    name={mạng bậc thang},
    description={(ladder network)}
}

\newglossaryentry{ma-tran-Hesse}
{
    name={ma trận Hesse},
    description={(Hessian matrix)}
}

\newglossaryentry{ma-tran-Jacobi}
{
    name={ma trận Jacobi},
    description={(Jacobi matrix)}
}

\newglossaryentry{nhung-tu}
{
    name={nhúng từ},
    description={(word embedding)}
}

\newglossaryentry{vector-nhung-tu}
{
    name={vector nhúng từ},
    description={(word embedding vector)}
}

\newglossaryentry{vector-don-troi}
{
    name={vector đơn trội},
    description={(one-hot vector)}
}

\newglossaryentry{ma-don-troi}
{
    name={mã đơn trội},
    description={(one-hot code)}
}

\newglossaryentry{bieu-dien-don-troi}
{
    name={biểu diễn đơn trội},
    description={(one-hot representation)}
}

\newglossaryentry{don-troi}
{
    name={đơn trội},
    description={(one-hot)}
}

\newglossaryentry{thich-ung-mien}
{
    name={thích ứng miền},
    description={(domain adaptation)}
}

\newglossaryentry{hoc-da-nhiem}
{
    name={học đa nhiệm},
    description={(multitask learning)}
}

\newglossaryentry{chuyen-dich-khai-niem}
{
    name={chuyển dịch khái niệm},
    description={(concept drift)}
}

\newglossaryentry{truong-trung-binh}
{
    name={trường trung bình},
    description={(mean field)}
}

\newglossaryentry{xap-xi-truong-trung-binh}
{
    name={xấp xỉ trường trung bình},
    description={(mean field) hay ``phép xấp xỉ trường trung bình''}
}

\newglossaryentry{suy-dien-truong-trung-binh}
{
    name={suy diễn trường trung bình},
    description={(mean-field inference)}
}

\newglossaryentry{suy-dien-truy-hoi-truong-trung-binh}
{
    name={suy diễn truy hồi trường trung bình},
    description={(mean-field recurrent inference)}
}

\newglossaryentry{suy-dien-hau-nghiem-cuc-dai}
{
    name={suy diễn hậu nghiệm cực đại},
    description={(maximization a posteriori inference - MAP inference)}
}

\newglossaryentry{uoc-luong-hau-nghiem-cuc-dai}
{
    name={ước lượng hậu nghiệm cực đại},
    description={(maximization a posteriori estimation)}
}

\newglossaryentry{truong-ngau-nhien-Markov}
{
    name={trường ngẫu nhiên Markov},
    description={(Markov random field)}
}

\newacronym{mrf}{MRF}{Markov random field}

\newglossaryentry{ham-loiker}
{
    name={hàm lõi},
    description={(kernel function)}
}

\newglossaryentry{may-loiker}
{
    name={máy lõi},
    description={(kernel machine) hoặc có thể gọi là ``mô hình hàm lõi''}
}

\newglossaryentry{thu-thuat-chon-ham-loiker}
{
    name={thủ thuật chọn hàm lõi},
    description={(kernel trick)}
}

\newglossaryentry{hoc-one-shot}
{
    name={học 1-mẫu},
    description={(one-shot learning)}
}

\newglossaryentry{hoc-zero-shot}
{
    name={học 0-mẫu},
    description={(zero-shot learning)}
}

\newglossaryentry{hoc-phi-du-lieu}
{
    name={học phi dữ liệu},
    description={(zero-data learning)}
}

\newglossaryentry{hoc-da-phuong-thuc}
{
    name={học đa phương thức},
    description={(multimodal learning)}
}

\newglossaryentry{phan-tu-tuyen-tinh-thich-nghi}
{
    name={phần tử tuyến tính thích nghi},
    description={(adaptive linear element)}
}

\newacronym{adaline}{ADALINE}{adaptive linear element}

\newglossaryentry{mang-trang-thai-vong-hoi}
{
    name={mạng trạng thái vọng hồi},
    description={(echo state network)}
}

\newacronym{esn}{ESN}{echo state network}

\newglossaryentry{dieu-khien-hoc}
{
    name={điều khiển học},
    description={(cybernetics)}
}

\newglossaryentry{thuyet-ket-noi}
{
    name={thuyết kết nối},
    description={(connectionism)}
}

\newglossaryentry{xu-ly-phan-tan-song-song}
{
    name={xử lí phân tán song song},
    description={(parallel distributed processing)}
}

\newglossaryentry{suy-luan-bieu-tuong}
{
    name={suy luận biểu tượng},
    description={(symbolic reasoning)}
}

\newglossaryentry{bien-to}
{
    name={biến tố},
    description={(factor of variation)}
}

\newglossaryentry{suy-dien-bien-phan}
{
    name={suy diễn biến phân},
    description={(variational inference)}
}

\newglossaryentry{hoc-bien-phan}
{
    name={học biến phân},
    description={(variational learning)}
}

\newglossaryentry{suy-dien-bien-phan-co-cau-truc}
{
    name={suy diễn biến phân có cấu trúc},
    description={(structured variational inference)}
}

\newglossaryentry{tri-tue-nhan-tao}
{
    name={trí tuệ nhân tạo},
    description={(artificial intelligence)}
}

\newglossaryentry{gian-luan-Bayes}
{
    name={giản luận Bayes},
    description={(naive Bayes)}
}

\newglossaryentry{bo-nho-ngan-han-huong-dai-han}
{
    name={bộ nhớ ngắn hạn hướng dài hạn},
    description={(long short-term memory)}
}

\newacronym{lstm}{LSTM}{long short-term memory}

\newglossaryentry{chuoi-chuoi}
{
    name={chuỗi-chuỗi},
    description={(sequence-to-sequence)}
}

\newglossaryentry{dich-may}
{
    name={dịch máy},
    description={(machine translation)}
}

\newglossaryentry{ma-hoa-giai-ma}
{
    name={mã hoá-giải mã},
    description={(encoder-decoder)}
}

\newglossaryentry{mo-hinh-hon-hop}
{
    name={mô hình hỗn hợp},
    description={(mixture model)}
}

\newglossaryentry{mo-hinh-hon-hop-Gauss}
{
    name={mô hình hỗn hợp Gauss},
    description={(Gaussian mixture model)}
}

\newglossaryentry{quy-tac-Bayes}
{
    name={quy tắc Bayes},
    description={(Bayes' rule)}
}

\newglossaryentry{thu-nho-mau}
{
    name={thu nhỏ mẫu},
    description={(downsampling). Ngược lại sẽ là \gls{phong-to-mau}}
}

\newglossaryentry{phong-to-mau}
{
    name={phóng to mẫu},
    description={(upsampling)}
}

\newglossaryentry{phep-gop}
{
    name={phép gộp},
    description={(pooling)}
}

\newglossaryentry{phep-gop-cuc-dai}
{
    name={phép gộp cực đại},
    description={(max pooling)}
}

\newglossaryentry{phep-toan-gop-cuc-dai}
{
    name={phép toán gộp cực đại},
    description={(max pooling operation)}
}

\newglossaryentry{ham-gop-cuc-dai}
{
    name={hàm gộp cực đại},
    description={(max pooling function)}
}

\newglossaryentry{phep-gop-cuc-dai-xac-suat}
{
    name={phép gộp cực đại xác suất},
    description={(probabilistic max pooling)}
}

\newglossaryentry{phep-gop-trung-binh}
{
    name={phép gộp trung bình},
    description={(average pooling)}
}

\newglossaryentry{phep-gop-ngau-nhien}
{
    name={phép gộp ngẫu nhiên},
    description={(stochastic pooling)}
}

\newglossaryentry{dac-trung-nhi-phan}
{
    name={đặc trưng nhị phân},
    description={(binary feature)}
}

\newglossaryentry{bo-do-dac-trung}
{
    name={bộ dò đặc trưng},
    description={(feature detector)}
}

\newglossaryentry{bo-trich-xuat-dac-trung}
{
    name={bộ trích xuất đặc trưng},
    description={(feature extractor)}
}

\newglossaryentry{bo-trich-xuat-dac-trung-cham-phi-tuyen-da-tang}
{
    name={bộ trích xuất đặc trưng chậm phi tuyến đa tầng},
    description={(deep nonlinear slow featature extractor)}
}

\newglossaryentry{xap-xi-pho-quat}
{
    name={xấp xỉ phổ quát},
    description={(universal approximation)}
}

\newglossaryentry{dinh-li-xap-xi-pho-quat}
{
    name={định lí xấp xỉ phổ quát},
    description={(universal approximation theorem)}
}

\newglossaryentry{bo-xap-xi-pho-quat}
{
    name={bộ xấp xỉ phổ quát},
    description={(universal approximator)}
}

\newglossaryentry{mang-tong-tich}
{
    name={mạng tổng-tích},
    description={(sum-product network - SPN)}
}

\newacronym{spn}{SPN}{sum-product network}

\newglossaryentry{ham-co-so-xuyen-tam}
{
    name={hàm cơ sở xuyên tâm},
    description={(radial basis function)}
}

\newacronym{rbf}{RBF}{radial basis function}

\newglossaryentry{mang-ham-co-so-xuyen-tam}
{
    name={mạng hàm cơ sở xuyên tâm},
    description={(radial basis function network)}
}

\newglossaryentry{phan-tich-dac-trung-cham}
{
    name={phân tích đặc trưng chậm},
    description={(slow feature analysis)}
}

\newacronym{sfa}{SFA}{slow feature analysis}

\newglossaryentry{hieu-trung-binh-binh-phuong}
{
    name={hiệu trung bình bình phương},
    description={(mean squared difference)}
}

\newglossaryentry{mach-da-thuc}
{
    name={mạch đa thức},
    description={(polynomial circuit)}
}

\newglossaryentry{mach-da-tang}
{
    name={mạch đa tầng},
    description={(deep circuit)}
}

\newglossaryentry{thong-tin-tuong-ho}
{
    name={thông tin tương hỗ},
    description={(mutual information)}
}

\newglossaryentry{lan-truyen-tiep-tuyen}
{
    name={lan truyền tiếp tuyến},
    description={(tangent propagation)}
}

\newglossaryentry{khoang-cach-tiep-tuyen}
{
    name={khoảng cách tiếp tuyến},
    description={(tangent distance)}
}

\newglossaryentry{lan-truyen-nguoc-kep}
{
    name={lan truyền ngược kép},
    description={(double backprop)}
}

\newglossaryentry{lan-truyen-nguoc-thoi-gian}
{
    name={lan truyền ngược thời gian},
    description={(back-propagation through time)}
}

\newacronym{bptt}{BPTT}{back-propagation through time}

\newglossaryentry{lan-truyen-nguoc}
{
    name={lan truyền ngược},
    description={(back-propagation)}
}

\newglossaryentry{mang-nguong-tuyen-tinh-da-tang}
{
    name={mạng ngưỡng tuyến tính đa tầng},
    description={(deep linear-threshold network)}
}

\newglossaryentry{can-duoi-bien-phan}
{
    name={cận dưới biến phân},
    description={(variational lower bound)}
}

\newglossaryentry{phuong-trinh-diem-co-dinh}
{
    name={phương trình điểm cố định},
    description={(fixed-point equation)}
}

\newglossaryentry{phiem-ham}
{
    name={phiếm hàm},
    description={(functional)}
}

\newglossaryentry{phiem-ham-chi-phi}
{
    name={phiếm hàm chi phí},
    description={(cost functional)}
}

\newglossaryentry{trung-binh-dong}
{
    name={trung bình động},
    description={(moving average hoặc running average)}
}

\newglossaryentry{nhan-dang-tieng-noi}
{
    name={nhận dạng tiếng nói},
    description={(speech recognition)}
}

\newglossaryentry{nhan-dang-vat-the}
{
    name={nhận dạng vật thể},
    description={(object recognition)}
}

\newglossaryentry{co-so-tri-thuc}
{
    name={cơ sở tri thức},
    description={(knowledge base)}
}

\newglossaryentry{he-phan-cap-khai-niem}
{
    name={hệ phân cấp khái niệm},
    description={(hierarchy of concepts)}
}

\newglossaryentry{bieu-dien}
{
    name={biểu diễn},
    description={(representation)}
}

\newglossaryentry{mo-thuc}
{
    name={mô thức},
    description={(pattern)}
}

\newglossaryentry{dac-trung}
{
    name={đặc trưng},
    description={(feature)}
}

\newglossaryentry{diem-anh}
{
    name={điểm ảnh},
    description={(pixel)}
}

\newglossaryentry{boc-tach}
{
    name={bóc tách},
    description={(disentangle)}
}

\newglossaryentry{luu-do}
{
    name={lưu đồ},
    description={(flow chart)}
}

\newglossaryentry{tang-bieu-dien}
{
    name={tầng biểu diễn},
    description={(representation layer)}
}

\newglossaryentry{phuong-phap-tinh}
{
    name={phương pháp tính},
    description={(calculus)}
}

\newglossaryentry{phuong-phap-tinh-bien-phan}
{
    name={phương pháp tính biến phân},
    description={(calculus of variations)}
}

\newglossaryentry{mang-neuron-nhan-tao}
{
    name={mạng neuron nhân tạo},
    description={(artificial neuron network - ANN)}
}

\newglossaryentry{mang-cong-chan}
{
    name={mạng cổng chặn},
    description={(gater (network))}
}

\newglossaryentry{thi-giac-may-tinh}
{
    name={thị giác máy tính},
    description={(computer vision)}
}

\newglossaryentry{phan-vung-anh}
{
    name={phân vùng ảnh},
    description={(image segmentation)}
}

\newglossaryentry{luoi-chinh-quy}
{
    name={lưới chính quy},
    description={(regular grid)}
}

\newglossaryentry{dang-huong}
{
    name={đẳng hướng},
    description={(isotropic)}
}

\newglossaryentry{diem-dung}
{
    name={điểm dừng},
    description={(stationary point)}
}

\newglossaryentry{leo-gradient}
{
    name={leo gradient},
    description={(gradient ascent)}
}

\newglossaryentry{leo-gradient-ngau-nhien}
{
    name={leo gradient ngẫu nhiên},
    description={(stochastic gradient ascent)}
}

\newglossaryentry{khong-gian-gia-thuyet}
{
    name={không gian giả thuyết},
    description={(hypothesis space)}
}

\newglossaryentry{tong-hop-tu-luc}
{
    name={tổng hợp tự lực},
    description={(bootstrap aggregating - BAGGING)}
}

\newglossaryentry{lua-chon-dac-trung}
{
    name={lựa chọn đặc trưng},
    description={(feature selection)}
}

\newglossaryentry{trich-xuat-dac-trung}
{
    name={trích xuất đặc trưng},
    description={(feature extraction)}
}

\newglossaryentry{tang-cuong-du-lieu}
{
    name={tăng cường dữ liệu},
    description={(dataset augmentation)}
}

\newglossaryentry{gradient-doi-nghich}
{
    name={gradient đối nghịch},
    description={(negative gradient)}
}

\newglossaryentry{phuong-phap-gradient-tang-toc}
{
    name={phương pháp gradient tăng tốc},
    description={(accelerated gradient method)}
}

\newglossaryentry{khong-gian-rong}
{
    name={không gian rỗng},
    description={(null space)}
}

\newglossaryentry{tang-ket-noi-day-du}
{
    name={tầng kết nối đầy đủ},
    description={(fully connected layer)}
}

\newglossaryentry{buoc-ngau-nhien}
{
    name={bước ngẫu nhiên},
    description={(random walk)}
}

\newglossaryentry{dung-chung-tham-so}
{
    name={dùng chung tham số},
    description={(parameter sharing)}
}

\newglossaryentry{troi-buoc-tham-so}
{
    name={trói buộc tham số},
    description={(parameter tying)}
}

\newglossaryentry{truong-tiep-nhan}
{
    name={trường tiếp nhận},
    description={(receptive field)}
}

\newglossaryentry{tich-chap-sai}
{
    name={tích chập sải},
    description={(strided convolution)}
}

\newglossaryentry{sai-chap}
{
    name={sải chập},
    description={(stride)}
}

\newglossaryentry{cau-truc-to-po}
{
    name={cấu trúc tô-pô},
    description={(topology)}
}

\newglossaryentry{tich-chap-xen-ke}
{
    name={tích chập xen kẽ},
    description={(tiled convolution)}
}

\newglossaryentry{cap-cau-phuong}
{
    name={cặp cầu phương},
    description={(quadrature pair)}
}

\newglossaryentry{mang-neuron-truy-hoi}
{
    name={mạng neuron truy hồi},
    description={(recurrent neural network)}
}

\newacronym{rnn}{RNN}{recurrent neural network}

\newglossaryentry{mang-neuron-truy-hoi-co-cong}
{
    name={mạng neuron truy hồi có cổng},
    description={(gated recurrent neural network)}
}

\newglossaryentry{don-vi-truy-hoi-co-cong}
{
    name={đơn vị truy hồi có cổng},
    description={(gated recurrent unit)}
}

\newacronym{gru}{GRU}{gated recurrent unit}

\newglossaryentry{may-vector-ho-tro}
{
    name={máy vector hỗ trợ},
    description={(support vector machine)}
}

\newacronym{svm}{SVM}{support vector machine}

\newglossaryentry{mang-de-quy}
{
    name={mạng đệ quy},
    description={(recursive network)}
}

\newglossaryentry{buoc-thoi-gian}
{
    name={bước thời gian},
    description={(time step)}
}

\newglossaryentry{RNN-song-huong}
{
    name={RNN song hướng},
    description={(Bidirectional RNN)}
}

\newglossaryentry{co-che-chu-y}
{
    name={cơ chế chú ý},
    description={(attention mechanism)}
}

\newglossaryentry{song-tuyen-tinh}
{
    name={song tuyến tính},
    description={(bilinear)}
}

\newglossaryentry{ban-xac-dinh-duong}
{
    name={bán xác định dương},
    description={(positive semidefinite)}
}

\newglossaryentry{ban-xac-dinh-am}
{
    name={bán xác định âm},
    description={(negative semidefinite)}
}

\newglossaryentry{toan-tu-vet}
{
    name={toán tử vết},
    description={(trace operator)}
}

\newglossaryentry{hoan-vi-tuan-hoan}
{
    name={hoán vị tuần hoàn},
    description={(circular permutation)}
}

\newglossaryentry{truy-van-can-thiep}
{
    name={truy vấn can thiệp},
    description={(intervention query)}
}

\newglossaryentry{sai-so-khai-quat-hoa}
{
    name={sai số khái quát hóa},
    description={(generalization error)}
}

\newglossaryentry{trung-binh-sai-so-tuyet-doi}
{
    name={trung bình sai số tuyệt đối},
    description={(mean absolute error)}
}

\newacronym{mae}{MAE}{mean absolute error}

\newglossaryentry{hoc-hop-li-cuc-dai}
{
    name={học hợp lí cực đại},
    description={(maximum likelihood learning)}
}

\newglossaryentry{ham-tuyen-tinh-tung-doan}
{
    name={hàm tuyến tính từng đoạn},
    description={(piecewise linear function)}
}

\newglossaryentry{don-vi-tuyen-tinh-tung-doan}
{
    name={đơn vị tuyến tính từng đoạn},
    description={(piecewise linear unit)}
}

\newglossaryentry{tuyen-tinh-tung-doan}
{
    name={tuyến tính từng đoạn},
    description={(piecewise linear)}
}

\newglossaryentry{khop-ban-mau}
{
    name={khớp bản mẫu},
    description={(template matching)}
}

\newglossaryentry{tinh-toan-bieu-tuong}
{
    name={tính toán biểu tượng},
    description={(symbolic computation)}
}

\newglossaryentry{tich-luy-che-do-nguoc}
{
    name={tích lũy chế độ ngược},
    description={(reverse mode accumulation)}
}

\newglossaryentry{tich-luy-che-do-thuan}
{
    name={tích lũy chế độ thuận},
    description={(forward mode accumulation)}
}

\newglossaryentry{hoi-quy-ngon-song}
{
    name={hồi quy ngọn sóng},
    description={(ridge regression)}
}

\newglossaryentry{khang-nhieu}
{
    name={kháng nhiễu},
    description={(robust to noise, noise robustness)}
}

\newglossaryentry{theo-duoi-doi-sanh-truc-giao}
{
    name={theo đuổi đối sánh trực giao},
    description={(orthogonal matching pursuit)}
}

\newacronym{omp}{OMP}{Orthogonal Matching Pursuit}

\newglossaryentry{trung-binh-hoa-mo-hinh}
{
    name={trung bình hóa mô hình},
    description={(model averaging)}
}

\newglossaryentry{co-gian-trong-so}
{
    name={co giãn trọng số},
    description={(weight scaling)}
}

\newglossaryentry{ngat-ket-noi}
{
    name={ngắt kết nối},
    description={(DropConnect)}
}

\newglossaryentry{don-mode}
{
    name={đơn mode},
    description={(unimodal)}
}

\newglossaryentry{da-mode}
{
    name={đa mode},
    description={(multimodal)}
}

\newglossaryentry{hoi-quy-da-mode}
{
    name={hồi quy đa mode},
    description={(multimodal regression)}
}

\newglossaryentry{tro-choi-co-tong-bang-khong}
{
    name={trò chơi có tổng bằng không},
    description={(zero-sum game)}
}

\newglossaryentry{giam-xoc}
{
    name={giảm xóc},
    description={(damping)}
}

\newglossaryentry{tim-kiem-dau-dac-trung}
{
    name={tìm kiếm dấu đặc trưng},
    description={(feature-sign search)}
}

\newglossaryentry{leo-theo-toa-do}
{
    name={leo theo tọa độ},
    description={(coordinate ascent)}
}

\newglossaryentry{kho-tinh-toan}
{
    name={khó tính toán},
    description={(intractable)}
}

\newglossaryentry{PCA-huong-xac-suat}
{
    name={PCA hướng xác suất},
    description={(probabilistic PCA)}
}

\newglossaryentry{dieu-hoa-song-song}
{
    name={điều hòa song song},
    description={(parallel tempering)}
}

\newglossaryentry{chuyen-tiep-dieu-hoa}
{
    name={chuyển tiếp điều hòa},
    description={(tempered transition)}
}

\newglossaryentry{phan-phoi-can-bang}
{
    name={phân phối cân bằng},
    description={(equilibrium distribution)}
}

\newglossaryentry{phan-phoi-dung}
{
    name={phân phối dừng},
    description={(stationary distribution)}
}

\newglossaryentry{tach-biet-d}
{
    name={tách-biệt-D},
    description={(D-seperation)}
}

\newglossaryentry{ai-luc}
{
    name={ái lực},
    description={(affinity)}
}

\newglossaryentry{cuong-do-thong-ke}
{
    name={cường độ thống kê},
    description={(statistical strength)}
}

\newglossaryentry{phan-tich-thanh-phan-doc-lap}
{
    name={phân tích thành phần độc lập},
    description={(independent component analysis - ICA)}
}

\newglossaryentry{chinh-sach}
{
    name={chính sách},
    description={(policy)}
}

\newglossaryentry{khai-pha}
{
    name={khai phá},
    description={(explore)}
}

\newglossaryentry{khai-thac}
{
    name={khai thác},
    description={(exploit)}
}

\newglossaryentry{ma-tran-dich-thuat}
{
    name={ma trận dịch thuật},
    description={(translation matrix)}
}

\newglossaryentry{vector-tu-xuyen-ngon-ngu}
{
    name={vector từ xuyên ngôn ngữ},
    description={(cross-lingual word vector)}
}

\newglossaryentry{can-chinh-xuyen-ngon-ngu}
{
    name={căn chỉnh xuyên ngôn ngữ},
    description={(cross-lingual alignment)}
}

\newglossaryentry{khong-tim-thay-trong-bo-dem}
{
    name={không tìm thấy trong bộ đệm},
    description={(cache-miss)}
}

\newglossaryentry{bo-dem}
{
    name={bộ đệm},
    description={(cache). Còn gọi là ``bộ nhớ đệm''}
}

\newglossaryentry{giao-tac-bo-nho}
{
    name={giao tác bộ nhớ},
    description={(memory transaction)}
}

\newglossaryentry{hop-nhat}
{
    name={hợp nhất},
    description={(coalesce)}
}

\newglossaryentry{song-song-du-lieu}
{
    name={song song dữ liệu},
    description={(data parallelism)}
}

\newglossaryentry{song-song-mo-hinh}
{
    name={song song mô hình},
    description={(model parallelism)}
}

\newglossaryentry{do-nhay}
{
    name={độ nhạy},
    description={(recall)}
}

\newglossaryentry{chuoi-phan-tang}
{
    name={chuỗi phân tầng},
    description={(cascade)}
}

\newglossaryentry{tui-tu}
{
    name={túi từ},
    description={(bag of words)}
}

\newglossaryentry{bang-bam}
{
    name={bảng băm},
    description={(hash table)}
}

\newglossaryentry{bam-ngu-nghia}
{
    name={băm ngữ nghĩa},
    description={(semantic hashing)}
}

\newacronym{asr}{ASR}{Automatic Speech Recognition}

\newacronym{ctc}{CTC}{Connectionist Temporal Classification}

\newacronym{ebm}{EBM}{Energy-Based Model}

\newacronym{em}{EM}{Expectation Maximization}

\newacronym{gcn}{GCN}{Global Contrast Normalization}

\newacronym{gpu}{GPU}{Graphics Processing Unit}

\newacronym{gp-gpu}{GP-GPU}{General Purpose Graphics Processing Unit}

\newacronym{ilsvrc}{ILSVRC}{ImageNet Large Scale Visual Recognition Challenge}

\newacronym{kkt}{KKT}{Karush-Kuhn-Tucker}

\newacronym{kl}{KL}{Kullback-Leibler}

\newacronym{iid}{i.i.d}{Independent and identically distributed}

\newacronym{mcmc}{MCMC}{Markov chain Monte Carlo method}

\newacronym{nade}{NADE}{neural auto-regressive density estimator}

\newacronym{draw}{DRAW}{deep recurrent attention writer}

\newacronym{nlm}{NLM}{neural language model}

\newacronym{nlp}{NLP}{Natural Language Processing}

\newacronym{reinforce}{REINFORCE}{REward Increment = nonnegative Factor x Offset
reinforcement x Characteristic Eligibility}

\newacronym{ica}{ICA}{Independent Component Analysis}

\newacronym{nice}{NICE}{nonlinear independent components estimation}

\newacronym{vc}{VC}{Vapnik-Chervonenkis}

\newacronym{lst}{LST}{liquid state machine}

\newacronym{abc}{ABC}{approximate Bayesian computation}

\newacronym{pot}{PoT}{product of Student t-distribution}

\newacronym{mpot}{mPoT}{mean product of Student t-distribution}

\newacronym{pdf}{PDF}{probability density function}

\newacronym{hmm}{HMM}{hidden Markov model}

\newacronym{gmm}{GMM}{Gaussian mixture model}

\newacronym{psd}{PSD}{predictive sparse decomposition}

\newacronym{ista}{ISTA}{Iterative Shrinkage and Thresholding Algorithm}

\newacronym{ai}{AI}{artificial intelligence}

\newacronym{it}{IT}{inferotemporal cortex / inferior temporal cortex}

\newacronym{cpu}{CPU}{Central Processing Unit}

\newacronym{relu}{ReLU}{rectified linear unit}

\newacronym{prelu}{PReLU}{parametric rectified linear unit}

\newacronym{mmd}{MMD}{maximum mean discrepancy}

\newacronym{fvbn}{FVBN}{fully-visible Bayes network}

\newacronym{rnade}{RNADE}{real-valued neural autoregressive density-estimator}

\newglossaryentry{toan-hoc-co-dien}
{
    name={toán học cổ điển},
    description={(classical mathematics)}
}

\newglossaryentry{nen-tang}
{
    name={nền tảng},
    description={(foundation)}
}

\newglossaryentry{cnlg}
{
    name={chủ nghĩa logic},
    description={(logicism)}
}

\newglossaryentry{cntg}
{
    name={chủ nghĩa trực giác},
    description={(intuitionism)}
}

\newglossaryentry{cnht}
{
    name={chủ nghĩa hình thức},
    description={(formalism)}
}

\newglossaryentry{truong-phai}
{
    name={trường phái},
    description={(school)}
}

\newglossaryentry{cn-hien-thuc}
{
    name={chủ nghĩa hiện thực},
    description={(realism)}
}

\newglossaryentry{mau-thuan}
{
    name={mâu thuẫn},
    description={(contradition)}
}

\newglossaryentry{li-thuyet-tap-hop}
{
    name={lí thuyết tập hợp},
    description={(set theory)}
}

\newglossaryentry{tien-de}
{
    name={tiên đề},
    description={(axiom)}
}

\newglossaryentry{tien-de-khung}
{
    name={tiên đề khung},
    description={(axiom schema)}
}

\newglossaryentry{ki-thuat-toan-hoc}
{
    name={kĩ thuật toán học},
    description={(technical mathematics)}
}

\newglossaryentry{sieu-hinh}
{
    name={siêu hình},
    description={(metaphysical)}
}

\newglossaryentry{triet-hoc}
{
    name={triết học},
    description={(philosophy)}
}

\newglossaryentry{logic-thuan-tuy}
{
    name={logic thuần túy},
    description={(pure logic)}
}

\newglossaryentry{logic-co-dien}
{
    name={logic cổ điển},
    description={(classical logic)}
}

\newglossaryentry{ngon-ngu-bac-nhat}
{
    name={ngôn ngữ bậc nhất},
    description={(first order language)}
}

\newglossaryentry{logic-bac-nhat}
{
    name={logic bậc nhất},
    description={(first order language), còn có cách dịch khác là \textbf{logic vị từ}}
}

\newglossaryentry{phi-logic}
{
    name={phi logic},
    description={(non-logical)}
}

\newglossaryentry{quy-tac-suy-luan}
{
    name={quy tắc suy luận},
    description={(deduction rule)}
}

\newglossaryentry{hoc-thuyet}
{
    name={học thuyết},
    description={(doctrine)}
}

\newglossaryentry{thuc-the}
{
    name={thực thể},
    description={(entity)}
}

\newglossaryentry{tap-vo-han}
{
    name={tập vô hạn},
    description={(infinite set))}
}

\newglossaryentry{menh-de}
{
    name={mệnh đề},
    description={(proposition))}
}

\newglossaryentry{menh-de-logic}
{
    name={mệnh đề logic},
    description={(proposition))}
}

\newglossaryentry{dinh-li}
{
    name={định lí},
    description={(theorem))}
}

\newglossaryentry{nghich-li}
{
    name={nghịch lí},
    description={(paradox))}
}

\newglossaryentry{luat-bai-trung}
{
    name={luật bài trung},
    description={(law of excluded middle)}
}

\newglossaryentry{pho-quat-tuyet-doi}
{
    name={phổ quát tuyệt đối},
    description={(complete generality)}
}

\newglossaryentry{hinh-thuc}
{
    name={hình thức},
    description={(form)}
}

\newglossaryentry{noi-dung}
{
    name={nội dung},
    description={(content)}
}

\newglossaryentry{dang-cu-phap}
{
    name={dạng cú pháp},
    description={(syntactical form)}
}

\newglossaryentry{truc-giac-nguyen-thuy}
{
    name={trực giác nguyên thủy},
    description={(primordial intuition)}
}

\newglossaryentry{nhan-thuc-tuc-thi}
{
    name={nhận thức tức thì},
    description={(immediate awareness)}
}

\newglossaryentry{tap-dong}
{
    name={tập đóng},
    description={(close set)}
}

\newglossaryentry{tinh-quy-nap}
{
    name={tính quy nạp},
    description={(inductive)}
}

\newglossaryentry{tinh-hieu-dung}
{
    name={tính hiệu dụng},
    description={(effective)}
}

\newglossaryentry{tinh-xay-dung}
{
    name={tính xây dựng},
    description={(constructive)}
}

\newglossaryentry{cnkn}
{
    name={chủ nghĩa khái niệm},
    description={(conceptualism)}
}

\newglossaryentry{toan-ven}
{
    name={toàn vẹn},
    description={(entirety)}
}

\newglossaryentry{tien-de-hoa}
{
    name={tiên đề hóa},
    description={(axiomatize, axiomatization)}
}

\newglossaryentry{hinh-thuc-hoa}
{
    name={hình thức hóa},
    description={(formalize, formalization)}
}

\newglossaryentry{ki-hieu}
{
    name={kí hiệu},
    description={(symbol)}
}

\newglossaryentry{li-thuyet-da-tien-de-hoa}
{
    name={lí thuyết đã tiên đề hóa},
    description={(axiomatized theory)}
}

\newglossaryentry{chuong-trinh-hilbert}
{
    name={chương trình Hilbert},
    description={(Hilbert program)}
}

\newglossaryentry{chu-nghia-duy-danh}
{
    name={chủ nghĩa duy danh},
    description={(nominalism)}
}

\newglossaryentry{ngon-ngu-tu-nhien}
{
    name={ngôn ngữ tự nhiên},
    description={(natural language)}
}

\newglossaryentry{lap-luan-huu-han}
{
    name={lập luận hữu hạn},
    description={(finitary reasoning)}
}

\newacronym{asr}{ASR}{Automatic Speech Recognition}

\newglossaryentry{nguoi-dung}
{
    name={người dùng},
    description={(user)}
}

\newglossaryentry{giao-dien-do-hoa}
{
    name={giao diện đồ họa},
    description={(Graphical User Interface - GUI)}
}

\newglossaryentry{giao-dien-dong-lenh}
{
    name={giao diện dòng lệnh},
    description={(Console User Interface - CUI)}
}

\newglossaryentry{khoang-trang}
{
    name={khoảng trắng},
    description={(space)}
}

\newglossaryentry{ki-tu}
{
    name={kí tự},
    description={(character)}
}

\newglossaryentry{doi-tuong}
{
    name={đối tượng},
    description={(object)}
}

\newglossaryentry{thuc-the}
{
    name={thực thể},
    description={(instance)}
}

\newglossaryentry{phuong-thuc}
{
    name={phương thức},
    description={(method)}
}

\newglossaryentry{thuoc-tinh}
{
    name={thuộc tính},
    description={(attribute)}
}

\newglossaryentry{lop-doi-tuong}
{
    name={lớp đối tượng},
    description={(class)}
}

\newglossaryentry{su-kien}
{
    name={sự kiện},
    description={(event)}
}

\newglossaryentry{luong}
{
    name={luồng},
    description={(thread)}
}

\newglossaryentry{da-luong}
{
    name={đa luồng},
    description={(multi-thread)}
}

\newglossaryentry{khoi-hinh}
{
    name={khối hình},
    description={(block)}
}

\newglossaryentry{chong-lan}
{
    name={chồng lấn},
    description={(collision)}
}

\newglossaryentry{bien-the-xoay}
{
    name={biến thể xoay},
    description={(rotation variation)}
}

\newglossaryentry{khung-hinh}
{
    name={khung hình},
    description={(frame)}
}

\newglossaryentry{khung-hinh-nen}
{
    name={khung hình nền},
    description={(background-frame)}
}

\newglossaryentry{khoi-vuong-co-ban}
{
    name={khối vuông cơ bản},
    description={(basic block)}
}

\newglossaryentry{truu-tuong}
{
    name={trừu tượng},
    description={(abstract)}
}

\newglossaryentry{thuc-thi}
{
    name={thực thi},
    description={(implement)}
}

\newglossaryentry{tuong-tranh}
{
    name={tương tranh},
    description={(race-condition)}
}

\newglossaryentry{co-che-huong-su-kien}
{
    name={cơ chế hướng sự kiện},
    description={(event-orientated mechanism)}
}

\newglossaryentry{ma-nguon}
{
    name={mã nguồn},
    description={(source-code)}
}

\newglossaryentry{go-loi}
{
    name={gõ lỗi},
    description={(debug)}
}

\newacronym{asr}{ASR}{Automatic Speech Recognition}